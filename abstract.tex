\chapter*{\centering Abstract}
\addcontentsline{toc}{chapter}{Abstract}

Distances arise in a wide variety of different contexts, one of which is
partitional clustering, that is, the problem of finding groups of similar
objects within a set of objects.  These groups are seemingly very easy to find
for humans, but very difficult to find for machines as there are two major
difficulties to be overcome: the first defining an objective criterion for the
vague notion of ``groups of similar objects'', and the second is the
computational complexity of finding such groups given a criterion.  In the
first part of this thesis, we focus on the first difficulty and show that even
seemingly similar optimisation criteria used for partitional clustering can
produce vastly different results.  In the process of showing this we develop a
new metric for comparing clustering solutions called the assignment metric.
We then prove some new NP-completeness results for problems using two related
``sum-of-squares'' clustering criteria.

Closely related to partitional clustering is the problem of hierarchical
clustering.  We extend and formalise this problem to the problem of
constructing rooted edge-weighted \textit{$X$-trees}, that is trees with a
leafset $X$. It is well known that an $X$-tree can be uniquely reconstructed
from a distance on $X$ if the distance is an ultrametric.  But in practice the
complete distance on $X$ may not always be available.  In the second part of
this thesis we look at some of the circumstances under which a tree can be
uniquely reconstructed from incomplete distance information.  We use a concept
called a \textit{lasso} and give some theoretical properties of a special type
of lasso.  We then develop an algorithm which can construct a tree together
with a lasso from partial distance information and show how this can be
applied to various incomplete datasets.
