\documentclass{article}

\usepackage[utf8]{inputenc}
\usepackage[T2A,T1]{fontenc}
\usepackage{lmodern}
\usepackage[english]{babel}
\usepackage{geometry}
\usepackage{setspace}
\usepackage[pdftex]{graphicx,color}
\usepackage{tikz}
\usepackage{pgfplots}
\usepackage{booktabs}
\usepackage[format=hang,font=small,labelfont=bf]{caption}
\usepackage{subcaption}
\usepackage[protrusion=true,expansion=true]{microtype}
\usepackage{amsmath}
\usepackage{amssymb}
\usepackage{amsthm}
\usepackage{upgreek}
\usepackage{nbaseprt}
\usepackage{appendix}
\usepackage{url}
\usepackage{hyperref}
\usepackage{underscore}
\usepackage[numbers]{natbib}
\usepackage{bibentry}
\nobibliography*
\usepackage{enumitem}
\usepackage{algorithm}
\usepackage{algorithmicx}
\usepackage{algpseudocode}

% \renewcommand{\baselinestretch}{1.62}
\onehalfspacing

\newcommand{\reporttitle}{New Algorithms and Methodology for Analysing
  Distances}
\newcommand{\reportauthor}{George Kettleborough}

\usepackage[nottoc,numbib]{tocbibind}

% --- hyperref stuff ---
\definecolor{darkblue}{rgb}{0,0,0.4}
\hypersetup{
  pdftex,
  bookmarks=true,
  bookmarksopen=true,
  colorlinks=true,
  citecolor=black,
  filecolor=black,
  linkcolor=black,
  urlcolor=darkblue,
  pdfauthor={\reportauthor},
  pdftitle={\reporttitle},
  pdfsubject={}
}
\providecommand{\doi}[1]{\href{http://dx.doi.org/#1}{doi: #1}}

% --- languages ---
% \newcommand{\fr}[1]{{\begin{otherlanguage*}{french} #1 \end{otherlanguage*}}}
% \newcommand{\ru}[1]{{\begin{otherlanguage*}{russian} #1 \end{otherlanguage*}}}

% --- TikZ stuff ---

\usetikzlibrary{calc,trees,positioning,arrows,chains,shapes.geometric,%
  decorations.pathreplacing,decorations.pathmorphing,shapes,%
  matrix,shapes.symbols}

\tikzstyle{dot}=[circle,fill,inner sep=0,minimum size=1ex]
\tikzstyle{dist}=[>=stealth,->,shorten >=2pt]
\tikzstyle{symdist}=[>=stealth,<->,shorten <=2pt,shorten >=2pt]
\tikzstyle{faint}=[gray!50]
\tikzstyle{clus}=[circle,draw,inner sep=0,minimum size=5mm]
\tikzstyle{setnode1}=[dot,regular polygon,regular polygon sides=3,
      minimum size=1.2ex,fill=black]
\tikzstyle{setnode2}=[dot,regular polygon,regular polygon sides=3,
      minimum size=1.2ex,fill=black,rotate=180]

\DeclareRobustCommand\tikzuptriangle{\tikz \node [setnode1] at (0,0) {};}
\DeclareRobustCommand\tikzupdtriangle{\tikz \node [setnode2] at (0,0) {};}
\DeclareRobustCommand\tikzbotriangle{\tikz{\node [setnode1] at (0,0) {};\node [setnode2] at (0,0) {};}}

% --- End TikZ stuff ---

% --- Subfigures ---

\DeclareCaptionSubType*{figure}
\captionsetup[subfigure]{labelfont=normalfont,textfont=normalfont}
\renewcommand\thesubfigure{\roman{subfigure}}

% make equations and figures numbered (sec.eq)
% \numberwithin{equation}{section}
% \numberwithin{figure}{section}

% new operators for maths
\DeclareMathOperator{\op}{op}
\DeclareMathOperator{\remainder}{remainder}
\DeclareMathOperator{\lc}{lc}
\DeclareMathOperator{\pquo}{pquo}
\DeclareMathOperator{\prem}{prem}
\DeclareMathOperator{\pp}{pp}
\DeclareMathOperator{\cont}{cont}
\DeclareMathOperator{\resultant}{resultant}
\DeclareMathOperator{\numerator}{numerator}
\DeclareMathOperator{\denominator}{denominator}
\DeclareMathOperator{\ADCO}{ADCO}
\DeclareMathOperator{\dens}{dens}
\DeclareMathOperator{\symdif}{\bigtriangleup}
\DeclareMathOperator*{\argmin}{arg\,min}
\DeclareMathOperator*{\argmax}{arg\,max}
\DeclareMathOperator*{\tr}{tr}
\DeclareMathOperator{\lca}{lca}
\DeclareMathOperator{\height}{height}
\DeclareMathOperator*{\Block}{Block}
\DeclareMathOperator*{\Cut}{Cut}

% use bold vectors
\let\oldhat\hat
\renewcommand{\vec}[1]{\mathbf{#1}}
\renewcommand{\hat}[1]{\oldhat{\mathbf{#1}}}

% -------------------- convenience things --------------------

\newcommand{\0}{{\emptyset}}
\newcommand{\rk}{{\rm rk}}
%\newcommand{\dim}{{\rm dim}}
\newcommand{\rr}{{\mathbb R}}
\newcommand{\rrnn}{{\mathbb R}_{\geq 0}}
\newcommand{\cq}{{\mathcal Q}}
\newcommand{\cp}{{\mathcal P}}
\newcommand{\cd}{{\mathcal D}}
\newcommand{\cc}{{\mathcal C}}
\newcommand{\cs}{{\mathcal S}}
\newcommand{\rM}{{\mathbb M}}
\newcommand{\sgn}{{\rm sgn}}
\newcommand{\supp}{{\rm supp}}
\newcommand{\ct}{{\mathcal T}}
\newcommand{\cl}{{\mathcal L}}
\newcommand{\s}{{\textfrak c}}
\newcommand{\ch}{{X \choose 2}}
\newcommand{\E}{{\mathcal E}}
\newcommand{\nn}{{\mathbb N}}
\newcommand{\F}{{\mathbb F}}
\newcommand{\RR}{{\mathbb R}}
\newcommand{\cA}{{\mathcal A}}
\newcommand{\cQ}{{\mathcal Q}}
\newcommand{\cP}{{\mathcal P}}
\newcommand{\cD}{{\mathcal D}}
\newcommand{\cC}{{\mathcal C}}
\newcommand{\cS}{{\mathcal S}}
\newcommand{\cL}{{\mathcal L}}
\newcommand{\cT}{{\mathcal T}}
%\newcommand{\cl}{{\mathcal B}}
\newcommand{\ra}{{\rightarrow}}
\newcommand{\Ra}{{\Rightarrow}}
\newcommand{\iV}{\mathring{V}}
\newcommand{\DTw}{D_{(T,\omega)}}
\newcommand{\AcL}{A_{\cL}}
\newcommand{\BecL}{\beta_{\cL}}

%\newcommand{\pf}{\noindent{\em Proof: }}
%\newcommand{\epf}{\hfill\hbox{\rule{3pt}{6pt}}\\}


\newcommand{\dset}{D}
\newcommand{\clus}{\mathcal{C}}
\newcommand{\parts}{\mathcal{P}}
\newcommand{\NP}{\text{NP}}

% for functions between two partitions C_1 and C_2
\newcommand{\partcompare}[1]{\Delta_{\mathcal{#1}}(\clus_1,\clus_2)}
\newcommand{\partcomparen}[1]{\Delta_{\mathcal{#1}}}
\newcommand{\partcomparest}[1]{\Delta_{\mathcal{#1}}(\clus^*_1,\clus^*_2)}
% for functions between two partitions C and C' and C and C''
\newcommand{\partcomparep}[1]{\Delta_{\mathcal{#1}}(\clus,\clus')}
\newcommand{\partcomparepp}[1]{\Delta_{\mathcal{#1}}(\clus,\clus'')}
% for definition of dissimilarity functions
\newcommand{\dissimdef}[1]{d_{#1} \colon M \times M \to \mathbb{R}^{\geq 0}}
\newcommand{\dissimdefn}[1]{d_{#1}}

% -------------------- pretty things --------------------

% works nicely for fraction indices
\newcommand{\prettyfrac}[2]{^#1\!/\!_#2}

% use this for functions on powersets
\newcommand{\psettimes}{\! \times}
% and this for functions on powersets of powersets
\newcommand{\pspsettimes}{\! \! \times}

% grey line numbers
\newcommand{\lnum}[1]{\textit{\color{gray}#1}}

% -------------------- environments --------------------

% theorems
\newtheorem{thm}{Theorem}
\newtheorem{lem}{Lemma}
\newtheorem{pro}{Proposition}
\newtheorem{cor}{Corollary}
\newtheorem{dfn}{Definition}

% problem template
\newenvironment{problem}[1]{\par\addvspace{\topsep}\noindent\underline{\textsc{#1}}\\}
{\par\addvspace{\topsep}\noindent}
\newcommand{\instance}[1]{\textsc{Instance:} #1\\}
\newcommand{\question}[1]{\textsc{Question:} #1}

% algorithmic stuff
\renewcommand{\algorithmicrequire}{\textbf{Input:}}
\renewcommand{\algorithmicensure}{\textbf{Output:}}

% list of algorithms code
% \makeatletter
% \AtBeginDocument{%
% \renewcommand{\listofalgorithms}{%
% \@cfttocstart \par \begingroup 
% \parindent\z@ \parskip\cftparskip 
% \addpenalty \@secpenalty 
% \if@cfthaschapter \vspace*{\cftbeforeloftitleskip } \else 
%                   \vspace {\cftbeforeloftitleskip } \fi
% \@cftpagestyle 
% {\interlinepenalty \@M 
%   {\cftloftitlefont \listalgorithmname }%
%   {\cftafterloftitle } 
% \@mkboth {\MakeUppercase \listalgorithmname }
%          {\MakeUppercase \listalgorithmname }
% \par \nobreak \vskip \cftafterloftitleskip \@afterheading }%
% \let\l@algorithm\l@figure
% \@starttoc {loa}\endgroup 
% \@cfttocfinish }}
% \makeatother

\begin{document}

% ---------- unrooted trees ----------

\subsubsection{Neighbor-joining}
\label{sec:neighbour-joining}

Neighbor-joining is a method for building an edge-weighted unrooted $X$-tree
which, like UPGMA, is consistent if the dissimilarity identifies a binary tree
but, unlike UPGMA, the consistency holds for general tree metrics
\cite{saitou1987nj}.  The algorithm works by replacing a cherry by a single
vertex and recursively applying Neighbor-joining to the resulting tree.
Neighbor-joining is shown in Algorithm~\ref{alg:neighbour-joining}.

\begin{algorithm}[h]
  \caption{Neighbor-joining.}
  \label{alg:neighbour-joining}

  \begin{algorithmic}
    \Require A set $X$, and a dissimilarity $d \colon X \times X \to \rr$.
    \Ensure  A binary $X$-tree $T$ and edge-weighting $\omega$.

    \If{$|X| = 2$} Let $X = \{x,y\}$ and \Return the tree obtained by joining
    $x$ and $y$ with an edge of length $d(x,y)$.
    \EndIf

    \State $\displaystyle (x,y) \gets \argmax_{x,y \in X} \left( d(x,y) + \frac{1}{2}
    \sum_{r \in X - \{x,y\}} (d(r,x)+d(r,y)-d(x,y))\right)$.

    \State $X' \gets X - \{x,y\} \cup \{v\}$ where $v \notin X$.

    \State Define a dissimilarity $d' \colon X' \times X' \to \rr$ by:
    \begin{equation*}
      d'(p,q) =
      \begin{cases}
        0 & \text{if $p,q \notin X$} \\
        (d(x,p)+d(y,p)-d(x,y))/2 & \text{if $p \in X, q \notin X$} \\
        (d(x,q)+d(y,q)-d(x,y))/2 & \text{if $p \notin X, q \in X$} \\
        d(p,q) & \text{otherwise.}
      \end{cases}
    \end{equation*}

    \State Let $(T',\omega')$ be the tree obtained by applying
    Neighbor-joining to $X'$ and $d'$.

    \State Attach $x$ and $y$ to the leaf vertex $v$ of $T'$ to obtain a tree
    $T$, put $\omega \gets \omega'$.

    \State Put $\displaystyle \omega(\{v,x\}) \gets \frac{1}{2} d(x,y) +
    \frac{1}{2(|X|-2)} \sum_{u \in X} (d(u,x)-d(u,y))$.

    \State Put $\omega(\{v,y\}) \gets d(x,y) - \omega(\{v,x\})$.

    \State \Return $(T,\omega)$.
  \end{algorithmic}
\end{algorithm}


The number of possible binary $X$-trees is $(2|X|-5)!!$ and the number of
possible rooted binary $X$-trees is $(2|X|-3)!!$
\cite{felsenstein2004inferring}.

% ---------- tree metrics ----------

A distance $\delta \colon X \times X \to \rr$ is called a \textit{tree metric}
if there exists an $X$-tree $T=(V,E)$ with edge-weighting $\omega \colon E \to
\rrnn$ such that $\delta(x,y) = D_{(T,\omega)}(x,y)$ for all $x,y \in X$.  So
for any edge-weighted $X$-tree $(T,\omega)$, the distance $D_{(T,\omega)}$ is
a tree metric.

\begin{figure}
  \centering
  \input{figures/background2/quartet.pdft}
  \caption{An $\{w,x,y,z\}$-tree with four leaves used to illustrate the
    four-point condition.}
  \label{fig:quartet-tree}
\end{figure}

The \textit{four-point condition} allows us to decide whether a given distance
is in fact a tree metric \cite{semple2003phylogenetics}.  A distance $\delta
\colon X \times X \to \rr$ satisfies the four-point condition if for every
$w,x,y,z \in X$ the following holds:
\begin{equation*}
  \delta(w,x) + \delta(y,z) \leq \max(\delta(w,y)+\delta(x,z),
                                      \delta(w,z)+\delta(x,y)).
\end{equation*}
Since the elements $w,x,y,z$ need not be distinct it is easy to see that if
$\delta$ satisfies the four-point condition then it satisfies the triangle
inequality and is therefore a metric.  Further, $\delta$ is a tree metric on
$X$ if and only if it satisfies the four-point condition
\cite[Theorem 7.2.6]{semple2003phylogenetics}.

To illustrate the fact that an induced distance $D_{(T,\omega)}$ must satisfy
the four-point condition let $T$ be the tree shown in
Figure~\ref{fig:quartet-tree} and $\omega$ be any proper edge-weighting for
it.  Then it is easy to see that $\DTw(w,x)+\DTw(y,z) < \DTw(w,y)+\DTw(x,z)$
by observing the dotted and dashed lines.  Also $\DTw(w,y)+\DTw(x,z) =
\DTw(w,z)+\DTw(x,y)$, so the distance induced by $T$ and $\omega$ on
$\{w,x,y,z\}$ satisfies the four-point condition.

An important property of tree metrics is that for any tree metric $\delta
\colon X \times X \to \rr$ there is, up to isomorphism of the underlying
$X$-tree, a unique edge-weighted $X$-tree $(T,\omega)$ for which $\delta(x,y)
= D_{(T,\omega)}(x,y)$ holds for all $x,y \in X$.  This tree can be recovered
from the metric in polynomial time \cite{semple2003phylogenetics}.  We will
see some of the methods for doing this in Section~\ref{sec:constr-from-dist}.

\end{document}
