% ---------- unrooted trees ----------

\subsubsection{Neighbor-joining}
\label{sec:neighbour-joining}

Neighbor-joining is a method for building an edge-weighted unrooted $X$-tree
which, like UPGMA, is consistent if the dissimilarity identifies a binary tree
but, unlike UPGMA, the consistency holds for general tree metrics
\cite{saitou1987nj}.  The algorithm works by replacing a cherry by a single
vertex and recursively applying Neighbor-joining to the resulting tree.
Neighbor-joining is shown in Algorithm~\ref{alg:neighbour-joining}.

\begin{algorithm}[h]
  \caption{Neighbor-joining.}
  \label{alg:neighbour-joining}

  \begin{algorithmic}
    \Require A set $X$, and a dissimilarity $d \colon X \times X \to \rr$.
    \Ensure  A binary $X$-tree $T$ and edge-weighting $\omega$.

    \If{$|X| = 2$} Let $X = \{x,y\}$ and \Return the tree obtained by joining
    $x$ and $y$ with an edge of length $d(x,y)$.
    \EndIf

    \State $\displaystyle (x,y) \gets \argmax_{x,y \in X} \left( d(x,y) + \frac{1}{2}
    \sum_{r \in X - \{x,y\}} (d(r,x)+d(r,y)-d(x,y))\right)$.

    \State $X' \gets X - \{x,y\} \cup \{v\}$ where $v \notin X$.

    \State Define a dissimilarity $d' \colon X' \times X' \to \rr$ by:
    \begin{equation*}
      d'(p,q) =
      \begin{cases}
        0 & \text{if $p,q \notin X$} \\
        (d(x,p)+d(y,p)-d(x,y))/2 & \text{if $p \in X, q \notin X$} \\
        (d(x,q)+d(y,q)-d(x,y))/2 & \text{if $p \notin X, q \in X$} \\
        d(p,q) & \text{otherwise.}
      \end{cases}
    \end{equation*}

    \State Let $(T',\omega')$ be the tree obtained by applying
    Neighbor-joining to $X'$ and $d'$.

    \State Attach $x$ and $y$ to the leaf vertex $v$ of $T'$ to obtain a tree
    $T$, put $\omega \gets \omega'$.

    \State Put $\displaystyle \omega(\{v,x\}) \gets \frac{1}{2} d(x,y) +
    \frac{1}{2(|X|-2)} \sum_{u \in X} (d(u,x)-d(u,y))$.

    \State Put $\omega(\{v,y\}) \gets d(x,y) - \omega(\{v,x\})$.

    \State \Return $(T,\omega)$.
  \end{algorithmic}
\end{algorithm}


The number of possible binary $X$-trees is $(2|X|-5)!!$ and the number of
possible rooted binary $X$-trees is $(2|X|-3)!!$
\cite{felsenstein2004inferring}.

% ---------- tree metrics ----------

A distance $\delta \colon X \times X \to \rr$ is called a \textit{tree metric}
if there exists an $X$-tree $T=(V,E)$ with edge-weighting $\omega \colon E \to
\rrnn$ such that $\delta(x,y) = D_{(T,\omega)}(x,y)$ for all $x,y \in X$.  So
for any edge-weighted $X$-tree $(T,\omega)$, the distance $D_{(T,\omega)}$ is
a tree metric.

\begin{figure}
  \centering
  \input{figures/background2/quartet.pdft}
  \caption{An $\{w,x,y,z\}$-tree with four leaves used to illustrate the
    four-point condition.}
  \label{fig:quartet-tree}
\end{figure}

The \textit{four-point condition} allows us to decide whether a given distance
is in fact a tree metric \cite{semple2003phylogenetics}.  A distance $\delta
\colon X \times X \to \rr$ satisfies the four-point condition if for every
$w,x,y,z \in X$ the following holds:
\begin{equation*}
  \delta(w,x) + \delta(y,z) \leq \max(\delta(w,y)+\delta(x,z),
                                      \delta(w,z)+\delta(x,y)).
\end{equation*}
Since the elements $w,x,y,z$ need not be distinct it is easy to see that if
$\delta$ satisfies the four-point condition then it satisfies the triangle
inequality and is therefore a metric.  Further, $\delta$ is a tree metric on
$X$ if and only if it satisfies the four-point condition
\cite[Theorem 7.2.6]{semple2003phylogenetics}.

To illustrate the fact that an induced distance $D_{(T,\omega)}$ must satisfy
the four-point condition let $T$ be the tree shown in
Figure~\ref{fig:quartet-tree} and $\omega$ be any proper edge-weighting for
it.  Then it is easy to see that $\DTw(w,x)+\DTw(y,z) < \DTw(w,y)+\DTw(x,z)$
by observing the dotted and dashed lines.  Also $\DTw(w,y)+\DTw(x,z) =
\DTw(w,z)+\DTw(x,y)$, so the distance induced by $T$ and $\omega$ on
$\{w,x,y,z\}$ satisfies the four-point condition.

An important property of tree metrics is that for any tree metric $\delta
\colon X \times X \to \rr$ there is, up to isomorphism of the underlying
$X$-tree, a unique edge-weighted $X$-tree $(T,\omega)$ for which $\delta(x,y)
= D_{(T,\omega)}(x,y)$ holds for all $x,y \in X$.  This tree can be recovered
from the metric in polynomial time \cite{semple2003phylogenetics}.  We will
see some of the methods for doing this in Section~\ref{sec:constr-from-dist}.

