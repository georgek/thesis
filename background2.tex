\chapter{Hierarchical Clustering Background}
\label{cha:background2}

\section{Trees}
\label{sec:trees}

In this section we introduce much of the terminology that is required for
dealing with trees.  Since trees are special cases of graphs, we begin with
general graph theory before moving on to trees and then special type of trees
that are of greatest interest to us.

A \textit{graph} is an ordered pair $(V,E)$ where $V$ is a set of
\textit{vertices} and $E$ is a set (or multiset) of \textit{edges}, each of
the form $\{x,y\}$ such that $x,y \in V$.  Two vertices $v,v' \in V$ are said
to be \textit{adjacent} if there exists an edge $\{v,v'\} \in E$.  A
\textit{path} in a graph is a sequence of vertices $v_1,v_2,\dotsc,v_k$ such
that for all $i \in \{1,2,\dotsc,k-1\}, v_i$ and $v_{i+1}$ are adjacent.  If
$v_1$ and $v_k$ are also adjacent then the graph is said to contain a
\textit{cycle}.  A graph is \textit{connected} if there exists a path joining
each pair of vertices in $V$.

A \textit{tree} is a connected graph with no cycles.  A \textit{forest} is a
disjoint union of trees or, equivalently, a graph with no cycles.  A
\textit{rooted tree} is a tree with one distinguished vertex called the root
which we usually call $\rho$.

An edge $\{x,y\} \in E$ is said to be \textit{incident} with the vertices $x$
and $y$.  The \textit{degree} of a vertex is the number of edges in the graph
incident with it.  A vertex in a tree is called a \textit{leaf} if it has
degree 1.

A \textit{phylogenetic $X$-tree} is a tree with no degree 2 vertices and set
of leaves $X$.  A \textit{rooted phylogenetic $X$-tree} (henceforth $X$-tree
for short) is a rooted tree with no degree 2 vertices except possibly the root
and with leaf set $X$.

An \textit{edge-weighted graph} is a graph $(V,E)$ paired with an
edge-weighting function $\omega \colon E \to \rr_{\geq 0}$.

A rooted tree has a natural partial order on the vertices.  A vertex $v$ is an
\textit{ancestor} of a vertex $v'$ if and only if $v$ is on the path from $v'$
to the root.  $v'$ is then said to be a \textit{descendant} of $v$.  If $v$
and $v'$ are adjacent then we also call $v$ the \textit{parent} of $v'$ and
$v'$ a \textit{child} of $v$.

Two graphs $(V,E)$ and $(V',E')$ are called \textit{isomorphic} if there
exists a bijection $\phi \colon V \to V'$ such that whenever $\{v,v'\} \in E$
we have $\{\phi(v),\phi(v')\} \in E'$.  So, in other words, adjacency of
vertices is preserved.  We call two $X$-trees \textit{equivalent} if there
exists a bijection which preserves adjacency, is the identity on $X$ and maps
the root of one tree to the root of the other.  This means that, in addition
to adjacency, the parent and child relationships are preserved.

The \textit{lowest common ancestor} of two vertices $v$ and $v'$ in a rooted
tree is the unique vertex that lies on the pat from $v$ to $v'$, the path from
$v$ to the root and the path from $v'$ to the root.  The lowest common
ancestor of $v$ and $v'$ is denoted $\lca(v,v')$.

A tree in which all vertices have degree 3 is called \textit{binary}.  Rooted
binary trees are often defined with the root having degree 2; we defined a
\textit{rooted binary $X$-tree} to be an $X$-tree where each vertex has degree
3 apart from the root which has degree 2.

A pair of leaves of a rooted $X$-tree that share the same parent is called a
\textit{cherry}.  In general, a set of leaves that share the same parent is
called a \textit{pseudocherry}.  A rooted $X$-tree with $|X| = 3$ that
contains a cherry is called a \textit{triplet}.  If $X = \{a,b,c\}$ and $a,b$
is a cherry then we denote the triplet by $ab|c$.

A graph $(V',E')$ is a \textit{subgraph} of a graph $(V,E)$ if $V' \subseteq
V$ and $E' \subseteq E$.  A connected subgraph of a tree is called a
\textit{subtree}.  If $T$ is an $X$-tree and $X' \subseteq X$ then we denote
by $T|X'$ the minimal subtree of $X$ whose vertex set contains $X'$, with
degree 2 vertices suppressed.  $T|X'$ is then called a \textit{restricted
  subtree}.  If $T$ is binary and $|X'| = 3$ then $T|X'$ is a triplet.

%%% Local Variables:
%%% TeX-master: "thesis"
%%% End:
