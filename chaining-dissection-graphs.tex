\begin{figure}
  \centering
  \begin{subfigure}[b]{0.4\textwidth}
    \begin{tikzpicture}
      \begin{axis}[width=1.2\textwidth,xmin=30,xmax=270,ymin=80,ymax=270,ticks=none]
        \addplot[only marks,color=blue,mark=x] table {figures/chaining-dissection/chaining};
        \addplot[only marks,color=red,mark=o] table {figures/chaining-dissection/chaining2};
      \end{axis}
    \end{tikzpicture}
    \caption{Chaining effect.}
    \label{fig:chaining}
  \end{subfigure}
  \begin{subfigure}[b]{0.4\textwidth}
    \begin{tikzpicture}
      \begin{axis}[width=1.2\textwidth,xmin=90,xmax=230,ymin=230,ymax=400,ticks=none]
        \addplot[only marks,color=blue,mark=x] table {figures/chaining-dissection/dissection};
        \addplot[only marks,color=red,mark=o] table {figures/chaining-dissection/dissection2};
      \end{axis}
    \end{tikzpicture}
    \caption{Dissection effect.}
    \label{fig:dissection}
  \end{subfigure}
  \caption{The chaining effect can produce clusters with poor homogeneity and
    the dissection effect can produce clusters with poor separation.}
  \label{fig:chaining-dissection}
\end{figure}
