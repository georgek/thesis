% stuff that is duplicated from background


In many topical studies in Computational Biology ranging from gene onthology
via genome wide association studies in population genetics to evolutionary
genomics, the following fundamental mathematical problem is encountered: Given
a distance $D$ on some set $X$ of objects, find a dendrogram $\mathcal D$ on
$X$ (essentially a rooted tree $T=(V,E)$ with no degree two vertices but
possibly the root whose leaf set is $X$ together with an edge-weighting
$\omega:E\to\mathbb R_{\geq 0}$ -- see Fig.~\ref{fig:block-graph-motivation}
for examples) such that the distance induced by $\mathcal D$ on any two of its
leaves $x$ and $y$ equals $D(x,y)$. In the ideal case that the distances
between any two elements of $X$ are available, it is well-understood when such
a tree is uniquely determined by them and fast algorithms for reconstructing
it from them are known (see e.\,g.\,\cite[Chapter 9.2]{DHKMS11} and
\cite[Chapter 7.2]{semple2003phylogenetics} where dendrograms are considered
in the slightly more general forms of dated rooted $X$-trees and equidistant
representations of dissimilarities, respectively, and \cite[Chapter 3]{BG91}
as well as the references in all three of them for more on this).
 
The reality however tends to be different in many cases in that distances
between pairs of objects might be missing or are not sufficiently reliable to
warrant inclusion of that distance in an analysis -- see
e.g. \cite{philippe2004phylogenomics,sanderson10phylogenomics,steel10characterizing}
for more on this topic in an evolutionary genomics context).  Exclusion of
such a distance might therefore be tempting but is clearly not always
desirable which raises interesting mathematical, statistical, and
algorithmical questions (see
e.\,g.\,\cite{de1984ultrametric,farach1995robust,schader1992mvl} for a study
concerning the latter and
\cite{farach1995robust,guenoche1999approximations,guenoche2004extension,makarenkov2001nouvelle}
for results concerning its unrooted variant).  One of them is the focus of
this paper: Calling any subset of a finite set $X$ of size two a {\em cord} of
$X$ then for what sets $\cL$ of cords of $X$ do we need to know the distances
so that both the topology of the underlying tree and the edge-weights of the
dendrogram on $X$ that induced the distances on the cords in $\cL$ is uniquely
determined by $\cL$?

To help illustrate the intricacies of this question which is concerned with
the structure of the set $\cL$ and not so much with the actual distances on
the cords in $\cL$, denote for any two distinct elements $a,b\in X$ the cord
$\{a,b\}$ by $ab$. Consider the dendrogram $\mathcal D$ with leaf set
$X=\{a,\ldots, e\}$ depicted in Fig.~\ref{fig:lasso-example}(i) and assume
that the distances on the cords of $\cL=\{ac,de,bc,ce,cd\}$ are induced by
$\mathcal D$ so, for example, the distance on the cord $ab$ is four. Then the
dendrogram $\mathcal D'$ depicted in Fig.~\ref{fig:lasso-example}(ii) induces
the same distances on the cords in $\cL$ as $\mathcal D$ but the topologies of
the underlying trees $T$ and $T'$ of $\mathcal D$ and $\mathcal D'$,
respectively, are clearly not the same in the sense that there exists no
bijection from $V(T)$ to $V(T')$ that is the identity on $\{a,\ldots, e\}$ and
induces a rooted graph isomorphism from $T$ to $T'$.  Thus, $\cL$ does not
uniquely determine $T$ and thus also not $\mathcal D$. However as can be
quickly checked the situation changes if and only if the cord $ab$ (or a
subset of ${X\choose 2}$ containing that cord) is added to $\cL$.  To make
this more precise, let $\cL'$ denote the resulting set of cords on $X$ and let
$\mathcal D_1$ denote a dendrogram on $X$ for which the topology of the
underlying tree is the same as that of $\mathcal D$. If $\mathcal D_2$ is a
dendrogram on $X$ such that the distances on the cords in $\cL'$ induced by
$\mathcal D_1$ and $\mathcal D_2$ coincide then, as is easy to verify, the
topologies of the underlying trees of $\mathcal D_1$ and $\mathcal D_2$,
respectively, must be the same and so must be their edge-weightings. Thus,
$\cL'$ uniquely determines $\mathcal D$.

Although an intriguing question, apart from some recent results in
\cite{huber13lassoing}, not much is known about it (see \cite{dress11lassoing}
and \cite{HS13} for some partial results in case the tree in question is
unrooted).  By formalising a dendrogram in terms of a certain edge-weighted
$X$-tree (see the next section for a precise definition of this concept as
well as all the other concepts mentioned below) and using the concept of a
topological lasso which was originally introduced for unrooted phylogenetic
trees with leaf set $X$ in \cite{dress11lassoing} and extended to $X$-trees in
\cite {huber13lassoing}, we study this question in the form of when a set of
cords of $X$ is a topological lasso for a given $X$-tree $T$. In the context
of this, we are particularly interested in (set-inclusion) minimal topological
lassos $\cL$ for $T$ for which $ \bigcup \cL:=\bigcup_{A\in\cL} A=X$ holds.

For $T$ an $X$-tree, we show for any such minimal topological lasso $\cL$ for
$T$ that in case the graph $\Gamma(\cL)$ whose vertex set is $X$ and any two
distinct elements $x$ and $y$ in $X$ joined by an edge if $xy\in \cL$ -- see
Fig~\ref{fig:block-graph-motivation}(i) for an example of that graph for
$\cL=\{ab,cd,ef,ac,ce,ea\}$ -- is a block graph then the blocks of
$\Gamma(\cL)$ are in one-to-one correspondence with the non-leaf vertices of
$T$ (Corollary~\ref{cor:bijection}).  Furthermore, we establish in
Theorem~\ref{theo:transform} that any minimal topological lasso $\cL$ for $T$
can be transformed into a very special type of minimal topological lasso
$\cL^*$ for $T$ in that $\Gamma(\cL^*)$ is a claw-free block graph where a
graph is called {\em claw-free} if it does not contain a {\em claw}, that is,
the complete bipartite graph $K_{1,3}$ as an induced subgraph \cite{H72}.

Claw-free graphs have been shown to enjoy numerous properties relating them
to, for example, perfect graphs, perfect matchings, and maximum independent
sets (see e.\,g.\,\cite{FFZ97} and \cite{CFHV12} for overviews).  Furthermore,
claw-free block graphs were related in \cite{BL09} to $k$-leaf powers of trees
and their spectrum was studied in \cite{GS01, MSST06} (see also \cite{BR13}
for a more general study of the adjacency matrix of such graphs).  Calling a
minimal topological lasso $\cL$ for $T$ {\em distinguished} if $\Gamma(\cL)$
is a claw-free block graph, we present in Theorem~\ref{theo:characterization}
a characterisation of a distinguished minimal topological lasso for $T$ in
terms of the novel concept of a cluster marker map for $T$. In addition, we
characterise when a distinguished minimal topological lasso for $T$ gives rise
to a distinguished minimal topological lasso for a subtree of $T$
(Theorem~\ref{theo:subtree}) and also present a partial answer to the
canonical analogue of a question raised for supertrees of unrooted
phylogenetic trees in \cite{dress11lassoing}.

The paper is organised as follows. In Section~\ref{sec:terminology}, we
introduce relevant terminology surrounding $X$-trees and lassos. In
Section~\ref{sec:gamma-l-graph}, we collect first properties of the graph
$\Gamma(\cL)$ associated to a topological lasso $\cL$ and in
Section~\ref{sec:blockgraph}, we establish Corollary~\ref{cor:bijection}. In
Section~\ref{sec:distinguished}, we commence our study of distinguished
minimal topological and establish Theorem~\ref{theo:transform}. In
Section~\ref{sec:sufficient}, we present a sufficient condition for when a
minimal topological lasso is distinguished (Theorem~\ref{theo:
  distinguished-lasso-verification}) and in
Section~\ref{sec:characterization-distinguished}, we prove
Theorem~\ref{theo:characterization}. We conclude with
Section~\ref{sec:subtree} where we establish Theorem~\ref{theo:subtree} and
also outline directions for further research.


\subsection{$X$-trees}
A {\em rooted tree} $T$ is a tree with a unique distinguished vertex called
the {\em root} of $T$, denoted by $\rho_T$. Throughout the paper, we assume
that the degree of the root of a rooted tree is at least two.  A {\em rooted
  phylogenetic $X$-tree}, or {\em $X$-tree} for short, is a rooted tree
$T=(V,E)$ with no degree two vertices but possibly the root $\rho_T$ whose
leaf set is $X$. We call an $X$-tree $T$ a {\em star-tree on $X$} if every
leaf of $T$ is adjacent with the root of $T$.

Suppose for the following that $T$ is an $X$-tree. Then we call a vertex of
$T$ that is not a leaf of $T$ an {\em interior vertex} of $T$ and denote the
set of interior vertices of $T$ by $\iV(T)$.  We call an edge of $T$ that is
incident with a leaf of $T$ a {\em pendant edge} of $T$ and every edge of $T$
that is not a pendant edge an {\em interior edge} of $T$.  Extending some of
the terminology for directed graphs to $X$-trees, we call for all vertices
$v\in V(T)-\{\rho_T\}$ an edge $e\in E(T)$ a {\em parent edge of $v$} if $e$
is incident with $v$ and lies on the path from the root $\rho_T$ of $T$ to
$v$. We refer to the vertex incident with $e$ but distinct from $v$ as a {\em
  parent} of $v$.
 
Suppose for the following that $v$ is an interior vertex of $T$. If $v$ is not
the root of $T$ then we call an edge $e\in E(T)$ a {\em child edge of $v$} if
$e$ is incident with $v$ but is not crossed by the path from $\rho_T$ to $v$.
In addition, we call every edge incident with $\rho_T$ a {\em child edge of
  $\rho_T$}.  We call the vertex incident with a child edge of an interior
vertex $w$ of $T$ but distinct from $w$ a {\em child of $w$} and denote the
set of all children of $v$ by $ch_T(v)$.

We call a vertex $w\in V(T)$ distinct from $v$ a {\em descendant} of $v$ if
either $w$ is a child of $v$ or there exists a path from $v$ to $w$ that
crosses a child of $v$.  We denote the set of leaves of $T$ that are also
descendants of $v$ by $L_T(v)$. If $v$ is a leaf of $T$ then we put
$L_T(v):=\{v\}$. If there is no ambiguity as to which $X$-tree $T$ we are
referring to then, for all $v\in V(T)$, we will write $L(v)$ rather than
$L_T(v)$ and $ch(v)$ rather than $ch_T(v)$.

We call a non-empty subset $L\subsetneq X$ of leaves of $T$ such that $L=L(v)$
holds for some $v\in \iV(T)$ a {\em pseudo-cherry} of $T$. In that case, we
also call $v$ the {\em parent} of that pseudo-cherry.

Note that every $X$-tree on three or more leaves must contain at least one
pseudo-cherry. Also note that a pseudo-cherry of size two is a {\em cherry} in
the usual sense (see e.g. \cite{semple2003phylogenetics}).

For $x$ and $y$ distinct elements in $X$, we call the unique vertex of $T$
that simultaneously lies on the path from $x$ to $y$, on thne path from $x$ to
$\rho_T$, and on the path from $y$ to $\rho_T$ the {\em last common ancestor
  of $x$ and $y$}, denoted by $lca_T(x,y)$. More generally, for any subset
$Y\subseteq X$ of size three or more, we denote the subtree of $T$ with leaf
set $Y$ and vertices of degree two suppressed (except possibly the root) by
$T|_Y$ and call the root of $T|_Y$ the {\em last common ancestor of $Y$},
denoted by $lca_T(Y)$.

Finally, suppose that $T'$ is be a further $X$-tree. Then we say that $T$ and
$T'$ are {\em equivalent} if there exists a bijection $\phi:V(T)\to V(T')$
that extends to a graph isomorphism between $T$ and $T'$ that is the identity
on $X$ and maps the root $\rho_T$ of $T$ to the root $\rho_{T'}$ of $T'$.

\section{Constructing a tree from a distinguished lasso}
\label{sec:constr-tree-from}

Let $(T,\omega)$ be an equidistant $X$-tree and let $\cL$ be a distinguished
minimal topological lasso for $T$.  To the graph $\Gamma(\cL)$ we can assign
an edge-weight function $\alpha_{\cL} \colon E(\Gamma(\cL)) \to \mathbb{R}^+$
by letting $\alpha_{\cL}(xy) = D_{(T,\omega)}(x,y)$ for all edges $xy \in
E(\Gamma(\cL))$.

We next associate an edge-weighted tree $(\AcL,\BecL)$ to $\Gamma(\cL)$ by
essentially replacing every block of $\Gamma(\cL)$ by an edge-weighted star
tree.  More precisely, let $B$ be a block in $\Gamma(\cL)$.  We first choose
distinct vertices $a,b \in V(B)$ and put $m_B := \alpha_{\cL}(\{a,b\})$.  Note
that since $\omega$ is a equidistant proper edge-weighting for $T$ this
definition of $m_B$ is independent of the choice of $a$ and $b$.  Next we
delete all edges from $B$, add a vertex $s_B$ to $V(B)$ and new edges
$\{a,s_B\}$ for all $a \in V(B) - \{s_B\}$ to obtain a star tree $S_B$ with
leaf set $V(B)$.  Finally, we put $\beta_B \colon E(S_B) \to \mathbb{R}^+,
\beta_B(e) = m_B/2$ for all $e \in E(S_B)$.  Once this has been done for all
blocks in $\Gamma(\cL)$ we obtain the tree $\AcL$.  We define an
edge-weighting $\beta_{\cL} \colon E(\AcL) \to \mathbb{R}^+$ by putting
$\beta_{\cL}(e) = \beta_B(e)$ for the unique block $B \in \Block(\Gamma(\cL))$
with $e$ an edge in $B$.

Now we present an algorithm for constructing an edge-weighted $X$-tree
$(T,\omega)$ from $\AcL$ and $\BecL$ which we call \textsc{TreeConstruct}.
The algorithm builds a tree $T$ using the graph $\AcL$ and the function
$\BecL$ by traversing $\AcL$ and building the tree by adding internal vertices
and their children.  The pseudocode for the algorithm is shown in
Figure~\ref{algorithm:treeconstruct2}.

Note that we make use of the distance function $D_{(T,\omega)} \colon V
\times V \to \mathbb{R}^+$ that is induced by the edge-weight function
$\omega$ by letting, for all pairs $x,y \in V$, $D_{(T,\omega)}(x,y)
= \sum_{e \in P(x,y)} \omega(e)$ where $P(x,y)$ is the set of edges on the
path from $x$ to $y$.

\begin{algorithm}
  \caption{\textsc{Order}($\AcL,u,v,o$)}
  \label{algorithm:order}
  
  \begin{algorithmic}
    \Require Graph $\AcL$ derived from a distinguished minimal topological
    lasso $\cL$, vertices $u,v \in V(\AcL)$ and an $n$-tuple $o$.
    \Ensure An $(n+1)$-tuple $o^*$.

    \State Let $C_u := \{c \in V(\AcL) \colon cu \in \cL, c \neq v\}$.
    \State Put $o^* := (o,u)$.
    \ForAll{$v \in C_u$}
    \If{$\deg_{\AcL}(v) > 1$}
    \State Let $u^* \in v(\AcL)$ be the vertex such that $u^*v \in \cL$,
    \State Put $o^* := \text{\textsc{Order}}(\AcL, u^*, v, o^*)$,
    \State Put $o^* := (o^*, u)$.
    \EndIf
    \EndFor

    \State \Return $o^*$.
  \end{algorithmic}
\end{algorithm}

\begin{figure}
\centering
\parbox{0cm}{\begin{tabbing}
XX\= XX\= XX\=  XX\= XX\= XX\= XX\=  XXXXX\=  XXXXXXXXXXX\= \kill \\
{\textsc{TreeConstruct}($\AcL,\BecL$)} \\
\rule{\columnwidth}{0.5pt}\\
\textbf{Input:} \> \> \> Graph $\AcL$ derived from a distinguished minimal topological lasso $\cL$\\
                \> \> \> and an edge-weight function $\BecL \colon \cL \to \mathbb{R}_{>0}$.\\
\textbf{Output:} \> \> \> An $X$-tree $T = (V,E)$ with root $\rho$ and an
                          equidistant proper edge-\\
                 \> \> \> weighting $\omega \colon E \to
                          \mathbb{R}_{\geq 0}$.\\\\

\lnum{\phantom{0}1} \> Label all vertices in $\AcL$ as unprocessed.\\

\lnum{\phantom{0}2} \> Let $v$ be some degree 1 vertex in $\AcL$ and $u$ be the vertex
           adjacent to $v$.\\

\lnum{\phantom{0}3} \> Let $(u_1,u_2,\dotsc,u_n) := \text{\textsc{Order}}(\AcL, u, v, \emptyset)$.\\

\lnum{\phantom{0}4} \> Let $C_{u_1} := \{c \in V(\AcL) \colon \{c,u_1\} \in E(\AcL)\}$.\\
\lnum{\phantom{0}5} \> Put $T:=(V,E)$ with $V := \{u_1\} \cup C_{u_1}$, $E := \{\{u_1,c\} \colon c \in C_{u_1}\}$
              and $ \rho := u_1$.\\
\lnum{\phantom{0}6} \> Define a function $\omega \colon E \to \rr_{\geq 0}$ and let
              $\omega(\{u_1,c\}) := \BecL(\{u_1,v\})$ for all $c \in C_{u_1}$.\\
\lnum{\phantom{0}7} \> Label all $c \in C_{u_1}$ in $\AcL$ as processed.\\

\lnum{\phantom{0}8} \> \textbf{Foreach} $i \in (2,\dotsc,n)$ \textbf{do}:\\
\lnum{\phantom{0}9} \> \> \textbf{If} $u_i$ is already processed, \textbf{continue}
                 with $i := i+1$.  \textbf{EndIf.}\\
\lnum{10} \> \> Let $C_{u_i} := \{c \in V(\AcL) \colon \{c,u_i\} \in E(\AcL)$ and $c$
                 is unprocessed.$\}$\\
\lnum{11} \> \> Let $v \in V(\AcL)$ be the unique vertex such that both $\{u_i,v\}
                 \in E(\AcL)$\\
            \> \> \> and $\{v,u_{i-1}\} \in E(\AcL)$.\\
\lnum{12} \> \> \textbf{If} $\DTw(\rho,v) < \BecL(\{v,u_i\})$
                 \textbf{then}:\\
\lnum{13} \> \> \> Put $V^* := V \cup \{u_i\} \cup C_{u_i}$,\\
           \> \> \> \> $E^* := E \cup \{\{u_i,c\} \colon c \in C_{u_i}\} \cup
                       \{\{\rho,u_i\}\}$, and\\
           \> \> \> \> $\rho^* := u_i$.\\
\lnum{14} \> \> \> Define a function $\omega^* \colon E^* \to
                    \mathbb{R}_{\geq 0}$ and let $\omega^*(e) := \omega(e)$
                    for all $e \in E$,\\
           \> \> \> \> $\omega^*(\{u_i,c\}) := \BecL(\{u_i,v\})$ for all $c
                       \in C_{u_i}$, and\\
           \> \> \> \> $\omega^*(\{\rho,u_i\}) := \BecL(\{v,u_i\}) -
                       \DTw(v,\rho)$.\\

\lnum{15} \> \> \> Label all $\{u_i\} \cup C_{u_i}$ in $\AcL$ as processed and put
                    $\rho := \rho^*$.\\

\lnum{16} \> \> \textbf{Else}:\\
\lnum{17} \> \> \> Let $p,q \in V$ be the unique vertices such that $\{p,q\}
                    \in E$ and\\
           \> \> \> \> $\DTw(v,p) > \BecL(\{v,u_i\}) > \DTw(v,q)$.\\
           \> \> \> Put $V^*:= V \cup \{u_i\} \cup C_{u_i}$, and\\
           \> \> \> \>  $E^*:= E \cup \{\{u_i,c\} \colon c \in C_{u_i}\} \cup
                        \{\{p,u_i\},\{u_i,q\}\} - \{\{p,q\}\}$.\\
\lnum{18} \> \> \> Define a function $\omega^* \colon E^* \to
                    \mathbb{R}_{\geq 0}$ and let\\
           \> \> \> \> $\omega^*(e) := \omega(e)$ for all $e \in E - \{\{p,q\}\}$,\\
           \> \> \> \> $\omega^*(\{u_i,c\}) := \BecL(\{u_i,v\})$ for all $c \in
                       C_{u_i}$,\\
           \> \> \> \> $\omega^*(\{p,u_i\}):= \DTw(v,p) - \BecL(\{u_i,v\})$, and\\
           \> \> \> \> $\omega^*(\{u_i,q\}):= \BecL(\{u_i,v\}) - \DTw(v,q)$.\\

\lnum{19} \> \> \> Label all $\{u_i\} \cup C_{u_i}$ in $\AcL$ as processed.\\

\lnum{20} \> \> \textbf{EndIf.}\\

\lnum{21} \> \> Put $V:=V^*, E:=E^*$ and $\omega:=\omega^*$.\\

\lnum{22} \> \textbf{EndFor.}

         \end{tabbing}}
\caption{Pseudocode for \textsc{TreeConstruct}.}
\label{algorithm:treeconstruct2}
\end{figure}

We next show that the construction ConstructTree is correct.  So assume for
the remainder of this section that $T_O$ is an $X$-tree, that $\cL$ is a
distinguished minimal topological lasso for $T_O$ and that the distance $D_O(x,y)$
is known for all $xy \in \cL$.  Let $(T,\omega)$ be the $X$-tree $T$ with
equidistant proper edge-weighting $\omega \colon E(T) \to
\mathcal{R}_{\geq 0}$ returned by \textsc{ConstructTree}.

We begin with a lemma which shows that the two conditional cases in step 4 are
valid.  Specifically, we cannot have a case where
$D_{(T,\omega^*)}(v,\rho) = \beta(\{v,u^*\})$ so the algorithm continues to
run until all vertices in $\AcL$ have been exhausted.

\begin{lem}
  \label{lem:nonequal-adjacent}
  Let $B,B' \in Block(\Gamma(\cL))$ denote two distinct blocks of
  $\Gamma(\cL)$ such that $V(B) \cap V(B') \neq \emptyset$.  Then for all $e
  \in E(B)$ and all $e' \in E(B')$ we have $\BecL(e) \neq \BecL(e')$.
\end{lem}

\begin{proof}
  Assume for the purpose of contradiction that there exists some $e \in E(B)$
  and some $e' \in E(B')$ such that $\BecL(e) = \BecL(e')$.  Let $\alpha =
  \BecL(e)$, $\alpha' = \BecL(e')$ and $y \in V(B) \cap V(B')$.  Note that $y$
  is a cut vertex of $\Gamma(\cL))$.  Choose some $a \in V(B) - \{y\}$ and $a'
  \in V(B') - \{y\}$ which must exist since the size of a block in
  $\Gamma(\cL)$ is at least two.  Note that 

  Then, on the one hand, our assumptions on
  $e$ and $e'$ combined with the definition of the edge-weighting $\omega$
  implies that

  the paths $P$ and $P'$ from $v = \lca_{T}(y,a)$ to $a$ and $v'
  = \lca_{T}(y,a')$ to $a'$, respectively, have the same length.  But on the
  other hand the choice of $y$ implies that either $v'$ is a descendant of $v$
  in $T$ or $v$ is a descendant of $v'$ in $T$ and so $P$ and $P'$ must have
  different lengths: a contradiction.
\end{proof}

\begin{lem}
  \label{lem:t-is-x-tree}
  $T$ is an $X$-tree and $\omega$ is a proper equidistant edge-weighting for
  $T$.
\end{lem}

\begin{proof}
  We proceed by induction on the number of times steps 2 and 3 are executed
  or, equivalently, the number of stars in $\AcL$ processed.

  The case is that $T$ is a $Y$-tree, where $Y \subseteq X$, following the
  first, and only, execution of step 2.  Since step 2 is simply setting $T$
  equal to a star from $\AcL$ it is clear that $T$ will be a $Y$-tree.
  Further, since the edge-weights are transferred directly from $\AcL$ and
  these are positive and equal for all edges within a star by definition,
  $\omega$ will be an equidistant proper edge-weighting.

  For the inductive step we assume that $T = (V,E)$ is a $Y$-tree and that
  $\omega$ is an equidistant proper edge-weighting prior to step 3 and we show
  that the tree $T^* = (V^*,E^*)$ obtained following step 3 is a $Y^*$-tree,
  where $Y^* \supset Y$ and that the edge-weighting $\omega^* \colon E(T^*)
  \to \rr_{>0}$ obtained is an equidistant proper edge-weighting.

  First we show that $T^*$ is a $Y^*$-tree.  Since $T$ is a tree it is
  connected and $|V| = |E| + 1$.  In step 3 there are two cases: either we add
  to $T$ one internal vertex $u$, a set of leaves $C_u$ and $|C_u|+1$ edges;
  or we replace an edge $\{u,v\}$ with a path $u,w,v$, where $w$ is a new
  vertex not present in $T$, add a set of leaves $C_w$ and $|C_w|$ edges.  In
  either case it is clear that $T^*$ is connected and it remains that
  $|V^*| = |E^*| + 1$ so $T^*$ is a $Y^*$-tree.

  Now we show that $\omega^*$ is an equidistant proper edge-weighting.  Again
  there are two cases in step 3 which we deal with separately.  In the first
  case we have that $D_{(T,\omega)}(v,\rho) < \BecL(\{v,u^*\})$ so
  $\BecL(\{v,u^*\}) - D_{(T,\omega)}(v,\rho)$ is positive.  All other values
  for $\omega^*$ come from either $\omega$ or $\BecL$ which are positive
  functions, so $\omega^*$ is a proper edge-weighting.  To see that it is
  equidistant note that both $T$ and all $c \in C_{u^*}$ become children of
  $u^*$.  Since $T$ is equidistant, $D_{(T,\omega)}(v,\rho)$ is the distance
  between all leaves of $T$ and $\rho$, therefore the distance between all
  leaves of $T^*$ and $\rho^*$ becomes $\BecL(\{v,u^*\})$, so $\omega^*$ is
  equidistant.

  In the second case we have that $D_{(T,\omega)}(v,\rho) > \BecL(\{v,u^*\})$
  and also the inequality $D_{(T,\omega)}(v,p) > \BecL(\{v,u^*\}) >
  D_{(T,\omega)}(v,q)$.  It is clear then that all calculated values for
  $\omega^*$ will be positive and, as before, the remainder come from either
  $\omega$ or $\BecL$, therefore $\omega^*$ is proper.  Similar to the first
  case, the children of $u^*$ will be the leaves $C_{u^*}$ and a tree with $q$
  as its root.  This tree will be ultrametric since it is a subtree of $T$ so
  the distance between $q$ and all of its leaf descendants will be
  $D_{(T,\omega)}(v,q)$.  It follows that the distance between $u^*$ and these
  leaves will be $\omega^*(\{u^*,q\}) + D_{(T,\omega)}(v,q) = \BecL(\{u^*,v\})
  - D_{(T,\omega)}(v,q) + D_{(T,\omega)}(v,q) = \BecL(\{u^*,v\})$ and this is
  equal to the edge-weight set for $\omega^*(\{u^*,c\})$ for all $c \in
  C_{u^*}$.  Now since the distance between $p$ and all of its leaf
  descendants in $T$ is $D_{(T,\omega)}(v,p)$ and we are setting
  $\omega^*(\{p,u^*\}) := D_{(T,\omega)}(v,p) - \BecL(\{u^*,v\})$, the
  distance between $p$ and its leaf descendants in $T^*$ remains the same,
  therefore $\omega^*$ is equidistant.

  Finally, it is clear that the algorithm will terminate after all stars in
  $\AcL$ have been exhausted and since each unprocessed leaf in $\AcL$ is
  processed for every star we will have that $T^*$ is a $Y^*$-tree with $Y^* =
  X$ following the final execution of either step 2 or 3.
\end{proof}

\begin{lem}
  \label{lem:t-t-star-equal-dists}
  For each cord $xy \in \cL$ we have that $D_{(T,\omega)}(x,y) = D_O(x,y)$.
\end{lem}

\begin{proof}
  By definition, a star with centre $u$ and leaves $v_1,\dotsc,v_k$ is present
  in $\AcL$ if and only if the cords $\{v_iv_j \colon 1 \leq i < j \leq k\}$
  are present in $\cL$.  Also by definition $\beta(\{u_i,v_i\}) =
  D_{(T,\omega)}(v_i,v_j)/2$ for all $1 \leq i,j \leq k$ and $i \neq j$.
  Whenever an internal vertex $u^*$ is added in either steps 2 or 3 we also
  attach the set $C_{u^*}$ of leaves to it.  It is ensured initially that
  $\omega^*(\{u^*,v\}) = \beta(\{u^*,v\})$ for all $v \in C_{u^*}$.  Any of
  these edges $\{u,v\}$ may later be removed and replaced by a path $u,w,v$ by
  subsequent executions of step 3, but the length of the path $u,w,v$ is kept
  equal to the original length of $\{u,v\}$ each time.  Therefore
  $D_{(T,\omega)}(x,y) = D_O(x,y)$ for all $x,y \in \cL$.
\end{proof}

As a result of Lemmata~\ref{lem:t-is-x-tree} and
\ref{lem:t-t-star-equal-dists} and the definitions of a topological lasso we
obtain the following theorem:

\begin{thm}
  \label{thm:t-t-star-l-isometric}
  $(T,\omega)$ and $(T^*,\omega^*)$ are $\cL$-isometric.  In particular, $T$
  and $T^*$ are equivalent.
\end{thm}

