\chapter{Distinguished Minimal Topological Lassos}
\label{cha:dist-minim-topl}

\textit{This chapter is based on the following paper of which I was a minor
  author:}

\vspace{0.5em}

\noindent

\begin{itemize}
\item Katharina T. Huber and George Kettleborough. Distinguished minimal
  topological lassos (submitted).
\end{itemize}

\vspace{1em}

\textit{I was involved in developing the idea of a special type of minimal
  topological lasso whose $\Gamma(\cL)$ graph is a block graph, the
  development of much of the terminology, lemma~\ref{lem:size-A(v)},
  propositions~\ref{prop:gamma-l-connected} and \ref{prop:x-i-unique},
  theorem~\ref{theo:unique-block} and all of the examples used throughout.
  This theory was intended to be used by a new algorithm, however the
  development of the algorithm was unsuccessful.}
\newpage

\section{Introduction}

\subsection{Summary}

In this chapter we focus on a type of lasso called a minimal topological
lasso, that is a topological lasso from which no cord can be removed such that
the set of cords remains a topological lasso.  We show that any set-inclusion
minimal topological lasso for such a tree $T$ can be transformed into a
``distinguished'' minimal topological lasso $\cL$ for $T$, that is, the graph
$(X,\cL)$ is a claw-free block graph. Furthermore, we characterise such lassos
in terms of the novel concept of a cluster marker map for $T$ and present
results concerning the heritability of such lassos in the context of the
subtree and supertree problems.

\subsection{Minimal topological lassos and the graph $\Gamma(\cL)$}
\label{sec:terminology}

In this section, we introduce the extra terminology required for this chapter
and establish some initial results.  Assume throughout that $X$ is a finite
set with at least 3 elements and that, unless stated otherwise, all sets $\cL$
of cords of $X$ considered satisfy the property that $X=\bigcup \cL$.  We say
that $\cL$ is a {\em (set-inclusion) minimal topological lasso for $T$} if
$\cL$ is a topological lasso for $T$ but no cord $c \in \cL$ can be removed
from $\cL$ such that $\cL-\{c\}$ is still a topological lasso for $T$.

We denote the set of leaves of $T$ that are also descendants of $v$ by
$L_T(v)$. If $v$ is a leaf of $T$ then we put $L_T(v):=\{v\}$. If there is no
ambiguity as to which $X$-tree $T$ we are referring to then, for all $v\in
V(T)$, we will write $L(v)$ rather than $L_T(v)$ and $ch(v)$ rather than
$ch_T(v)$.

To facilitate the discussion of lassos we will very often refer to a graph
called $\Gamma(\cL)$.  For a set of cords $\cL$ of $X$ the graph $\Gamma(\cL)$
has vertex set $X$ and an edge between distinct elements $x$ and $y$ in $X$
whenever $xy \in \cL$.  If there is no danger of confusion, we denote an edge
$\{a,b\}$ of $\Gamma(\cL)$ by $ab$ rather than $\{a,b\}$.

\begin{figure}
  \begin{center}
    \input{figures/dist-min-lass/lasso-2-trees.pdft}
  \end{center}
  \caption{(i) The graph $\Gamma(\cL)$ with vertex set $X=\{a,b,\ldots,f\}$
    for the set $\cL=\{ab,cd,ef,ac,ce,ea\}$. (ii) Two non-equivalent $X$-trees
    $T$ and $T'$ that are both topologically lassoed by $\cL$. In fact, $\cL$
    is a minimal topological lasso for either one of them.}
  \label{fig:block-graph-motivation}
\end{figure}

To illustrate these definitions, let $X=\{a,\cdots,f\}$ and let $\cL$ be the
set of cords such that $\Gamma(\cL)$ is the graph depicted in
Fig.~\ref{fig:block-graph-motivation}~(i).  It is easy to see that the
$X$-trees depicted in Fig.~\ref{fig:block-graph-motivation}~(ii) are
topologically lassoed by $\cL$. In fact, $\cL$ is a minimal topological lasso
for both of them.

Denoting for an $X$-tree $T$, a topological lasso $\cL$ for $T$, and an
interior vertex $v\in \iV(T)$ the set of all cords $ab\in \cL$ for which
$v=lca_T(a,b)$ holds by $\cA(v)$, Theorem~\ref{thm:child-edge-graph-complete}
readily implies $|\cA(v)|\geq {|ch(v)|\choose 2}$.  The next observation is
almost trivial yet central to this chapter and concerns the special case that
$\cL$ is a minimal topological lasso for $T$.  To able to state it, we denote
for an interior vertex $v\in \iV(T)$ and a child edge $e\in E(T)$ of $v$ the
child of $v$ incident with $e$ by $v_e$.

\begin{lem}\label{lem:size-A(v)}
  Suppose $T$ is an $X$-tree and $\cL$ is a minimal topological lasso for
  $T$. Then, for all $v\in \iV(T)$, we have $|\cA(v)|={|ch(v)|\choose 2}$. In
  particular, for any two distinct child edges $e_1$ and $e_2$ of $v$ there
  exists precisely one pair $(a_1,a_2)\in L(v_{e_1})\times L(v_{e_2})$ such
  that $a_1a_2\in\cL$.
\end{lem}

Note that Lemma~\ref{lem:size-A(v)} immediately implies that any two minimal
topological lassos for the same $X$-tree must be of equal size.

To be able to establish Proposition~\ref{prop:gamma-l-connected}, we require a
further definition.  Suppose $T$ is an $X$-tree and $\cL$ is a topological
lasso for $T$. Then for all $v\in V(T)$, we denote by $\Gamma_v(\cL)$ the
subgraph of $\Gamma(\cL)$ induced by $L(v)$. Note that in case $v$ is a leaf
of $T$ and thus an element in $X$ the only vertex in $\Gamma_v(\cL)$ is $v$
(and $E(\Gamma_v(\cL))=\emptyset$).

\begin{pro}~\label{prop:gamma-l-connected} Suppose $T$ is an $X$-tree and
  $\cL$ is a topological lasso for $T$.  Then, for all $v\in V(T)$, the graph
  $\Gamma_v(\cL)$ is connected.  In particular, $\Gamma(\cL)$ is connected.
\end{pro}
\begin{proof}
  Assume for contradiction that there exists some vertex $v\in V(T)$ such that
  $\Gamma_v(\cL)$ is not connected. Then $v$ cannot be a leaf of $T$ and so
  $v\in \iV(T)$ must hold. Without loss of generality we may assume that $v$
  is such that for all descendants $w\in V(T)$ of $v$ the induced graph
  $\Gamma_w(\cL)$ is connected. Since $\cL$ is a topological lasso for $T$ and
  so $G(\cL,v)$ is a clique, it follows for any two distinct children
  $v_1,v_2\in ch(v)$ that there exists a pair $(x_1,x_2)\in L(v_1)\times
  L(v_2)$ such that $x_1x_2\in \cL$.  Since the assumption on $v$ implies that
  the graphs $\Gamma_{w}(\cL)$ are connected for all children $w\in ch(v)$, it
  follows that $\Gamma_v(\cL)$ is connected which is impossible.  Thus,
  $\Gamma_v(\cL)$ is connected, for all $v\in V(T)$.  That $\Gamma(\cL)$ is
  connected is a trivial consequence.  \qquad
\end{proof}

\section{The case that $\Gamma(\cL)$ is a block graph}
\label{sec:blockgraph}

To establish a further property of $\Gamma(\cL)$ which we will do in
Proposition~\ref{prop:x-i-unique}, we require some terminology related to
block graphs (see e.\,g.\,\cite{diestel}).  Suppose $G$ is a graph. Then a
vertex of $G$ is called a {\em cut vertex} if its deletion (plus its incident
edges) disconnects $G$. A graph is called a {\em block} if it has at least one
vertex, is connected, and does not contain a cut vertex. A {\em block of a
  graph $G$} is a maximal connected subgraph of $G$ that is a block and a
graph is called a {\em block graph} if all of its blocks are cliques. For
convenience, we refer to a block graph with vertex set $X$ as a {\em block
  graph on X}.

As the example of the two minimal topological lassos $
%\cL=
\{ab,cd,ef,ac,ce,ea\}$ and $
%\cL'=
\{ab, bc,$ $cd, de, ef, fa\}$ for the $\{a,\ldots,f\}$-tree depicted in
Fig.~\ref{fig:block-graph-motivation}~(ii) indicates, the graph $\Gamma(\cL)$
associated to a minimal topological lasso $\cL$ may be but need not be a block
graph.  However if it is then Lemma~\ref{lem:size-A(v)} can be strengthened to
the following central result where for all positive integers $n$ we put
$\langle n\rangle :=\{1,\ldots, n\}$ and set $\langle 0\rangle:=\emptyset$.

\begin{pro}\label{prop:x-i-unique}
  Suppose $T$ is an $X$-tree and $\cL$ is a minimal topological lasso for $T$
  such that $\Gamma(\cL)$ is a block graph.  Let $v\in \iV(T)$ and let
  $v_1,\ldots, v_l\in V(T)$ denote the children of $v$ where $l=|ch(v)|$.
  Then, for all $i\in \langle l\rangle$, there exists a unique leaf $x_i\in
  L(v_i)$ such that $x_sx_t\in \cL$ holds for all $s,t\in\langle l\rangle$
  distinct.
\end{pro}
\begin{proof}
  For all $v\in \iV(T)$ and all $w\in ch(v)$, put
$$
L^v_w:=\{x\in L(w): \mbox{ there exist } w'\in ch(v)-\{w\}
\mbox{ and } y\in L(w')
\mbox{ such that } xy\in \cL  \}.
$$
We need to show that $|L^v_w|=1$ holds for all $v\in \iV(T)$ and all $w\in
ch(v)$. To see this, note first that since $G(\cL,v)$ is a clique for all
$v\in \iV(T)$, we have, for all $w\in ch(v)$ with $v\in \iV(T)$, that
$L^v_w\not=\emptyset$.  Thus, $|L^v_w|\geq 1$ holds for all such $v$ and $w$.

To establish equality, suppose there exists some interior vertex $v\in \iV(T)$
and some child $v_1\in ch(v)$ such that $|L^v_{v_1}|\geq 2$.  Choose two
distinct leaves $x_1$ and $y_1$ of $T$ contained in $L^v_{v_1}$ and denote the
parent edge of $v_1$ by $e_1$. Note that $v_1=v_{e_1}$.  Since $y_1\in
L^v_{v_1}$, there exists a child edge $e_2$ of $v$ distinct from $e_1$ and
some $x_2\in L(v_{e_2})$ such that $y_1x_2\in\cL$. In view of $x_1\in
L^v_{v_1}$, we distinguish between the cases that (i) $x_1z\not\in\cL$ holds
for all $z\in L(v_{e_2})$ and (ii) there exists some $z\in L(v_{e_2})$ such
that $x_1z\in\cL$.

Assume first that Case~(i) holds.  Then since $x_1\in L^v_{v_1}$ there exists
a further child edge $e_3$ of $v$ and some $y_3\in L(v_{e_3})$ such that
$x_1y_3\in\cL$.  Since, by Theorem~\ref{thm:child-edge-graph-complete},
$G(\cL,v)$ is a clique and so $\{e_2,e_3\}$ is an edge in $G(\cL,v)$, there
must exist leaves $y_2\in L(v_{e_2})$ and $x_3\in L(v_{e_3})$ such that
$y_2x_3\in\cL$. By Proposition~\ref{prop:gamma-l-connected}, the graphs
$\Gamma_{v_{e_i}}(\cL)$, $i=2,3$, are connected and, by definition, clearly do
not share a vertex. Hence, there must exist a cycle in $\Gamma(\cL)$ whose
vertex set contains $\bigcup_{j\in\langle 3\rangle} \{x_j,y_j\}$. But then
$x_1x_2\in \cL$ must hold since $\Gamma(\cL)$ is a block graph and so every
block in $\Gamma(\cL)$ is a clique. By Lemma~\ref{lem:size-A(v)} applied to
$e_1$ and $e_2$, it follows that $x_1=y_1$ as $x_1,y_1\in L(v_1)$ and
$y_1x_2\in \cL$ which is impossible.

Now assume that Case~(ii) holds, that is, there exists some $z\in L(v_{e_2})$
such that $x_1z\in\cL$. Then Lemma~\ref{lem:size-A(v)} applied to $e_1$ and
$e_2$ implies $x_1=y_1$ as $y_1x_2\in\cL$ also holds which is impossible.
\qquad
\end{proof}

To illustrate Proposition~\ref{prop:x-i-unique}, let $T$ be the $X$-tree
depicted in Fig.~\ref{fig:block-graph-motivation}~(ii) and let $\cL$ be the set
of cords of $X$ whose $\Gamma(\cL)$ graph is pictured in
Fig.~\ref{fig:block-graph-motivation}~(i).  Using the notation from
Proposition~\ref{prop:x-i-unique} and labelling the children of the root of
$T$ from left to right by $v_1$, $v_2$ and $v_3$ it is easy to see that
Proposition~\ref{prop:x-i-unique} holds for $x_1=a$, $x_2=c$ and $x_3=e$.

The next result is the main result of this section and lies at the heart of
Corollary~\ref{cor:bijection} which provides for an $X$-tree $T$ and a minimal
topological lasso $\cL$ for $T$ such that $\Gamma(\cL)$ is a block graph a
close link between the blocks of $\Gamma(\cL)$, the interior vertices of $T$
and, for all $v\in \iV(T)$, the child-edge graphs $G(\cL,v)$. To establish it,
we denote for all $v\in V(T)-\{\rho_T\}$ the parent edge of $v$ by $e_v$ and
the set of blocks of a graph $G$ by $Block(G)$.


\begin{thm}\label{theo:unique-block}
  Suppose $T$ is an $X$-tree and $\cL$ is a minimal topological lasso for $T$
  such that $\Gamma(\cL)$ is a block graph. Then, for all $v\in \iV(T)$, there
  exists a unique block $B\in Block(\Gamma(\cL))$ such that $v=lca_T(V(B))$.
\end{thm}
\begin{proof}
  We first show existence. Suppose $v\in \iV(T)$. Let $v_1,\ldots,v_l\in V(T)$
  denote the children of $v$ where $l=|ch(v)|$.  By
  Proposition~\ref{prop:x-i-unique}, there exists, for all $i\in\langle
  l\rangle$, a unique leaf $x_i\in L(v_i)$ such that, for all $s,t\in \langle
  l\rangle$ distinct, we have $x_sx_t\in \cL$. Put $A=\{x_1,\ldots,x_l\}$.
  Clearly, $v=lca_T(A)$ and the graph $G(v)$ with vertex set $A$ and edge set
  $E=\{\{x,y\}\in {A\choose 2}: xy\in\cL\}$ is a clique.  Then since
  $\Gamma(\cL)$ is a block graph there must exist a block $B\in
  Block(\Gamma(\cL))$ that contains $G(v)$ as an induced subgraph.

  We claim that the graphs $G(v)$ and $B$ are equal.  In view of the facts
  that $A\subseteq V(B)$, the blocks in a block graph are cliques, and $G(v)$
  is a clique it suffices to show that $V(B)\subseteq A$. Suppose for
  contradiction that there exists some $y\in V(B)-A$. Note first that $yx\in
  \cL$ must hold for all $x\in A$.  Next note that $y$ cannot be a descendant
  of $v$ since otherwise there would exist some $i\in\langle l\rangle$ such
  that $y\in L(v_i)$. Choose some $j\in \langle l\rangle-\{i\}$. Then
  Lemma~\ref{lem:size-A(v)} applied to $e_{v_i}$ and $e_{v_j}$ implies $x_i=y$
  as $yx_j,x_ix_j\in\cL$ which is impossible.

  Choose some $x\in A$ and put $w=lca_T(x,y)$. Then $v$ is a descendant of $w$
  and $w=lca_T(x,y)$ holds for all $x\in A$. Let $w_1\in V(T)$ and $w_2\in
  \iV(T)$ denote two distinct children of $w$ such that $y\in L(w_1)$ and
  $x\in L(w_2)$. Then Lemma~\ref{lem:size-A(v)} applied to $e_{w_1}$ and
  $e_{w_2}$ implies $x_i=x_j$ for all $i,j\in\langle l\rangle$ distinct since
  $yx\in\cL$ holds for all $x\in A$ which is impossible.  Thus, $V(B)\subseteq
  A$, as required. This concludes the proof of the existence part of the
  theorem.

  We next show uniqueness.  Suppose for contradiction that there exists some
  $v\in \iV(T)$ and distinct blocks $B,B'\in Block(\Gamma(\cL))$ such that
  $lca_T(B)=v=lca_T(B')$.  Since every block of $\Gamma(\cL)$ contains at
  least two vertices as $\Gamma(\cL)$ is connected and $|X|\geq 3$, we may
  choose distinct vertices $b_1,b_2\in V(B)$ and $b_1',b_2'\in V(B')$ such
  that $lca_T(b_1,b_2)=lca_T(B)=v=lca_T(B')=lca_T(b_1',b_2')$.  Note that
  $b_1b_2$ and $b_1'b_2' $ must be cords in $\cL$ as $B$ and $B'$ are cliques
  of $\Gamma(\cL)$.  We distinguish between the cases that (i)
  $\{b_1,b_2\}\cap\{b_1',b_2'\}=\emptyset $ and (ii)
  $\{b_1,b_2\}\cap\{b_1',b_2'\}\not=\emptyset $.

  We first show that Case (i) cannot hold. Assume for contradiction that Case
  (i) holds, that is, $\{b_1,b_2\}\cap\{b_1',b_2'\}=\emptyset $.  We claim
  that $lca_T(b_1,b_1')=v$. Assume for contradiction that
  $w:=lca_T(b_1,b_1')\not=v$. Let $v_1\in ch(v)$ such that $v_1$ lies on the
  path from $v$ to $w$. If $v\not =lca_T(b_2,b_2')$ then there exists a
  descendant $w'\in V(T)$ of $v$ such that $lca_T(b_2,b_2')=w'$. Let $v_2\in
  ch(v)$ such that $v_2$ that lies on the path from $v$ to $w'$. Then
  Lemma~\ref{lem:size-A(v)} applied to $e_{v_1}$ and $e_{v_2}$ implies
  $b_1=b_1'$ and $b_2=b_2'$ as $b_1b_2,b_1'b_2'\in \cL$ which is
  impossible. Thus, $lca_T(b_2,b_2')=v$ must hold. Let $v_2,v_2'\in ch(v)$
  such that $b_2\in L(v_2)$ and $b_2'\in L(v_2')$.  Then since $b_1,b_1'\in
  L(v_1)$ and $b_1b_2, b_1'b_2'\in \cL$, Proposition~\ref{prop:x-i-unique}
  implies $b_1'=b_1$.  Consequently,
  $\{b_1,b_2\}\cap\{b_1',b_2'\}\not=\emptyset $ which is impossible.

  Thus, $lca_T(b_2,b_2')=v$ cannot hold and so
  $$
  lca_T(b_1,b_1')=v,
  $$ 
  as claimed.  Swapping the roles of $b_1,b_1'$ and $b_2,b_2'$ in the previous
  claim implies that $v= lca_T(b_2,b_2')$ must hold, too.  For $i=1,2$ let
  $v_i,v_i'\in ch(v)$ such that $b_i\in L(v_i)$ and $b_i'\in L(v_i')$.  Then,
  by Lemma~\ref{lem:size-A(v)}, there exist pairs $(c,c')\in L(v_1)\times
  L(v_1')$ and $(d,d')\in L(v_2)\times L(v_2')$ such that $cc',dd'\in\cL$.
  Since $(b_1,b_2)\in L(v_1)\times L(v_2)$ and $(b_1',b_2')\in L(v_1')\times
  L(v_2')$ and $b_1b_2,b_1'b_2'\in\cL$, Proposition~\ref{prop:x-i-unique}
  implies that $c=b_1$, $b_2=d$, $d'=b_2'$ and $c'=b_1'$. But then $C$:
  $c'=b_1',b_2'=d', d=b_2, b_1=c,c'$ is a cycle in $\Gamma(\cL)$.  Since
  $\Gamma(\cL)$ is a block graph it follows that there must exist a block
  $B^C$ in $\Gamma(\cL)$ that contains $C$.  Since $\{b_1,b_2\}\subseteq
  V(B^C)\cap V(B)$ and two distinct blocks of a block graph can share at most
  one vertex it follows that $B^C$ and $B$ must coincide. Since
  $\{b_1',b_2'\}\subseteq V(B^C)\cap V(B')$ holds too, similar arguments imply
  that $B^C$ must also coincide with $B'$.  Thus, $B$ and $B'$ must be equal
  which is impossible.  Hence Case~(i) cannot hold, as required.

  Thus, Case (ii) must hold, that is,
  $\{b_1,b_2\}\cap\{b_1',b_2'\}\not=\emptyset $. Since any two distinct blocks
  in a block graph can share at most one vertex it follows that
  $|\{b_1,b_2\}\cap\{b_1',b_2'\}|=1$.  Without loss of generality we may
  assume that $b_1=b_1'$.  We first claim that
  $$
  lca_T(b_2,b_2')=v.
  $$
  Assume to the contrary that $lca_T(b_2,b_2')\not=v$.  Then there exist
  distinct children $v_1,v_2 \in ch(v)$ such that $b_1\in L(v_1)$ and
  $b_2,b_2'\in L(v_2)$ hold.  Since both $b_1b_2$ and $b_1'b_2'=b_1b_2'$ are
  cords in $\cL$, Lemma~\ref{lem:size-A(v)} applied to $e_{v_1}$ and $e_{v_2}$
  implies $b_2'=b_2$. Hence, $|\{b_1,b_2\}\cap\{b_1',b_2'\}|=2$ which is
  impossible. Thus, $lca_T(b_2,b_2')=v$, as claimed.

  Let $v_1,v_2,v_2'\in ch(v)$ such that $b_1\in L(v_1)$, $b_2\in L(v_2)$, and
  $b_2'\in L(v_2')$.  By Lemma~\ref{lem:size-A(v)}, there exist some
  $(c,c')\in L(v_2)\times L(v_2')$ such that $cc'\in \cL$.  Since we also have
  $(b_1,b_2)\in L(v_1)\times L(v_2)$ with $b_1b_2\in \cL$ holding and
  $(b_1,b_2')\in L(v_1)\times L(v_2')$ with $b_2'b_1=b_2'b_1'\in \cL$ holding,
  Proposition~\ref{prop:x-i-unique} implies that $b_2=c$ and $b_2'=c'$. Hence,
  $C$: $b_1=b_1',b_2'=c', c=b_2,b_1$ is a cycle in $\Gamma(\cL)$ and so
  similar arguments as in the corresponding subcase for Case (i) imply that

  $B$ and $B'$ must coincide which is impossible. Thus, $lca_T(b_2,b_2')=v$
  cannot hold which concludes the discussion of Case (ii) and thus the proof
  of the uniqueness part of the theorem.
\end{proof}

In view of Theorem~\ref{theo:unique-block}, we denote for $T$ an $X$-tree, a
minimal topological lasso $\cL$ for $T$ such that $\Gamma(\cL)$ is a block
graph, and a vertex $v\in \iV(T)$ the unique block $B$ in $\Gamma(\cL)$ for
which $v=lca_T(V(B))$ holds by $B_v^{\cL}$, or simply by $B_v$ if the set
$\cL$ of cords is clear from the context.  Moreover, we denote for all $x\in
L(v)$ the child of $v$ on the path from $v$ to $x$ by $v_x$.

\begin{cor}\label{cor:bijection}
  Suppose $T$ is an $X$-tree and $\cL$ is a minimal topological lasso for $T$
  such that $\Gamma(\cL)$ is a block graph. Then the map
  $$
  \psi:\iV(T)\to Block(\Gamma(\cL)) : v\mapsto B_v
  $$
  is a bijection with inverse map $\psi^{-1}:Block(\Gamma(\cL))\to \iV(T)$:
  $B\mapsto lca_T(V(B))$. Moreover, the map
  $$
  \chi:Block(\Gamma(\cL)) \to \{G(\cL, v)\,:\, v\in \iV(T)\}
  : B\mapsto G(\cL,\psi^{-1}(B))
  $$
  is bijective and, for all $B\in Block(\Gamma(\cL))$, the map
  $$
  \xi_B:V(B) \to V_{\psi^{-1}(B)} 
  : x\mapsto e_{\psi^{-1}(B)_x}
  $$ 
  induces a graph isomorphism between $B$ and the child-edge graph $G(\cL,
  \psi^{-1}(B))$.
\end{cor}
\begin{proof}
  In view of Theorem~\ref{theo:unique-block}, the map $\psi$ is clearly
  well-defined and injective. To see that $\psi$ is surjective let $B\in
  Block(\Gamma(\cL))$ and put $v_B=lca_T(V(B))$. Clearly, $v_B\in
  \iV(T)$. Since $B_{v_B}=\psi(v_B)$ is a block in $\Gamma(\cL)$ for which
  also $v_B=lca_T(V(B_{v_B}))$ holds, Theorem~\ref{theo:unique-block} implies
  that $\psi(v_B)$ and $B$ must coincide.  Consequently, $\psi$ must also be
  surjective and thus bijective. That the map $\psi^{-1}$ is as stated is
  trivial.  Combined with Theorem~\ref{thm:child-edge-graph-complete}, the
  bijectivity of the map $\psi$ implies in particular that, for all $B\in
  Block(\Gamma(\cL))$, the map $\xi_B: V(B) \to V_{\psi^{-1}(B)} $ from $V(B)$
  to the vertex set $V_{\psi^{-1}(B)} $ of the child-edge graph $G(\cL,
  \psi^{-1}(B))$ induces a graph isomorphism between $B$ and $G(\cL,
  \psi^{-1}(B))$.

  To see that the map $\chi$ is bijective note first that $\chi$ is
  well-defined since $\psi^{-1}(B)\in \iV(T)$ holds for all blocks $B\in
  Block(\Gamma(\cL))$. To see that $\chi$ is injective assume that there exist
  blocks $B_1,B_2\in Block(\Gamma(\cL))$ such that $\chi(B_1)=\chi(B_2)$ but
  $B_1$ and $B_2$ are distinct.  Then $\psi^{-1}(B_1)\not= \psi^{-1}(B_2)$ as
  $\psi$ is a bijection from $\iV(T)$ to $Block(\Gamma(\cL)) $.  Combined with
  the fact that, for all $B\in Block(\Gamma(\cL))$, the map $\xi_B$ induces a
  graph isomorphism between $B$ and $G(\cL, \psi^{-1}(B))$ it follows that
  $\chi(B_1)=G(\cL,\psi^{-1}(B_1))\not=G(\cL,\psi^{-1}(B_2))=\chi(B_2)$ which
  is impossible. Thus, $\chi$ must be injective. Combined with the fact that
  $|Blocks(\Gamma(\cL))|=|\iV(T)|=|\{G(\cL, v)\,:\, v\in \iV(T)\}|$ it follows
  that $\chi$ must also be surjective and thus bijective.
\end{proof}



\section{A special type of minimal topological lasso}
\label{sec:distinguished}

Returning to the example depicted in Fig.~\ref{fig:block-graph-motivation}, it
should be noted that, in addition to being a block graph, $\Gamma(\cL)$ enjoys
a very special property where $\cL$ is the minimal topological lasso
considered in that example. More precisely, every vertex of $\Gamma(\cL)$ is
contained in at most two blocks.  Put differently, $\Gamma(\cL)$ is a
claw-free graph. Motivated by this, we call a minimal topological lasso $\cL$
{\em distinguished} if $\Gamma(\cL)$ is a claw-free block graph.  Note that
such block graphs are precisely the {\em line graphs of (unrooted) trees}
where for any graph $G$ the associated line graph has vertex set $E(G)$ and
two vertices $a,b\in E(G)$ are joined by an edge if $a\cap b\not=\emptyset$
\cite{H72}.

In this section, we show in Theorem~\ref{theo:transform} that distinguished
minimal topological lassos are a very special type of lasso in that for every
$X$-tree $T$ any minimal topological lasso $\cL$ for $T$ can be transformed
into a distinguished minimal topological lasso $\cL^*$ for $T$ via a {\em
  repeated application} (that is, $l\geq 0$ applications) of the rule:

\begin{enumerate}
\item[(R)] If $xy,yz\in \cL$ and $lca_T(y,z)$ is a descendant of $lca_T(x,y)$
  in $T$ then delete $xy$ from the edge set of $\Gamma(\cL)$ and add the edge
  $xz$ to it.
\end{enumerate}

Before we make this more precise which we will do next, we remark that since a
topological lasso for a star tree is in particular a distinguished minimal
topological lasso for it, we will for this and the next two sections restrict
our attention to {\em non-degenerate} $X$-trees, that is, $X$-trees that are
not star trees on $X$.

Suppose $T$ is a non-degenerate $X$-tree and $\cL$ is a set of cords of
$X$. Let $\iV(T)$ denote a set of colours and let
$$
\gamma_{(\cL,T)}:\cL\to \iV(T):\, ab\mapsto lca_T(a,b)
$$
denote an edge colouring of $\Gamma(\cL)$ in terms of the interior vertices of
$T$. Note that if $\cL$ is a topological lasso for $T$ then
Theorem~\ref{thm:child-edge-graph-complete} implies that $\gamma_{(\cL,T)}$
is surjective. Returning to Rule (R), note that a repeated application of that
rule to such a set $\cL$ of cords results in a set $\cL'$ of cords that is
also a topological lasso for $T$. Furthermore, note that if $\cL$ is a minimal
topological lasso for $T$ then $\cL'$ is necessarily also a minimal
topological lasso for $T$. Finally note for all $v\in \iV(T)$ that
$|\gamma_{(\cL,T)}^{-1}(v)|=1$ or $|\gamma_{(\cL,T)}^{-1}(v)|\geq 3$ must hold
in this case.


\begin{lem}
  \label{lem:3-edges-cycle}
  Suppose $T$ is a non-degenerate $X$-tree and $\cL$ is a minimal topological
  lasso for $T$.  Put $\gamma=\gamma_{(\cL,T)}$ and assume that $v\in \iV(T)$
  such that $|\gamma^{-1}(v)|\geq 3 $. Then for any three pairwise distinct
  cords $c_1,c_2,c_3\in \gamma^{-1}(v)$, there exists a cycle $C_v$ in
  $\Gamma(\cL)$ such that $c_1,c_2,c_3\in E(C_v)$ and, for all $c\in E(C_v)$,
  $\gamma(c)$ either equals $v$ or is a descendant of $v$.
\end{lem}
\begin{proof}
  Let $v\in \iV(T)$ and let $c_1=x_1y_1$, $c_2=x_2y_2$ and $c_3=x_3y_3$ denote
  three pairwise distinct cords in $\gamma^{-1}(v)$.  For all $i\in\langle
  3\rangle$, let $v_i\in ch(v) $ such that $v_i$ lies on the path from $v$ to
  $x_i$ in $T$ and let $w_i\in ch(v)$ such that $w_i$ lies on the path from
  $v$ to $y_i$ in $T$.  Then, by Lemma~\ref{lem:size-A(v)}, there exists
  unique pairs $(s_1,t_1)\in L(v_1)\times L(v_2)$, $(s_2,t_2)\in L(w_2)\times
  L(w_3)$, and $(s_3,t_3)\in L(w_1)\times L(v_3)$ such that, for all
  $i\in\langle 3\rangle$, we have $s_it_i\in \cL$.  Since for all such $i$, we
  also have that $x_i\in L(v_i)$ and $y_i\in L(w_i)$ and, by
  Proposition~\ref{prop:gamma-l-connected}, the graphs $\Gamma_{v_i}(\cL)$ and
  $\Gamma_{w_i}(\cL)$ are connected, it follows that there exists a cycle
  $C_v$ in $\Gamma(\cL)$ that contains, for all $i\in\langle 3\rangle$, the
  cords $c_i$ and $s_it_i$ in its edge set.

  It remains to show that for every edge $c\in E(C_v)$, we have that
  $\gamma(c)$ either equals $v$ or is a descendant of $v$.  Suppose $c\in
  E(C_v)$.  If there exists some $i\in\langle 3\rangle$ such that
  $c\in\{c_i,s_it_i\}$ then $\gamma (c)=v$ clearly holds. So assume that this
  is not the case. Without loss of generality, we may assume that $c$ lies on
  the path $P$ from $x_1$ to $s_1$ in $C_v$ that does not cross $y_1$. Since
  $P$ is a subgraph of $\Gamma_{v_1}(\cL)$ and, implied by
  Proposition~\ref{prop:gamma-l-connected}, every edge in $\Gamma_{v_1}(\cL)$
  is coloured via $\gamma$ with a descendant of $v_1$, it follows that
  $\gamma(c)$ is a descendant of $v$.
\end{proof}

To establish Theorem~\ref{theo:transform}, we require further terminology.
Suppose $T$ is a non-degenerate $X$-tree, $\cL$ is a minimal topological lasso
for $T$, and $v\in \iV(T)$.  Then we denote by $H_{\cL}(v)$ the induced
subgraph of $\Gamma(\cL)$ whose vertex set is the set of all $x\in X$ that are
incident with some cord $c\in \cL$ for which $\gamma_{(\cL,T)}(c)=v$ holds.
Moreover, we denote the set of cut vertices of a connected block graph $G$ by
$Cut(G)$ and note that in every connected block graph $G$ there must exist a
vertex that is contained in at most one block of $G$. This last observation is
central to the proof of Theorem~\ref{theo:transform}~(ii).

\begin{thm}
  \label{theo:transform}
  Suppose $T$ is a non-degenerate $X$-tree and $\cL$ is a minimal topological
  lasso for $T$. Then there exists an ordering $\sigma: v_0, v_1,\ldots,
  v_k=\rho_T$, $k=|\iV(T)|$, of $\iV(T)$ such that the following holds:
  \begin{enumerate}
  \item[(i)] There exists a sequence $ \cL_{v_0}=\cL,\cL_{v_1},\ldots,
    \cL^{\dagger}=\cL_{v_{k}}$ of minimal topological lassos $\cL_{v_i}$ for
    $T$, $i\in \langle k\rangle$, such that for all such $i$, we have:
    \begin{enumerate}
    \item[(L1)] $\cL_{v_i}$ is obtained from $\cL_{v_{i-1}}$ via a repeated
      application of Rule (R) and $H_{\cL_{v_i}}(v_i)$ is a maximal clique in
      $\Gamma(\cL_{v_i})$.

    \item[(L2)] For all $j\in\langle i-1\rangle$, $H_{\cL_{v_{i}}}(v_j)$ is a
      maximal clique in $\Gamma(\cL_{v_{i}})$.
    \end{enumerate}
    In particular, $\Gamma(\cL^{\dagger})$ is a block graph.
  \item[(ii)] If $\Gamma(\cL)$ is a block graph then there exists a sequence
    $\cL_{v_0}=\cL,\cL_{v_1},\ldots, \cL^*=\cL_{v_{k}}$ of minimal topological
    lassos $\cL_{v_i}$ for $T$, $i\in\langle k\rangle$, such that for all such
    $i$, we have:
    \begin{enumerate}
    \item[(L1')] $\cL_{v_i}$ is obtained from $\cL_{v_{i-1}}$ via a repeated
      application of Rule (R) and $\Gamma(\cL_{v_i})$ is a block graph.
    \item[(L2')] $\Gamma_{v_i}(\cL_{v_i})$ is a claw-free block graph.
    \end{enumerate}
    In particular, $\cL^*$ is a distinguished minimal topological lasso for
    $T$.
  \end{enumerate}
\end{thm}
\begin{proof}
  For all $i\in \langle k\rangle$, put $\cL_i=\cL_{v_{i}}$ and
  $\gamma_i=\gamma_{(\cL_i,T)}$.  Clearly, if $\cL$ is distinguished then the
  sequences as described in (i) and (ii) exist. So assume that $\cL$ is not
  distinguished.  For all $v\in \iV(T)$, let $l(v)$ denote the length of the
  path from the root $\rho_T$ of $T$ to $v$ and put $h=\max_{v\in
    \iV(T)}\{l(v)\}$.  Note that $h\geq 1$ as $T$ is non-degenerate.  For all
  $i\in \langle h\rangle$, let $V(i)\subseteq \iV(T)$ denote the set of all
  interior vertices $v$ of $T$ such that $l(v)=i$.  Let $\sigma$ denote an
  ordering of the vertices in $\iV(T)$ such that the vertices in $V(h)$ come
  first (in any order), then (again in any order) the vertices in $V(h-1)$ and
  so on with the last vertex in that ordering being $\rho_T$.

  (i) Suppose $v\in \iV(T)$. If $v\in V(h)$ then we may assume without loss of
  generality that $v=v_1$. Then $v_1$ is the parent of a pseudo-cherry of $T$
  and so Theorem~\ref{thm:child-edge-graph-complete} implies that
  $H_{\cL}(v_1)$ is a maximal clique in $\Gamma(\cL)$. Thus, $\cL_1:=\cL$ is a
  minimal topological lasso for $T$ that satisfies Properties~(L1) and (L2).

  So assume that $v\not\in V(h)$. Then there exists some $|V(h)|< i\leq k$
  such that $v=v_i$. Without loss of generality, we may assume that $v_i$ is
  such that, for all $j\in \langle i-1\rangle$, $\cL_j$ is a minimal
  topological lasso for $T$ that satisfies Properties~(L1) and (L2).  If $v_i$
  is the parent of a pseudo-cherry of $T$ then similar arguments as before
  imply that $\cL_i:=\cL_{i-1}$ is a minimal topological lasso for $T$ that
  satisfies Properties~(L1) and (L2). So assume that $v_i$ is not the parent
  of a pseudo-cherry of $T$.  We distinguish between the cases that
  $H_{\cL_{i-1}}(v)$ is a maximal clique in $\cL_{i-1}$ and that it is not.

  Assume first that $H_{\cL_{i-1}}(v)$ is a maximal clique in
  $\cL_{i-1}$. Then since $\cL_{i-1}$ is a minimal topological lasso for $T$
  that satisfies Properties~(L1) and (L2), it is easy to see that
  $\cL_{i}:=\cL_{i-1}$ is also a minimal topological lasso for $T$ that
  satisfies Properties~(L1) and (L2).  So assume that $H_{\cL_{i-1}}(v)$ is
  not a maximal clique in $\cL_{i-1}$.  Then $H_{\cL_{i-1}}(v)$ must contain
  three pairwise distinct edges, $e_1=x_1y_1$, $e_2=x_2y_2$, and $e_3=x_3y_3$
  say, such that $\{e_1, e_2,e_3\}$ is not the edge set of a $3$-clique in
  $H_{\cL_{i-1}}(v)$.  For all $i\in\langle 3\rangle $, put
  $z_i=lca_T(x_i,y_i)$.  Then Lemma~\ref{lem:3-edges-cycle} combined with a
  repeated application of Rule (R) to $\cL_{i-1}$ implies that, for all
  $i\in\langle 3\rangle$, we can find elements $x_i'\in L(z_i)$ such that
  $$
  \cL_{i-1}'=\cL_{i-1}-\{x_1y_1,x_2y_2,x_3y_3\}\cup 
  \{x_1'x_2', x_2'x_3',x_3'x_1'\}
  $$
  is a minimal topological lasso for $T$ and the cords $x_1'x_2'$, $x_2'x_3'$,
  and $x_3'x_1'$ form a $3$-clique in $H_{\cL_{i-1}'}(v)$.  Transforming
  $\cL_{i-1}'$ further by processing any three pairwise distinct edges in
  $H_{\cL_{i-1}'}(v)$ that do not already form a $3$-clique in the same way
  and so on eventually yields a minimal topological lasso $\cL_{i}$ for $T$
  such that any three pairwise distinct edges in $H_{\cL_{i}}(v)$ form a
  $3$-clique. But this implies that $H_{\cL_{i}}(v)$ is a maximal clique in
  $\Gamma(\cL_{i})$ and so Property~(L1) is satisfied by $\cL_i$. Since only
  edges $e$ of $\Gamma(\cL_{i-1})$ have been modified by the above
  transformation for which $\gamma_{i-1}(e)=v$ holds and, by assumption,
  $\cL_{i-1}$ satisfies Property~(L2) it follows that $\cL_{i}$ also satisfies
  that property.

  Processing the successor of $v_i$ in $\sigma$ in the same way and so on
  yields a minimal topological lasso $\cL^{\dagger}$ for $T$ for which
  $\Gamma(\cL^{\dagger})$ is a block graph. This completes the proof of (i).


  (ii) For all $i\in \langle k\rangle$ and all vertices $w\in \iV(T)$ put
  $B^i_w=B^{\cL_i}_w$.  Suppose that $v\in \iV(T)$. If $v\in V(h)$ then we may
  assume without loss of generality that $v=v_1$. Then $v$ is the parent of a
  pseudo-cherry of $T$ and so $\cL_1:=\cL$ clearly satisfies Properties~(L1')
  and (L2').

  So assume that $v\not\in V(h)$. Then there exists some $|V(h)|< i\leq k$
  such that $v=v_i$. Without loss of generality, we may assume that $v_i$ is
  minimal, that is, for all $j\in\langle i-1\rangle$, we have that $\cL_j$ is
  a minimal topological lasso for $T$ that satisfies Properties~(L1') and
  (L2'). If $v$ is the parent of a pseudo-cherry of $T$ then similar arguments
  as before imply that $\cL_i:=\cL_{i-1}$ satisfies Properties~(L1') and
  (L2'). So assume that $v$ is not the parent of a pseudo-cherry of $T$.  If
  $\Gamma_v(\cL_{i-1})$ is a claw-free block graph then setting
  $\cL_i:=\cL_{i-1}$ implies that $\cL_i$ satisfies Properties~(L1') and
  (L2').

  So assume that this is not the case, that is, there exists a vertex $x\in
  L(v)$ that, in addition to being a vertex in the block $B_v^{i-1}$ of
  $\Gamma(\cL_{i-1})$ and thus of $\Gamma_v(\cL_{i-1})$, is also a vertex in
  $l\geq 2$ further blocks $B_1,\ldots, B_l$ of $\Gamma_v(\cL_{i-1})$ which
  are also blocks in $\Gamma(\cL)$. Then there exists a path $P$ from $v$ to
  $x$ in $T$ that contains, for all $l\geq 2$, the vertices
  $\psi^{-1}(B_1),\ldots, \psi^{-1}(B_l)$ in its vertex set where
  $\psi:\iV(T)\to \Block(\Gamma(\cL))$ is the map from
  Corollary~\ref{cor:bijection}. Let $w\in ch(v)$ denote the child of $v$ that
  lies on $P$. Note that since $l\geq 2$, we have $w\in \iV(T)$. Without loss
  of generality, we may assume that $w=v_{i-1}$.  The fact that
  $\Gamma(\cL_{i-1})$ is a block graph and so $\Gamma_{v_{i-1}}(\cL_{i-1})$ is
  a block graph combined with the fact that $\Gamma_{v_{i-1}}(\cL_{i-1})$ is
  connected implies, in view of the observation preceding
  Theorem~\ref{theo:transform}, that we may choose some $y\in
  L(v_{i-1})-Cut(\Gamma_{v_{i-1}}(\cL_{i-1}))$. Then $y$ is a vertex in
  precisely one block of $\Gamma_{v_{i-1}}(\cL_{i-1})$ and thus can be a
  vertex in at most two blocks of $\Gamma_v(\cL_{i-1})$.  Consequently,
  $y\not=x$.  Applying Rule (R) repeatedly to $\cL_{i-1}$, let $\cL_i$ denote
  the set of cords obtained from $\cL_{i-1}$ by replacing, for all $i\leq
  l\leq k$, every cord of $\cL_{i-1}$ of the form $xa$ with $a\in
  V(B_{v_l}^{i-1})$ by the cord $ya$. Then, by construction, $\cL_i$ is a
  minimal topological lasso for $T$ and $\Gamma(\cL_i)$ is a block
  graph. Hence, $\cL_i$ satisfies Property~(L1'). Moreover, since
  $\Gamma_{v_{i-1}}(\cL_{i-1})$ is claw-free it follows that
  $\Gamma_{v_i}(\cL_i)$ is claw-free and so $\cL_i$ satisfies Property~(L2'),
  too.

  Applying the above arguments to the successor of $v_i$ in $\sigma$ and so on
  eventually yields a minimal topological lasso $\cL_k$ for $T$ that satisfies
  Properties~(L1') and (L2'). Thus, $\Gamma_{v_k}(\cL_k) $ is a claw-free
  block graph and, so, $\cL^*$ is a distinguished minimal topological lasso
  for $T$.
\end{proof}

To illustrate Theorem~\ref{theo:transform}, let $X=\{a,\ldots, f\}$ and
consider the $X$-tree $T'$ depicted in
Fig.~\ref{fig:block-graph-motivation}~(iii) along with the set
$\cL=\{ad,ec,fa,ef,cd,bd\}$ of cords of $X$ which we depict in
Fig.~\ref{fig:transformation}~(i) in the form of $\Gamma(\cL)$.

\begin{figure}
  \begin{center}
    \input{figures/dist-min-lass/transformation.pdft}
  \end{center}
  \caption{ For $X=\{a,\ldots, f\}$ and the $X$-tree $T'$ pictured in
    Fig.~\ref{fig:block-graph-motivation}~(iii), we depict in (i) the minimal
    topological lasso $\cL=\{ad,ec,fa,fe,cd,bd\}$ for $T'$ in the form of
    $\Gamma(\cL)$.  In the same way as in (i), we depict in (ii) the
    transformed minimal topological lasso $\cL^{\dagger}$ for $T'$ such that
    $\Gamma(\cL^{\dagger})$ is a block graph and in (iii) the distinguished
    minimal topological lasso $\cL^*$ for $T'$ obtained from $\cL^{\dagger}$
    -- see text for details.}
  \label{fig:transformation}
\end{figure}
%
Using for example Theorem~\ref{thm:child-edge-graph-complete}, it is
straight-forward to check that $\cL$ is a minimal topological lasso for $T'$
but $\Gamma(\cL)$ is clearly not a block graph and so $\cL$ is also not
distinguished. To transform $\cL$ into a distinguished minimal topological
lasso $\cL^*$ for $T'$ as described in Theorem~\ref{theo:transform}, consider
the ordering $v_1=lca_{T'}(e,f)$, $v_2=lca_{T'}(c,d)$, $v_3=lca_{T'}(a,d)$,
$v_4=\rho_{T'}$ of the interior vertices of $T'$. For all $i\in\langle
4\rangle$, put $\cL_i=\cL_{v_i}$. Then we first transform $\cL$ into a minimal
topological lasso $\cL^{\dagger}$ for $T'$ as described in
Theorem~\ref{theo:transform}~(i). For this we have $\cL=\cL_0=\cL_1=\cL_2$ and
$\cL_3$ is obtained from $\cL_2$ by first applying Rule (R) to the cords $ec,
cd \in \cL_2$ resulting in the deletion of the cord $ce$ from $\cL_2 $ and the
addition of the cord $ed$ to $\cL_2$ and then to the cords $fe,ed\in\cL_2$
resulting in the deletion of the cord $ed$ from $\cL_2$ and the addition of
the cord $fd$ to it. The graph $\Gamma(\cL_3)$ is depicted in
Fig.~\ref{fig:transformation}~(ii).  Note that $\cL_3=\cL^{\dagger}$ and that
although $\Gamma(\cL^{\dagger})$ is clearly a block graph $\cL^{\dagger}$ is
not distinguished.

To transform $\cL^{\dagger}$ into a distinguished minimal topological lasso
$\cL^*$ for $T'$, we next apply Theorem~\ref{theo:transform}~(ii). For this, we
need only consider the vertex $d$ of $\Gamma(\cL^{\dagger})$ that is, we have
$\cL^{\dagger}=\cL_0=\cL_1=\cL_2=\cL_3$.  Since the child of $v_4$ on the path
from $v_4$ to $d$ is $v_3$, we may choose $a$ as the element $y$ in
$L(v_3)-Cut(\Gamma_{v_3}(\cL_3))$. Then applying Rule (R) to the cords
$bd,da\in \cL_3$ implies the deletion of $bd$ from $\cL_3$ and the addition of
the cord $ab$ to it. The resulting minimal topological lasso for $T'$ is
$\cL^*$ which we depict in Fig.~\ref{fig:transformation}~(iii) in the form of
$\Gamma(\cL^*)$.
 
We conclude this section by remarking in passing that combined with
Theorem~\ref{thm:child-edge-graph-complete} which implies that any minimum
sized topological lasso for an $X$-tree $T$ must have $\sum_{v\in
  \iV(T)}{|ch(v)|\choose 2}$ cords, Theorem~\ref{theo:transform} and
Corollary~\ref{cor:bijection} imply that the minimum sized topological lassos
of an $X$-tree $T$ are precisely the minimal topological lassos of $T$.
 
\section{A sufficient condition for being distinguished}
\label{sec:sufficient}
In this section, we turn our attention towards presenting a sufficient
condition for a minimal topological lasso for some $X$-tree $T$ to be a
distinguished minimal topological lasso for $T$.  In the next section, we will
show that this condition is also necessary.

We start our discussion with introducing some more terminology.  Suppose $T$
is a non-degenerate $X$-tree. Put $cl(T)=\{L(v): v\in \iV(T)-\{\rho_T\}\}$ and
note that $cl(T)\not=\emptyset$. For all $A\in cl(T)$, put $cl_A(T):=\{B\in
cl(T): B\subsetneq A\}$ and note that a vertex $v\in \iV(T)-\{\rho_T\}$ is the
parent of a pseudo-cherry of $T$ if and only if $cl_{L(v)}(T)=\emptyset$.  For
$\sigma$ a total ordering of $X$ and $\min_{\sigma}(C)$ denoting the minimal
element of a non-empty subset $C$ of $X$, we call a map of the form
$$
f:cl(T)\to X:
A\mapsto \left\{\begin{array}{cc}
\min_{\sigma}(A-\{f(B): B\in cl_A(T)\})
 & \mbox{ if }cl_A(T)\not=\emptyset,\\
\min_{\sigma}(A)  & \mbox{ else. }
\end{array}
\right.
$$ 
a {\em cluster marker map (for $T$ and $\sigma$)}.  Note that since
$|\iV(T')|\leq |X|-1$ holds for all $X$-trees $T'$ and so $A-\{f(B): B\in
cl_A(T)\}\not=\emptyset$ must hold for all $A\in cl(T)$ with
$cl_A(T)\not=\emptyset $, it follows that $f$ is well-defined.  Also note that
if $v\in \iV(T)$ is the parent of a pseudo-cherry $C$ of $T$ then
$f(L(v))=f(C)= \min_{\sigma}(C)$ as $cl_C(T)=\emptyset$ in this case. Finally,
note that it is easy to see that a cluster marker map must be injective but
need not be surjective.


We are now ready to present a construction of a distinguished minimal
topological lasso which underpins the aforementioned sufficient condition that
a minimal topological lasso must satisfy to be distinguished.  Suppose that
$T$ is a non-degenerate $X$-tree, that $\sigma$ is a total ordering of $X$,
and that $f:cl(T)\to X$ is a cluster marker map for $T$ and $\sigma$. We first
associate to every interior vertex $v\in \iV(T)$ a set $\cL_{(T,f)} (v)$
defined as follows. Let $l_1,\ldots, l_{k_v}$ denote the children of $v$ that
are leaves of $T$ and let $v_1,\ldots v_{p_v}$ denote the children of $v$ that
are also interior vertices of $T$. Note that $k_v=0$ or $p_v=0$ might hold but
not both. Put ${\emptyset \choose 2}={\langle 1\rangle \choose
  2}=\emptyset$. Then we set
$$
\cL_{(T,f)}(v):=\bigcup_{\{i,j\}\in {\langle k_v\rangle\choose 2}}\{l_il_j\}
\cup
\bigcup_{\{i,j\}\in {\langle p_v\rangle\choose 2}}\{f(L(v_i))f(L(v_j))\}
\cup
\bigcup_{i\in \langle k_v\rangle,\,\,j\in \langle p_v\rangle} 
\{l_if(L(v_j))\}.
$$
Note that $|\cL_{(T,f)}(v)|\geq 1$ must hold for all $v\in \iV(T)$. Finally,
we set
$$
\cL_{(T,f)}:=\bigcup_{v\in \iV(T)} \cL_{(T,f)}(v).
$$

To illustrate these definitions, consider the $X=\{a,\ldots, f\}$-tree $T'$
depicted in Fig.~\ref{fig:block-graph-motivation}~(iii).  Let $\sigma$ denote
the lexicographic ordering of the elements in $X$. Then the map $f:cl(T')\to
X$ defined by setting
$$
f(\{c,d\})=c, \,\,\,
f(\{e,f\})=e,\,\,\mbox{ and } f(X-\{b\})=a
$$
is a cluster marker map for $T'$ and $\sigma$ and 
$\cL_{(T,f)}$
(or more precisely the graph $\Gamma(\cL_{(T',f)})$) is depicted in
Fig.~\ref{fig:block-graph-motivation}~(i).

To help establish Theorem~\ref{theo: distinguished-lasso-verification}, we
require some intermediate results which are of interest in their own right and
which we present next. To this end, we denote for a vertex $v\in
\iV(T)-\{\rho_T\}$ by $T(v)$ the $L(v)$-tree with root $v$ obtained from $T$
by deleting the parent edge of $v$.

\begin{lem}
  \label{lem:insights}
  Suppose $T$ is a non-degenerate $X$-tree, $\sigma$ is a total ordering of
  $X$, and $f:cl(T)\to X$ is a cluster marker map for $T$ and $\sigma$. Then
  the following hold
  \begin{enumerate}
  \item[(i)] $\cL_{(T,f)}$ is a minimal topological lasso for $T$.
  \item[(ii)] $\Gamma(\cL_{(T,f)})$ is connected.
  \item[(iii)] If $v$ and $w$ are distinct interior vertices of $T$ then
    $|\bigcup \cL_{(T,f)}(v)\cap \bigcup \cL_{(T,f)}(w)|\leq 1$.
  \item[(iv)] Suppose $x\in X$. Then there exist distinct vertices $v,w\in
    \iV(T)$ such that $x\in \bigcup \cL_{(T,f)}(v)\cap \bigcup \cL_{(T,f)}(w)$
    if and only if there exists some $u\in \iV(T)-\{\rho_T\}$ such that
    $x=f(L(u))$.
  \end{enumerate}
\end{lem}
\begin{proof}
  For all $v\in \iV(T)$, set $\cL(v)=\cL_{(T,f)}(v)$.

  (i) This is an immediate consequence of
  Theorem~\ref{thm:child-edge-graph-complete} and the respective definitions
  of the set $\cL(v)$ where $v\in \iV(T)$ and the graph $G(\cL',v)$ where
  $\cL'$ is a set of cords of $X$ and $v$ is again an interior vertex of $T$.
 

  (ii) This is an immediate consequence of
  Proposition~\ref{prop:gamma-l-connected} and Lemma~\ref{lem:insights}~(i).


  (iii) This is an immediate consequence of the fact that, for all vertices
  $u\in \iV(T)$ and all $x,y\in \bigcup\cL(u)$ distinct, we have
  $u=lca_T(x,y)$.

  (iv) Let $x\in X$ and assume first that there exist distinct vertices
  $v,w\in \iV(T)$ such that $x\in \bigcup \cL(v)\cap \bigcup \cL(w)$ but
  $x\not =f(L_T(u))$, for all $u\in \iV(T)-\{\rho_T\}$. Then $x$ must be a
  leaf of $T$ that is simultaneously adjacent with $v$ and $w$ which is
  impossible. Thus, there must exist some $u\in \iV(T)$ such that $x=f(L(u))$.

  Conversely, assume that $x=f(L(u))$ for some $u\in \iV(T)-\{\rho_T\}$. Then
  $x\in L(u)$ and so there must exist an interior vertex $w$ of $T(u)$ that is
  adjacent with $x$.  Hence, $x\in\bigcup \cL(w)$.  Let $v$ denote the parent
  of $u$ in $T$ which exists since $u\not=\rho_T$.  Then $x=f(L(u))\in \bigcup
  \cL(v)$ and so $x\in \bigcup \cL(v)\cap \bigcup \cL(w)$, as required.
\end{proof}

Note that $u\in \{v,w\}$ need not hold for $u$, $v$ and $w$ as in the
statement of Lemma~\ref{lem:insights}~(iv).  Indeed, suppose $T$ is the
$X=\{a,b,c,d\}$-tree with unique cherry $\{a,b\}$ and $d$ adjacent with the
root $\rho_T$ of $T$. Let $\sigma$ denote the lexicographic ordering of $X$
and let $f:cl(T)\to X$ be (the unique) cluster marker map for $T$ and
$\sigma$.  Set $x=b$, $v=lca_T(a,b)$, $w=\rho_T$.  Then $x=f(L(u))$ where
$u=lca_T(a,c)$ and $x\in \bigcup\cL(v)\cap \bigcup\cL(w)$ but $u\not\in
\{v,w\}$.

\begin{pro}
  \label{prop:block}
  Suppose $T$ is a non-degenerate $X$-tree, $\sigma$ is a total ordering of
  $X$, and $f:cl(T)\to X$ is a cluster marker map for $T$ and $\sigma$. Then
  $\Gamma(\cL_{(T,f)})$ is a connected block graph and every block of
  $\Gamma(\cL_{(T,f)})$ is of the form $\Gamma(\cL_{(T,f)}(v))$, for some
  $v\in \iV(T)$.
\end{pro}
\begin{proof}
  For all $v\in \iV(T)$, set $\cL(v)=\cL_{(T,f)}(v)$ and put
  $\cL=\cL_{(T,f)}$.  We claim that if $C$ is a cycle in $\Gamma(\cL)$ of
  length at least three then there must exist some $v\in \iV(T)$ such that $C$
  is contained in $\Gamma(\cL(v))$. Assume to the contrary that this is not
  the case, that is, there exists some cycle $C:u_1,u_2,\ldots,
  u_l,u_{l+1}=u_1$, $l\geq 3$, in $\Gamma(\cL)$ such that, for all $v\in
  \iV(T)$, we have that $C$ is not a cycle in $\Gamma(\cL(v))$. Without loss
  of generality, we may assume that $C$ is of minimal length. For all
  $i\in\langle l\rangle$, put $v_i=lca_T(u_i,u_{i+1})$. Then, by the
  construction of $\Gamma(\cL)$, we have for all such $i$ that $u_iu_{i+1}$ is
  an edge in $\Gamma(\cL(v_i))$ and, by the minimality of $C$, that
  $v_i\not=v_j$ for all $i,j\in\langle l\rangle$ distinct. Put $Y=V(C)$ and
  let $T'=T|_Y$ denote the $Y$-tree obtained by restricting $T$ to $Y$. Note
  that $lca_T(u_i,u_{i+1})=lca_{T'}(u_i,u_{i+1})$ holds for all $i\in\langle
  l\rangle$.  Thus, the map $\phi:E(C)\to \iV(T')$ defined by putting
  $u_iu_{i+1}\mapsto lca_{T}(u_i,u_{i+1})$, $i\in\langle l \rangle$, is
  well-defined.  Since $|E(C)|=l$ and for any finite set $Z$ with three or
  more elements a $Z$-tree has at most $|Z|-1$ interior vertices, it follows
  that there exist $i,j\in \langle l \rangle$ distinct such that
  $\phi(u_i,u_{i+1})=\phi(u_j,u_{j+1})$. Consequently,
  $v_i=lca_T(u_i,u_{i+1})=lca_T(u_j,u_{j+1})=v_j$ which is impossible and thus
  proves the claim.  Combined with Lemma~\ref{lem:insights}~(ii) and (iii), it
  follows that $\Gamma(\cL)$ is a connected block graph. That the blocks of
  $\Gamma(\cL)$ are of the required form is an immediate consequence of the
  construction of $\Gamma(\cL)$.
\end{proof}


To be able to establish that $\cL_{(T,f)}(v)$ is indeed a distinguished
minimal topological lasso for $T$ and $f$ as above, we require a further
concept. Suppose $A, B\subseteq X$ are two distinct non-empty subsets of
$X$. Then $A$ and $B$ are said to be {\em compatible} if $A\cap
B\in\{\emptyset, A,B\}$. As is well-known (see
e.\,g.\,\cite{DHKMS11,semple2003phylogenetics}), for any $X$-tree $T'$ and any
two vertices $v,w\in V(T')$ the subsets $L(v)$ and $L(w)$ of $X$ are
compatible.

\begin{thm}
  \label{theo: distinguished-lasso-verification}
  Suppose $T$ is a non-degenerate $X$-tree, $\sigma$ is a total ordering of
  $X$ and $f:cl(T)\to X$ is a cluster marker map for $T$ and $\sigma$. Then
  $\cL_{(T,f)}$ is a distinguished minimal topological lasso for $T$.
\end{thm}
\begin{proof}
  For all $v\in \iV(T)$ put $\cL(v) = \cL_{(T,f)}(v)$ and put
  $\cL=\cL_{(T,f)}$.  In view of Proposition~\ref{prop:block} and
  Lemma~\ref{lem:insights}~(i), it suffices to show that $\Gamma(\cL)$ is
  claw-free.  Assume to the contrary that this is not the case and that there
  exists some $x\in X$ that is contained in the vertex set of $m\geq 3$ blocks
  $A_1,\ldots,A_m$ of $\Gamma(\cL)$. Then, by Proposition~\ref{prop:block},
  there exist distinct interior vertices $v_1, \ldots, v_m$ of $T$ such that,
  for all $i\in\langle m\rangle$, we have $V(A_i)=\bigcup\cL(v_i)\subseteq
  L(v_i)$.  Since for all $v,w\in V(T)$ distinct, the sets $L(v)$ and $L(w)$
  are compatible, it follows that there exists a path $P$ from $\rho_T$ to $x$
  that contains the vertices $v_1,\ldots, v_m$ in its vertex set. Without loss
  of generality we may assume that $m=3$ and that, starting at $\rho_T$ and
  moving along $P$ the vertex $v_1$ is encountered first then $v_2$ and then
  $v_3$. Note that $cl_{L(v_i)}(T)\not=\emptyset$, for $i=1,2$.  Since $T$ is
  a tree and so $x$ can neither be adjacent with $v_1$ nor with $v_2$ it
  follows that there must exist for $i=1,2$ some $B_i\in cl_{L(v_i)}(T)$ such
  that $x=f(B_i)$. But this is impossible as $B_2\in cl_{L(v_1)}(T)$ and so
  $f(B_1)\not=f(B_2)$ as $f$ is a cluster marker map for $T$ and $\sigma$.
\end{proof}

\section{Characterisation of distinguished minimal topological lassos}
\label{sec:characterization-distinguished}

In this section, we establish the converse of Theorem~\ref{theo:
  distinguished-lasso-verification} which allows us to characterise
distinguished minimal topological lasso of non-degenerate $X$-trees.  We start
with a well-known construction for associating an unrooted tree to a connected
block graph (see e.\,g.\,\cite{diestel}).  Suppose that $G$ is a connected
block graph. Then we denote by $T_G$ the (unrooted) tree associated to $G$
with vertex set $Cut(G)\cup Block(G)$ and whose edges are of the from
$\{a,B\}$ where $a\in Cut(G)$, $B\in Block(G)$ and $a\in B$. Note that if a
vertex $v\in V(T_G)$ is a leaf of $T_G$ then $v\in Block(G)$.

Suppose $T$ is a non-degenerate $X$-tree and $\cL$ is a distinguished minimal
topological lasso for $T$.  Let $v$ denote an interior vertex of $T$ whose
children are $v_1\ldots,v_l$ where $l=|ch(v)|$. Then
Corollary~\ref{cor:bijection} combined with Proposition~\ref{prop:x-i-unique}
implies that for all $i\in\langle l\rangle$ there exists a unique leaf $x_i\in
L(v_i)$ of $T$ such that, for all $i,j\in\langle l\rangle$ distinct,
$x_ix_j\in \cL$ and $\{x_1,\ldots, x_l\}=V(B_v)$. Since $\Gamma(\cL)$ is
claw-free, every vertex of $B_v$ is contained in at most one further block of
$\Gamma(\cL)$. Thus, if $w\in V(B_v)$ and $w\in V(B) $ holds too for some
block $B\in Block(\Gamma(\cL))$ distinct from $B_v$ then $w$ must be a cut
vertex of $\Gamma(\cL)$. For every vertex $v'\in \iV(T)$ that is the child of
some vertex $v\in \iV(T)$, we denote the unique element $x\in L(v')$ contained
in $V(B_v)$ by $c_{B_{v'}}$ in case $x\in Cut(\Gamma(\cL))$.  Note that it is
not difficult to observe that, in the tree $T_{\Gamma(\cL)}$, the vertex
$c_{B_{v'}}$ is the vertex adjacent with $B_v$ that lies on the path from
$B_v$ to $B_{v'}$.

The following result lies at the heart of Theorem~\ref{theo:characterization}
and establishes a crucial relationship between the non-root interior vertices
of $T$ and the cut vertices of $\Gamma(\cL)$.

\begin{lem}
  \label{lem:bijection-theta}
  Suppose $T$ is an $X$-tree and $\cL$ is a distinguished minimal topological
  lasso for $T$. Then the map
$$
\theta :\iV(T)-\{\rho_T\} \to Cut(\Gamma(\cL)):\,\,\,v\mapsto c_{B_{v}}
$$ 
is bijective.
\end{lem}
\begin{proof}
  Clearly, $\theta$ is well-defined and injective. To see that $\theta$ is
  bijective let $ T_{\Gamma(\cL)}^-$ denote the tree obtained from $
  T_{\Gamma(\cL)}$ by suppressing all degree two vertices. Then
  $Block(\Gamma(\cL))=V(T_{\Gamma(\cL)}^-)$ and Corollary~\ref{cor:bijection}
  implies that $|Block(\Gamma(\cL))|=|\iV(T)|$ as $\Gamma(\cL)$ is a block
  graph. Since $\Gamma(\cL)$ is claw-free, we clearly also have
  $|Cut(\Gamma(\cL))|=|E(T_{\Gamma(\cL)}^-)|$. Combined with the fact that f
  $|V(T')|= |E(T')|+1$ holds for every tree $T'$, it follows that
  $|Cut(\Gamma(\cL))|=|Block(\Gamma(\cL))|-1=|\iV(T)|-1=
  |\iV(T)-\{\rho_T\}|$. Thus, $\theta$ is bijective.  \qquad
\end{proof}

Armed with this result, we are now ready to establish the converse of
Theorem~\ref{theo: distinguished-lasso-verification} which yields the
aforementioned characterisation of distinguished minimal topological lassos of
non-degenerate $X$-trees.

\begin{thm}
  \label{theo:characterization}
  Suppose $T$ is a non-degenerate $X$-tree and $\cL$ is a set of cords of
  $X$. Then $\cL$ is a distinguished minimal topological lasso for $T$ if and
  only if there exists a total ordering $\sigma$ of $X$ and a cluster marker
  map $f$ for $T$ and $\sigma$ such that $\cL_{(T,f)}=\cL$.
\end{thm}
\begin{proof}
  Assume first that $\sigma$ is some total ordering of $X$ and that
  $f:cl(T)\to X$ is a cluster marker map for $T$ and $\sigma$. Then, by
  Theorem~\ref{theo: distinguished-lasso-verification}, $\cL_{(T,f)}$ is a
  distinguished minimal topological lasso for $T$.

  Conversely assume that $\cL$ is a distinguished minimal topological lasso
  for $T$ and consider an embedding of $T$ into the plane.  By abuse of
  terminology, we will refer to this embedding of $T$ also as $T$.  We start
  with defining a total ordering $\sigma$ of $X$.  To this end, we first
  define a map $t:\iV(T)-\{\rho_T\}\to \mathbb N$ by setting, for all $v\in
  \iV(T)-\{\rho_T\}$, $t(v)$ to be the length of the path from $\rho_T$ and
  $v$. Put $h=\max\{t(v)\,:\, v\in \iV(T)-\{\rho_T\}\}$ and note that $h\geq
  1$ as $T$ is non-degenerate.

  Starting at the left most interior vertex $v$ of $T$ for which $t(v)=h$
  holds and moving, for all $l \in \langle h\rangle$, from left to right, we
  enumerate all interior vertices of $T$ but the root. We next put $n=|X|$ and
  $X=\langle n\rangle$ and relabel the elements in $X$ such that when
  traversing the circular ordering induced by $T$ on $X\cup\{\rho_T\}$ in a
  counter-clockwise fashion we have $\rho_T,1,2,3,\ldots, n,\rho_T$. To
  reflect this with regards to $\cL$, we relabel the elements of the cords in
  $\cL$ accordingly and denote the resulting distinguished minimal topological
  lasso for $T$ also by $\cL$.

  By Lemma~\ref{lem:bijection-theta}, the map $\theta :\iV(T)-\{\rho_T\} \to
  Cut(\Gamma(\cL))$ defined in that lemma is bijective. Put
  $m=|Cut(\Gamma(\cL))|$ and let $v_1,v_2,\ldots, v_m$ denote the enumeration
  of the vertices in $\iV(T)-\{\rho_T\}$ obtained above. Also, set
  $Y=X-\{\theta(v_i): i\in \langle m\rangle\}$. Let $y_1,y_2,\ldots, y_l$
  denote an arbitrary but fixed total ordering of the elements of $Y$ where
  $l=|Y|$. Then we define $\sigma$ to be the total ordering of $X$ given by
  $$
  \sigma:\,\, \theta(v_1),\theta(v_2),\ldots, \theta(v_{i-1}),
  \theta(v_i),\theta(v_{i+1}), ,\ldots, \theta(v_m),y_1,y_2,\ldots, y_l
  $$ 
  where $\theta(v_1)$ is the minimal element and $y_l$ is the maximal element.
  Note that if $v\in \iV(T)$ is the parent of a pseudo-cherry $C$ of $T$ then
  $\theta(v)=\min_{\sigma}C$.


  We briefly interrupt the proof of the theorem to illustrate these
  definitions by means of an example. Put $X=\langle 13\rangle$ and consider
  the $X$-tree $T$ depicted in Fig.~\ref{fig:illustration-main-theorem}~(i)
  (ignoring the labelling of the interior vertices for the moment) and the
  distinguished minimal topological lasso $\cL$ for $T$ pictured in the form
  of $\Gamma(\cL)$ in Fig.~\ref{fig:illustration-main-theorem}~(ii). Then the
  labelling of the interior vertices of $T$ gives the enumeration of those
  vertices considered in the proof of Theorem~\ref{theo:characterization}. The
  total ordering $\sigma$ of $X$ restricted to the elements in
  $\{\theta(v_1),\ldots, \theta(v_6)\}$ is $3,5,12,1,10,7$.

  \begin{figure}
    \begin{center}
      \input{figures/dist-min-lass/main-theorem.pdft}
    \end{center}
    \caption{ For $X=\langle 13\rangle$ and the depicted $X$-tree $T$, the
      enumeration of the interior vertices of $T$ considered in the proof of
      Theorem~\ref{theo:characterization} is indicated in (i). With regards to
      this enumeration and the distinguished minimal topological lasso $\cL$ for
      $T$ pictured in the form of $\Gamma(\cL)$ in (ii), the total ordering
      $\sigma$ of $X$ considered in that proof restricted to the elements in
      $\{\theta(v_1),\ldots, \theta(v_6)\}$ is $3,5,12,1,10,7$.}
    \label{fig:illustration-main-theorem}
  \end{figure}

  Returning to the proof of the theorem, we claim that the map $f:cl(T)\to X$
  given, for all $A\in cl(T)$, by setting $f(A)=\theta(lca(A))$ is a cluster
  marker map for $T$ and $\sigma$ where for all such $A$ we put
  $lca(A)=lca_T(A)$. Indeed, suppose $A\in cl(T)$.  Then
  $\theta(lca(A))=c_{B_{lca(A)}}\in L(lca(A))$ holds by construction.  We
  distinguish between the cases that $cl_A(T)\not =\emptyset$ and that
  $cl_A(T)=\emptyset$. If $cl_A(T)\not =\emptyset$ then since $\theta$ is
  bijective it follows that $\theta(lca(A))\not=\theta(v)$ holds for all
  descendants $v\in \iV(T)$ of $lca(A)$.  Combined with the definition of
  $\sigma$, we obtain
  $f(A)=\theta(lca(A))=\min_{\sigma}(A-\{\theta(lca(D))\,:\, D\in cl_A(T)\})=
  \min_{\sigma}(A-\{f(D)\,:\, D\in cl_A(T)\})$, as required.  If
  $cl_A(T)=\emptyset$ then, as was observed above,
  $f(A)=\theta(lca(A))=\min_{\sigma}A$. Thus, $f$ is a cluster marker map for
  $T$ and $\sigma$, as claimed.

  It remains to show that $\cL_{(T,f)}=\cL$. To see this note first that, by
  Theorem~\ref{theo: distinguished-lasso-verification}, $\cL_{(T,f)}$ is a
  distinguished minimal topological lasso for $T$. Since
  Lemma~\ref{lem:size-A(v)} implies that any two minimal topological lasso for
  $T$ must be of the same size and thus $|\cL_{(T,f)}|=|\cL|$ holds, it
  therefore suffices to show that $\cL\subseteq \cL_{(T,f)}$. Suppose $a,b\in
  X$ distinct such that $ab\in \cL$.  Then there exists some interior vertex
  $v\in \iV(T)$ such that $v=lca_T(a,b)$.  Hence, $a,b\in V(B_v)$. We claim
  that $ab\in \cL_{(T,f)}(v)$.  To establish this claim, we distinguish
  between the cases that (i) $a\in ch(v)$ and (ii) that $a\not\in ch(v)$.

  Assume first that Case (i) holds, that is, $a$ is a child of $v$.  If $b\in
  ch(v)$ then the claim is an immediate consequence of the definition of
  $\cL_{(T,f)}(v)$. So assume that $b\not\in ch(v)$. Let $v'\in \iV(T)$ denote
  the child of $v$ for which $b\in L(v)$ holds. Then
  $b=c_{B_{v'}}=\theta(v')=f(L(v'))$ follows by the observation preceding
  Lemma~\ref{lem:bijection-theta} combined with the fact that $b\in
  V(B_v)$. Hence, $ab=af(L(v'))\in \cL_{(T,f)}(v)$, as claimed.

  Assume next that Case (ii) holds, that is, $a$ is not a child of $v$.  In
  view of the previous subcase it suffices to consider the case that $b\not\in
  ch(v)$. Let $v',v''\in \iV(T)$ denote the children of $v$ such that $a\in
  L(v')$ and $b\in L(v'')$.  Then, again by the observation preceding
  Lemma~\ref{lem:bijection-theta} combined with the fact that $a,b\in V(B_v)$,
  we have $a=c_{B_{v'}}=\theta(v')=f(L(v'))$ and
  $b=c_{B_{v''}}=\theta(v'')=f(L(v''))$ and so $ab=f(L(v'))f(L(v''))\in
  \cL_{(T,f)}(v)$ follows, as claimed.  This concludes the proof of the claim
  and thus the proof of the theorem.
\end{proof}

Recall that Theorem~\ref{thm:child-edge-graph-complete} tells us that a set
$\cL$ of cords of $X$ is an equidistant lasso for an $X$-tree $T$ if and only
if, for every vertex $v\in \iV(T)$, the graph $G(\cL,v)$ has at least one
edge.  Since for $\sigma$ some total ordering of $X$ and
$f:\iV(T)-\{\rho_T\}\to X$ a cluster marker map for $T$ and $\sigma$ the
graphs $G(\cL_{(T,f)},v)$ clearly satisfy this property for all $v\in \iV(T)$,
it follows that $\cL_{(T,f)}$ is also an equidistant lasso for $T$ and thus a
strong lasso for $T$.

Defining a strong lasso $\cL$ of an $X$-tree to be {\em minimal} in analogy to
when a topological lasso is minimal, Theorem~\ref{theo:characterization}
implies

\begin{cor}
  \label{corollary:strong-lasso-characterization}
  Suppose $T$ is a non-degenerate $X$-tree, $\cL$ is a set of cords of $X$,
  $\sigma$ is a total ordering of $X$, and $f:cl(T)\to X$ is a cluster marker
  map for $T$ and $\sigma$.  Then $\cL_{(T,f)}$ is a minimal strong lasso for
  $T$.
\end{cor}

\section{Heredity of distinguished minimal topological lassos}
\label{sec:subtree}

In this section, we turn our attention to the problems of characterising when
a distinguished minimal topological lasso of an $X$-tree $T$ induces a
distinguished minimal topological lasso for a subtree of $T$ and, conversely
when distinguished minimal topological lassos of $X$-trees can be combined to
form a distinguished minimal topological lasso of a supertree for those trees
(see e.\,g.\,\cite{bininda04phylogenetic} for more on such trees). This will
also allow us to partially answer the rooted analogue of a question raised in
\cite{dress11lassoing} for supertrees within the unrooted framework.  To make
this more precise, we require further terminology.  Suppose $\cL$ a set of
cords of $X$ and $Y\subseteq X$ is a non-empty subset. Then we set
$$
\cL|_Y=\{ab\in\cL\,:\, a,b\in Y\}.
$$
Clearly, $\Gamma(\cL|_Y)$ is the subgraph of $\Gamma(\cL)$ induced by $Y$ but
$Y=\bigcup \cL|_Y$ need not hold. Moreover, if $\cL$ is a minimal topological
lasso for an $X$-tree $T$ and $|Y|\geq 3$ such that every interior vertex of
$T$ is also an interior vertex of $T|_Y$ then
Theorem~\ref{thm:child-edge-graph-complete} implies that $\cL|_Y$ is a
minimal topological lasso for $T|_Y$. In particular, $\Gamma(\cL|_Y)$ must be
connected in this case.  The next result is a strengthening of this
observation.

\begin{thm}
  \label{theo:subtree}
  Suppose $T$ is an $X$-tree, $\cL$ is a distinguished minimal topological
  lasso for $T$, and $Y\subseteq X$ is a subset of size at least three. Then
  $\cL|_Y$ is a distinguished minimal topological lasso for $T|_Y$ if and only
  if $\Gamma(\cL|_Y)$ is connected.
\end{thm}
\begin{proof}
  Assume first that $\cL|_Y$ is a distinguished minimal topological lasso for
  $T|_Y$. Then, by Proposition~\ref{prop:gamma-l-connected}, $\Gamma(\cL|_Y)$
  is connected.

  Conversely, assume that $\Gamma(\cL|_Y)$ is connected. Then the statement
  clearly holds if $T$ is the star tree on $X$. So assume that $T$ is
  non-degenerate. Let $Y\subseteq X$ be of size at least three and assume
  first that $T|_Y$ is the star tree on $Y$. We claim that $\Gamma(\cL|_Y)$ is
  a clique. Assume to the contrary that this is not the case, that is, there
  exist elements $y,y'\in Y$ distinct such that $yy'\not\in \cL$. Since
  $\Gamma(\cL|_Y)$ is connected, there must exist a path
  $P:x_1=y,x_2,\ldots,x_l=y'$, $l\geq 2$, in $\Gamma(\cL|_Y)$ from $y$ to
  $y'$. Since the vertex set of $\Gamma(\cL|_Y)$ is $Y$, it follows that
  $X'=\{x_1,x_2,\ldots,x_l\}\subseteq Y$. Combined with the fact that
  $lca_T(x,x')=lca_T(Y)$ holds for all $x,x'\in X'$ distinct as $T|_Y$ is a
  star tree on $Y$, we obtain $X'\subseteq V(B_{lca_T(Y)})$.  Thus, $yy'\in
  \cL$ which is impossible and thus proves the claim.  That $\cL|_Y$ is a
  distinguished minimal topological lasso for $T|_Y$ is a trivial consequence.

  So assume that $T|_Y$ is non-degenerate. Since $\cL$ is a distinguished
  minimal topological lasso for $T$, Theorem~\ref{theo:characterization}
  implies that there exists a total ordering $\omega$ of $X$ and a cluster
  marker map $f_{\omega}: cl(T)\to X$ for $T$ and $\omega$ such that
  $\cL=\cL_{(T,f_{\omega})}$. Moreover, Lemma~\ref{lem:insights}~(iv) implies
  that the cut-vertices of $\Gamma(\cL)$ are of the form $f_{\omega}(L_T(v))$
  where $v\in \iV(T)$.

  To see that $\cL|_Y$ is a distinguished minimal topological lasso for $T|_Y$
  and some total ordering of $Y$ note first that the restriction $\sigma$ of
  $\omega$ to $Y$ induces a total ordering of $Y$.  Furthermore, the
  aforementioned form of the cut-vertices of $\Gamma(\cL)$ combined with the
  assumption that $\Gamma(\cL|_Y)$ is connected implies that, for all $A\in
  cl(T)$ with $A\cap Y\not=\emptyset$, we must have $f_{\omega}(A)\in Y$. For
  all $A\in cl(T|_Y)$ denote by $A^T$ the set-inclusion minimal superset of
  $A$ contained in $cl(T)$.  Then since $f_{\omega}$ is a cluster marker map
  for $T$ and $\omega$ it follows that the map
  $$
  f_{\sigma}:cl(T|_Y)\to Y\,:\, A\mapsto f_{\omega}(A^T)
  $$
  is a cluster marker map for $T|_Y$ and $\sigma$.  By
  Theorem~\ref{theo:characterization} it now suffices to establish that
  $\cL|_Y=\cL_{(T|_Y,f_{\sigma})}$. Since both $\cL|_Y$ and
  $\cL_{(T|_Y,f_{\sigma})}$ are minimal topological lassos for $T|_Y$ and so
  $|\cL|_Y|=|\cL_{(T|_Y,f_{\sigma})}|$ is implied by Lemma~\ref{lem:size-A(v)}
  it suffices to show that $\cL|_Y\subseteq \cL_{(T|_Y,f_{\sigma})}$.

  Suppose $ab\in \cL|_Y$, that is, $ab\in\cL$ and $a,b\in Y$. Since $Y$ is the
  leaf set of $T|_Y$, there must exist a vertex $v\in \iV(T|_Y)$ such that
  $v=lca_{T|_Y}(a,b)$. Clearly, $v\in \iV(T)$. If $a$ and $b$ are both
  adjacent with $v$ in $T$ then $a$ and $b$ are also adjacent with $v$ in
  $T|_Y$. Thus $ab\in\cL_{(T|_Y,f_{\sigma})}(v)$ in this case.  So assume that
  at least one of $a$ and $b$ is not adjacent with $v$ in $T$.  Without loss
  of generality let $a$ denote that vertex.  Then since $ab\in \cL=
  \cL_{(T,f_{\omega})}$, it follows that there must exist a unique child
  $v'\in \iV(T)$ of $v$ such that $a\in L_T(v')$ and
  $a=f_{\omega}(L_T(v'))$. Hence, $a\in V(B_v)$ and a cut-vertex of
  $\Gamma(\cL)$.

  We claim that $v'\in \iV(T|_Y)$. Assume for contradiction that $v'\not\in
  \iV(T|_Y)$. Then since $f_{\omega}$ is a cluster marker map for $T$ and
  $\omega$, it follows that $a$ cannot be a cut vertex in $\Gamma(\cL|_Y)$.
  Since $\Gamma(\cL)$ is a claw-free block graph, no edge in the unique block
  $B'\in Block(\Gamma(\cL))-\{B_v\}$ that also contains $a$ in its vertex set
  can therefore be incident with $a$ in $\Gamma(\cL|_Y)$. Since
  $\Gamma(\cL|_Y)$ is assumed to be connected, to obtain the required
  contradiction it now suffices to show that there exists some $c\in Y\cap
  L_T(v')$ distinct from $a$ such that every path from $c$ to $b$ in
  $\Gamma(\cL)$ crosses $a$.  But this is a consequence of the facts that $v$
  is not the parent of $a$ in $T|_Y$ and, implied by
  Proposition~\ref{prop:gamma-l-connected}, that the subgraph
  $\Gamma_{v'}(\cL)$ of $\Gamma(\cL)$ induced by $L_T(v')$ is the connected
  component of $\Gamma(\cL)$ containing $a$ obtained from $\Gamma(\cL)$ by
  deleting all edges in $B_v$ that are incident with $a$. This concludes the
  proof of the claim

  To conclude the proof of the theorem, note that if $b$ is adjacent with $v$
  in $T|_Y$ then $ab= f_{\omega}(L_T(v'))b=f_{\omega}((L_{T|_Y}(v'))^T)b=
  f_{\sigma}(L_{T|_Y}(v'))b\in\cL_{(T|_Y,f_{\sigma})}(v) \subseteq
  \cL_{(T|_Y,f_{\sigma})}$.  If $b$ is not adjacent with $v$ in $T|_Y$ then
  there exists a child $v''\in \iV(T)$ of $v$ such that
  $b=f_{\omega}(L_T(v''))$.  In view of the previous claim, we have $v''\in
  \iV(T|_Y)$.  But now arguments similar to the ones used before imply that
  $ab\in \cL_{(T|_Y,f_{\sigma})}(v)\subseteq \cL_{(T|_Y,f_{\sigma})}$.
\end{proof}

We now turn our attention to supertrees which are formally defined as
follows. Suppose $\mathcal T=\{T_1,\ldots, T_l\}$, $l\geq 1$, is a set of
$Y_i$-trees $T_i$ with $Y_i\subseteq X$ and $|Y_i|\geq 3$, $i\in\langle
l\rangle$, and $T$ is an $X$-tree. Then $T$ is a called a {\em supertree} of
$\mathcal T $ if $T$ displays every tree in $\mathcal T$ where we say that
some $X$-tree $T$ {\em displays} some $Y$-tree $T'$ for $Y\subseteq X$ with
$|Y|\geq 3$ if $T|_Y$ and $T'$ are equivalent. More precisely, we have the
following result which relies on the fact that in case $\cL$ is a
distinguished minimal topological lasso for a {\em binary} $X$-tree $T$, that
is, every vertex of $T$ but the leaves has two children, $\Gamma(\cL)$ must be
a path. In particular, $\cL$ induces a total ordering of the elements in $X$
in this case.  For $Y\subseteq X$ a non-empty subset of $X$, we denote the
maximal and minimal element in $Y$ with regards to that ordering by
$\min_{\cL}(Y)$ and $\max_{\cL}(Y)$, respectively.

\begin{cor}
  \label{cor:supertree}
  Suppose $X'$ and $X''$ are two non-empty subsets of $X$ such that $X=X'\cup
  X''$ and $X'\cap X''\not=\emptyset$ and $T'$ and $T''$ are $X'$-trees and
  $X''$-tree, respectively. Suppose also that $\cL'$ and $\cL''$ are
  distinguished minimal topological lassos for $T'$ and $T''$, respectively,
  such that $\cL'|_{X'\cap X''} =\cL''|_{X'\cap X''}$ and
  $\Gamma(\cL''|_{X'\cap X''})$ is connected.  If $T$ is a binary $X$-tree
  that displays both $T'$ and $T''$ then $\cL=\cL'\cup\cL''$ is a
  distinguished minimal topological lasso for $T$ if and only if
  $\min_{\cL'}(X'\cap X'')\in \{\min_{\cL'}(X'), \min_{\cL''}(X'')\}$ and
  $\max_{\cL'}(X'\cap X'')\in \{\max_{\cL'}(X'), \max_{\cL''}(X'')\}$.
\end{cor}

Continuing with the assumptions of Corollary~\ref{cor:supertree}, we also have
that if
\begin{equation*}
  \min_{\cL'}(X'\cap X'')\in \{\min_{\cL'}(X'), \min_{\cL''}(X'')\}
\end{equation*}
and
\begin{equation*}
  \max_{\cL'}(X'\cap X'')\in \{\max_{\cL'}(X'), \max_{\cL''}(X'')\}
\end{equation*}
holds then $\cL'\cup \cL''$ is a (minimal) strong lasso for $T$ as every
minimal topological lasso for an $X$-tree is also an equidistant lasso for
that tree. However, not all strong lassos for $T$ are of this form. An example
for this is furnished for $X'=\{a,c,d\}$ and $X''=\{a,b,c\}$ by the $X'$-tree
$T'$, the $X''$-tree $T''$ and the $X'\cup X''$-tree $T$ depicted in
Fig.~\ref{fig:supertree} along with the set $\cL'=\{cd\}$ and
$\cL''=\{ab,bc\}$ of cords of $X'$ and $X''$, respectively. Clearly, $T$ is a
supertree of $\{T',T''\}$ and $\cL=\cL'\cup\cL''$ is a strong lasso for $T$
but $\cL'$ is not even an equidistant lasso for $T'$. Investigating further
the interplay between minimal topological lassos for $X$-trees and minimal
topological lassos for supertrees that display them might therefore be of
interest.

\begin{figure}
  \begin{center}
    \input{figures/dist-min-lass/supertree.pdft}
  \end{center}
  \caption{ For $X'=\{a,c,d\}$ and $X''=\{a,b,c\}$ the $X'\cup X''$-tree $T$
    is a supertree for the depicted $X'$ and $X''$ trees $T'$ and $T''$,
    respectively. Clearly, $\cL'=\{cd\}$ and $\cL''=\{ab,bc\}$ are sets of
    cords of $X'$ and $X''$, respectively, and $\cL=\cL'\cup\cL''$ is a strong
    lasso for $T$ but $\cL'$ is not even an equidistant lasso for $T'$.  }
  \label{fig:supertree}
\end{figure}

We conclude with returning to Fig.~\ref{fig:block-graph-motivation} which
depicts two non-equivalent $X$-trees that are topologically lassoed by the
same set $\cL$ of cords of $X$. In fact, $\cL$ is even a minimal topological
lasso for both of them.  Understanding better the relationship between
$X$-trees that are topologically lassoed by the same set of cords of $X$ might
also be of interest to study further.

\section{Conclusion}
\label{sec:conclusion-dist}

In this chapter we introduced a special type of lasso called a distinguished
minimal topological lasso for which the associated $\Gamma(\cL)$ graph is a
claw-free block graph.  We have shown that for such a lasso $\cL$ each block
in $\Gamma(\cL)$ corresponds to one interior vertex in the tree
(Theorem~\ref{theo:unique-block} and Corollary~\ref{cor:bijection}).  A
distinguished minimal topological lasso is special in that for any $X$-tree a
minimal topological lasso for it can be transformed into a distinguished
minimal topological lasso by repeated application of a simple rule.  We
characterised these lassos in terms of a cluster marker map and have given a
criterion for inheritance by subtrees and supertrees.

%%% Local Variables:
%%% TeX-master: "thesis"
%%% End:
