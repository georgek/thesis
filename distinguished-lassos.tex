\chapter{Distinguished Minimal Topological Lassos}
\label{cha:dist-minim-topl}

This chapter is based on the following paper:
\vspace{0.5em}

\noindent

\begin{itemize}
\item Katharina T. Huber and George Kettleborough. Distinguished minimal
  topological lassos.  Submitted 2013.
\end{itemize}

\newpage

A classical result in distance based tree-reconstruction characterises when
for a distance $D$ on some finite set $X$ there exist a uniquely determined
dendrogram on $X$ (essentially a rooted tree $T=(V,E)$ with leaf set $X$ and
no degree two vertices but possibly the root and an edge weighting
$\omega:E\to \mathbb R_{\geq 0}$) such that the distance $D_{(T,\omega)}$
induced by $(T,\omega)$ on $X$ is $D$.  Moreover, algorithms that quickly
reconstruct $(T,\omega)$ from $D$ in this case are known.  However in many
areas where dendrograms are being constructed such as Computational Biology
not all distances on $X$ are always available implying that the sought after
dendrogram need not be uniquely determined anymore by the available distances
with regards to topology of the underlying tree, edge-weighting, or both.  To
better understand the structural properties a set $\cL\subseteq {X\choose 2}$
has to satisfy to overcome this problem,
%that the distances on the leaf pairs in $\cL$ 
%do in fact uniquely determine the dendrogram 
%in question, 
various types of lassos have been introduced.  Here, we focus on the question
of when a lasso uniquely determines the topology of a dendrogram's underlying
tree, that is, it is a topological lasso for that tree.  We show that any
set-inclusion minimal topological lasso for such a tree $T$ can be transformed
into a 'distinguished' minimal topological lasso $\cL$ for $T$, that is, the
graph $(X,\cL)$ is a claw-free block graph. Furthermore, we characterise such
lassos in terms of the novel concept of a cluster marker map for $T$ and
present results concerning the heritability of such lassos in the context of
the subtree and supertree problems.

\section{Introduction}

In many topical studies in Computational Biology ranging from gene onthology
via genome wide association studies in population genetics to evolutionary
genomics, the following fundamental mathematical problem is encountered: Given
a distance $D$ on some set $X$ of objects, find a dendrogram $\mathcal D$ on
$X$ (essentially a rooted tree $T=(V,E)$ with no degree two vertices but
possibly the root whose leaf set is $X$ together with an edge-weighting
$\omega:E\to\mathbb R_{\geq 0}$ -- see Fig.~\ref{fig:block-graph-motivation}
for examples) such that the distance induced by $\mathcal D$ on any two of its
leaves $x$ and $y$ equals $D(x,y)$. In the ideal case that the distances
between any two elements of $X$ are available, it is well-understood when such
a tree is uniquely determined by them and fast algorithms for reconstructing
it from them are known (see e.\,g.\,\cite[Chapter 9.2]{DHKMS11} and
\cite[Chapter 7.2]{semple2003phylogenetics} where dendrograms are considered
in the slightly more general forms of dated rooted $X$-trees and equidistant
representations of dissimilarities, respectively, and \cite[Chapter 3]{BG91}
as well as the references in all three of them for more on this).
 
The reality however tends to be different in many cases in that distances
between pairs of objects might be missing or are not sufficiently reliable to
warrant inclusion of that distance in an analysis -- see
e.g. \cite{P04,SMS10,SS10} for more on this topic in an evolutionary genomics
context).  Exclusion of such a distance might therefore be tempting but is
clearly not always desirable which raises interesting mathematical,
statistical, and algorithmical questions (see
e.\,g.\,\cite{de1984ultrametric,farach1995robust,schader1992mvl} for a study
concerning the latter and
\cite{farach1995robust,guenoche1999approximations,guenoche2004extension,makarenkov2001nouvelle}
for results concerning its unrooted variant).  One of them is the focus of
this paper: Calling any subset of a finite set $X$ of size two a {\em cord} of
$X$ then for what sets $\cL$ of cords of $X$ do we need to know the distances
so that both the topology of the underlying tree and the edge-weights of the
dendrogram on $X$ that induced the distances on the cords in $\cL$ is uniquely
determined by $\cL$?

To help illustrate the intricacies of this question which is concerned with
the structure of the set $\cL$ and not so much with the actual distances on
the cords in $\cL$, denote for any two distinct elements $a,b\in X$ the cord
$\{a,b\}$ by $ab$. Consider the dendrogram $\mathcal D$ with leaf set
$X=\{a,\ldots, e\}$ depicted in Fig.~\ref{fig:lasso-example}(i) and assume
that the distances on the cords of $\cL=\{ac,de,bc,ce,cd\}$ are induced by
$\mathcal D$ so, for example, the distance on the cord $ab$
%is the distance between $a$ and $b$ which 
is four. Then the dendrogram $\mathcal D'$ depicted in
Fig.~\ref{fig:lasso-example}(ii) induces the same distances on the cords in
$\cL$ as $\mathcal D$ but the topologies of the underlying trees $T$ and $T'$
of $\mathcal D$ and $\mathcal D'$, respectively, are clearly not the same in
the sense that there exists no bijection from $V(T)$ to $V(T')$ that is the
identity on $\{a,\ldots, e\}$ and induces a rooted graph isomorphism from $T$
to $T'$.  Thus, $\cL$ does not uniquely determine $T$ and thus also not
$\mathcal D$. However as can be quickly checked the situation changes if and
only if the cord $ab$ (or a subset of ${X\choose 2}$ containing that cord) is
added to $\cL$.  To make this more precise, let $\cL'$ denote the resulting
set of cords on $X$ and let $\mathcal D_1$ denote a dendrogram on $X$ for
which the topology of the underlying tree is the same as that of $\mathcal
D$. If $\mathcal D_2$ is a dendrogram on $X$ such that the distances on the
cords in $\cL'$ induced by $\mathcal D_1$ and $\mathcal D_2$ coincide then, as
is easy to verify, the topologies of the underlying trees of $\mathcal D_1$
and $\mathcal D_2$, respectively, must be the same and so must be their
edge-weightings. Thus, $\cL'$ uniquely determines $\mathcal D$.

Although an intriguing question, apart from some recent results in
\cite{HP13}, not much is known about it (see \cite{DHS11} and \cite{HS13} for
some partial results in case the tree in question is unrooted).  By
formalising a dendrogram in terms of a certain edge-weighted $X$-tree (see the
next section for a precise definition of this concept as well as all the other
concepts mentioned below) and using the concept of a topological lasso which
was originally introduced for unrooted phylogenetic trees with leaf set $X$ in
\cite{DHS11} and extended to $X$-trees in \cite {HP13}, we study this question
in the form of when a set of cords of $X$ is a topological lasso for a given
$X$-tree $T$. In the context of this, we are particularly interested in
(set-inclusion) minimal topological lassos $\cL$ for $T$ for which $ \bigcup
\cL:=\bigcup_{A\in\cL} A=X$ holds.

For $T$ an $X$-tree, we show for any such minimal topological lasso $\cL$ for
$T$ that in case the graph $\Gamma(\cL)$ whose vertex set is $X$ and any two
distinct elements $x$ and $y$ in $X$ joined by an edge if $xy\in \cL$ -- see
Fig~\ref{fig:block-graph-motivation}(i) for an example of that graph for
$\cL=\{ab,cd,ef,ac,ce,ea\}$ -- is a block graph then the blocks of
$\Gamma(\cL)$ are in one-to-one correspondence with the non-leaf vertices of
$T$ (Corollary~\ref{cor:bijection}).  Furthermore, we establish in
Theorem~\ref{theo:transform} that any minimal topological lasso $\cL$ for $T$
can be transformed into a very special type of minimal topological lasso
$\cL^*$ for $T$ in that $\Gamma(\cL^*)$ is a claw-free block graph where a
graph is called {\em claw-free} if it does not contain a {\em claw}, that is,
the complete bipartite graph $K_{1,3}$ as an induced subgraph \cite{H72}.
%Calling a minimal topological
%lasso $\cL$ for $T$ {\em distinguished} if $\Gamma(\cL)$ 
%does not contain a {\em claw}, that is, the complete
%bipartite graph $K_{1,3}$, we show that every minimal
%topological lasso for $T$ can be transformed into
%a distinguished minimal topological lasso for $T$ via a 
%sequence of minimal topological lassos for $T$
%whose length is bounded by the number of 
%non-leaf vertices of $T$ 
%(Theorem~\ref{theo:transform}).
%Thus, to every minimal topological lasso we can associate
%a minimal topological lasso $\cL$ for which $\Gamma(\cL)$
%is a {\em claw-free} block graph, that is, a
%block graph that does not contain a claw. 
Claw-free graphs have been shown to enjoy numerous properties relating them
to, for example, perfect graphs, perfect matchings, and maximum independent
sets (see e.\,g.\,\cite{FFZ97} and \cite{CFHV12} for overviews).  Furthermore,
claw-free block graphs were related in \cite{BL09} to $k$-leaf powers of trees
and their spectrum was studied in \cite{GS01, MSST06} (see also \cite{BR13}
for a more general study of the adjacency matrix of such graphs).  Calling a
minimal topological lasso $\cL$ for $T$ {\em distinguished} if $\Gamma(\cL)$
is a claw-free block graph, we present in Theorem~\ref{theo:characterization}
a characterisation of a distinguished minimal topological lasso for $T$ in
terms of the novel concept of a cluster marker map for $T$. In addition, we
characterise when a distinguished minimal topological lasso for $T$ gives rise
to a distinguished minimal topological lasso for a subtree of $T$
(Theorem~\ref{theo:subtree}) and also present a partial answer to the
canonical analogue of a question raised for supertrees of unrooted
phylogenetic trees in \cite{DHS11}.

%As it turns out, a $X$-tree cannot be uniquely
%reconstructed from a distinguished
%minimal topological lasso $\cL$ for it since there exist
%non-equivalent $X$-trees that are lassoed by the same set of
%cords -- see e.g. Fig.~\ref{fig:block-graph-motivation} for an 
%example. However the situation changes if the actual distances
%on the cords in $\cL$ are being used.
%We conclude with presenting such a construction. 

The paper is organised as follows. In Section~\ref{sec:terminology}, we
introduce relevant terminology surrounding $X$-trees and lassos. In
Section~\ref{sec:gamma-l-graph}, we collect first properties of the graph
$\Gamma(\cL)$ associated to a topological lasso $\cL$ and in
Section~\ref{sec:blockgraph}, we establish Corollary~\ref{cor:bijection}. In
Section~\ref{sec:distinguished}, we commence our study of distinguished
minimal topological and establish Theorem~\ref{theo:transform}. In
Section~\ref{sec:sufficient}, we present a sufficient condition for when a
minimal topological lasso is distinguished (Theorem~\ref{theo:
  distinguished-lasso-verification}) and in
Section~\ref{sec:characterization-distinguished}, we prove
Theorem~\ref{theo:characterization}. We conclude with
Section~\ref{sec:subtree} where we establish Theorem~\ref{theo:subtree} and
also outline directions for further research.
%We conclude with Section~\ref{sec:reconstruction}
%where we present an algorithm for reconstructing
%an $X$-tree $T$ from the distances on the elements of the
%cords of a distinguished minimal topological lasso for
%$T$.


\section{Basic terminology and assumptions}\label{sec:terminology}
In this section, we introduce some relevant basic terminology surrounding
$X$-trees, their edge-weighted counterparts, and lassos.  Assume throughout
the paper that $X$ is a finite set with at least 3 elements and that, unless
stated otherwise, all sets $\cL$ of cords of $X$ considered in this paper
satisfy the property that
 %We call a subset $\cL\subseteq {X\choose 2}$ of $X$ a
%{\em covering} of $X$ is 
$X=\bigcup \cL$. 


\subsection{$X$-trees}
A {\em rooted tree} $T$ is a tree with a unique distinguished vertex called
the {\em root} of $T$, denoted by $\rho_T$. Throughout the paper, we assume
that the degree of the root of a rooted tree is at least two.  A {\em rooted
  phylogenetic $X$-tree}, or {\em $X$-tree} for short, is a rooted tree
$T=(V,E)$ with no degree two vertices but possibly the root $\rho_T$ whose
leaf set is $X$. We call an $X$-tree $T$ a {\em star-tree on $X$} if every
leaf of $T$ is adjacent with the root of $T$
%and refer to an $X$-tree that is not the star-tree on $X$
%as a {\em non-degenerate} $X$-tree.

Suppose for the following that $T$ is an $X$-tree. Then we call a vertex of
$T$ that is not a leaf of $T$ an {\em interior vertex} of $T$ and denote the
set of interior vertices of $T$ by $\iV(T)$.  We call an edge of $T$ that is
incident with a leaf of $T$ a {\em pendant edge} of $T$ and every edge of $T$
that is not a pendant edge an {\em interior edge} of $T$.  Extending some of
the terminology for directed graphs to $X$-trees, we call for all vertices
$v\in V(T)-\{\rho_T\}$ an edge $e\in E(T)$ a {\em parent edge of $v$} if $e$
is incident with $v$ and lies on the path from the root $\rho_T$ of $T$ to
$v$. We refer to the vertex incident with $e$ but distinct from $v$ as a {\em
  parent} of $v$.
 
Suppose for the following that $v$ is an interior vertex of $T$. If $v$ is not
the root of $T$ then we call an edge $e\in E(T)$ a {\em child edge of $v$} if
$e$ is incident with $v$ but is not crossed by the path from $\rho_T$ to $v$.
In addition, we call every edge incident with $\rho_T$ a {\em child edge of
  $\rho_T$}.  We call the vertex incident with a child edge of an interior
vertex $w$ of $T$ but distinct from $w$ a {\em child of $w$} and denote the
set of all children of $v$ by $ch_T(v)$.
% and put $deg^+(w)=|ch_T(v)|$. 
 %If $deg^+(w)=2$ holds for all vertices 
 % If $|ch_T(w)|=2$ holds for all vertices 
% $w\in \iV(T)$ then we call $T$ {\em binary}. 
We call a vertex $w\in V(T)$ distinct from $v$ a {\em descendant} of $v$ if
either $w$ is a child of $v$ or there exists a path from $v$ to $w$ that
crosses a child of $v$.  We denote the set of leaves of $T$ that are also
descendants of $v$ by $L_T(v)$. If $v$ is a leaf of $T$ then we put
$L_T(v):=\{v\}$. If there is no ambiguity as to which $X$-tree $T$ we are
referring to then, for all $v\in V(T)$, we will write $L(v)$ rather than
$L_T(v)$ and $ch(v)$ rather than $ch_T(v)$.

We call a non-empty subset $L\subsetneq X$ of leaves of $T$ such that $L=L(v)$
holds for somen $v\in \iV(T)$ a {\em pseudo-cherry} of $T$. In that case, we
also call $v$ the {\em parent} of that pseudo-cherry.
%If $\{x_1,\ldots, x_k\}$, $k\geq 2$, is a pseudo-cherry of
%some $X$-tree $T$ then we will sometimes write $x_1,\ldots, x_k$
%rather than $\{x_1,\ldots, x_k\}$.  
Note that every $X$-tree on three or more leaves must contain at least one
pseudo-cherry. Also note that a pseudo-cherry of size two is a {\em cherry} in
the usual sense (see e.g. \cite{semple2003phylogenetics}).
%In the special
%case that $|X|=3$, say $X=\{a,b,c\}$, and that $T$ has a cherry
%$a,b$, say,
%we call $T$ a {\em (rooted) triplet} and denote $T$ by $ab|c$ (or,
%equivalently, $c|ab$). 

For $x$ and $y$ distinct elements in $X$, we call the unique vertex of $T$
that simultaneously lies on the path from $x$ to $y$, on thne path from $x$ to
$\rho_T$, and on the path from $y$ to $\rho_T$ the {\em last common ancestor
  of $x$ and $y$}, denoted by $lca_T(x,y)$. More generally, for any subset
$Y\subseteq X$ of size three or more, we denote the subtree of $T$ with leaf
set $Y$ and vertices of degree two suppressed (except possibly the root) by
$T|_Y$ and call the root of $T|_Y$ the {\em last common ancestor of $Y$},
denoted by $lca_T(Y)$.

Finally, suppose that $T'$ is be a further $X$-tree. Then we say that $T$ and
$T'$ are {\em equivalent} if there exists a bijection $\phi:V(T)\to V(T')$
that extends to a graph isomorphism between $T$ and $T'$ that is the identity
on $X$ and maps the root $\rho_T$ of $T$ to the root $\rho_{T'}$ of $T'$.
%We say that 
%$T'$ {\em refines} $T$ if, up to equivalence,
%$T$ can be obtained from $T'$ by collapsing edges of $T'$
%(see e.\,g.\, \cite{semple2003phylogenetics}). In that case, we will also refer to
%$T'$ as a {\em refinement} of $T$. 


\subsection{Edge-weighted $X$-trees and lassos}

Suppose for the following again that $T$ is an $X$-tree.  An {\em edge
  weighting $\omega$ of $T$} is a map $\omega :E(T)\to \mathbb R_{\geq 0}$
that maps every edge of $T$ to a non-negative real. Suppose that $\omega$ is
an edge-weighting for $T$. Then we call the pair $(T,\omega)$ an {\em
  edge-weighted} $X$-tree and $\omega$ {\em proper} if $\omega(e)>0$ holds for
every interior edge $e$ of $T$. We denote the distance induced by $(T,\omega)$
on the leaves of $T$ by $D_{(T,\omega)}$ and call $\omega$ {\em equidistant}
if
\begin{enumerate}
\item[(i)] $D_{(T,\omega)}(x,\rho_T)=  D_{(T,\omega)}(y,\rho_T)$, for all
$x,y\in X$, and
\item[(ii)] $D_{(T,\omega)}(x,u)\geq D_{(T,\omega)}(x,v)$, for all $x\in X$
  and all $u,v\in V$ such that $u$ is encountered before $v$ on the path from
  $\rho_T$ to $x$.
\end{enumerate} 

Suppose $\cL$ is a set of cords of $X$. Then we call two edge-weighted
$X$-trees $(T_1,\omega_1)$ and $(T_2,\omega_2)$ {\em $\cL$-isometric} if
$D_{(T_1,\omega_1)}(x,y)=D_{(T_2,\omega_2)}(x,y)$ holds for all cords $xy\in
\cL$. We say that $\cL$ is a {\em topological lasso} for $T$ if for every
$X$-tree $T'$ and any equidistant, proper edge-weightings $\omega$ of $T$ and
$\omega'$ of $T$', we have that $T$ and $T'$ are equivalent whenever
$(T,\omega)$ and $(T',\omega')$ are $\cL$-isometric.  If $\cL$ is a
topological lasso for $T$ then we also say that $T$ is {\em topologically
  lassoed} by $\cL$. Moreover, we say that $\cL$ is a {\em (set-inclusion)
  minimal topological lasso for $T$} if $\cL$ is a topological lasso for $T$
but no cord $A\in \cL$ can be removed from $\cL$ such that $\cL-\{A\}$ is
still a topological lasso for $T$.  For ease of readability, if the $X$-tree
to which a topological lasso $\cL$ refers is of no relevance to the
discussion, we will simply say that $\cL$ is a topological lasso.

To illustrate some of these definitions, let $X=\{a,\ldots, f\}$ and let $\cL$
be the set of cords such that $\Gamma(\cL)$ is the graph depicted in
Fig.~\ref{fig:block-graph-motivation}(i).  Using
e.\,g.\,\cite[Theorem~7.1]{HP13} (see also
Theorem~\ref{theo:characterization-topology} below) it is easy to see that the
$X$-trees depicted in Fig.~\ref{fig:block-graph-motivation}(ii) and (iii)
respectively are topologically lassoed by $\cL$. In fact, $\cL$ is a minimal
topological lasso for both of them.
\begin{figure}[h]
\begin{center}
\input{figures/dist-min-lass/lasso-2-trees.pdft}
\end{center}
\caption{\label{fig:block-graph-motivation}
(i) The graph $\Gamma(\cL)$ with vertex set $X=\{a,b,\ldots,f\}$
for the set $\cL=\{ab,cd,ef,ac,ce,ea\}$. (ii) and (iii)
Two non-equivalent $X$-trees $T$ and $T'$
that are both topologically lassoed by $\cL$. In fact, 
$\cL$ is a minimal topological lasso for either one of them.
}
\end{figure}


\section{The graphs $\Gamma(\cL)$ and 
$G(\cL,v)$}\label{sec:gamma-l-graph}


In this section, we investigate properties of the graph $\Gamma(\cL)$
associated to a %non-empty
set $\cL$ of cords of $X$. We start by remarking that if there is no danger of
confusion, we denote an edge $\{a,b\}$ of $\Gamma(\cL)$ by $ab$ rather than
$\{a,b\}$.

To establish our first structural result for $\Gamma(\cL)$ (see
Proposition~\ref{prop:gamma-l-connected}), we require further terminology.
Suppose $T$ is an $X$-tree, $v\in \iV(T)$, and $\cL$ is a set of cords of
$X$. Then we call the graph $G_T(\cL,v)=(V_{T,v},E_{T,v})$ with vertex set
$V_{T,v}$ the set of all child edges of $v$ and edge set $E_{T,v}$ the set of
all $\{e,e'\}\in {V_{T,v}\choose 2}$ for which there exist leaves $a,b\in X$
such that $e$ and $e'$ are edges on the path from $a$ to $b$ in $T$ and $ab\in
\cL$ holds the {\em child-edge graph of $v$ (with respect to $T$ and
  $\cL$)}. Note that in case there is no danger of ambiguity with regards to
the $X$-tree $T$ we are referring to, we will write $G(\cL,v)$ rather than
$G_T(\cL,v)$ and $V_v$ and $E_v$ rather than $V_{T,v}$ and $E_{T,v}$. The next
result which was originally established in \cite[Theorem 7.1]{HP13} states a a
crucial property of child-edge graphs.

\begin{thm}\label{theo:characterization-topology}
Suppose $T$ is an $X$-tree and 
$\cL$ is a set of cords of $X$. 
Then the following are equivalent:
\begin{enumerate}
\item[(i)] $\cL$ is a topological lasso for $T$.
\item[(ii)] for every vertex $v\in \iV(T)$, the graph
$G(\cL,v)$ is a clique.
\end{enumerate}
\end{thm}



Denoting for an $X$-tree $T$, a topological lasso $\cL$ for $T$, and an
interior vertex $v\in \iV(T)$ the set of all cords $ab\in \cL$ for which
$v=lca_T(a,b)$ holds by $\cA(v)$, Theorem~\ref{theo:characterization-topology}
readily implies $|\cA(v)|\geq {|ch(v)|\choose 2}$.  The next observation is
almost trivial yet central to the paper and concerns the special case that
$\cL$ is a minimal topological lasso for $T$. Its proof which combines a
straightforward counting argument with
Theorem~\ref{theo:characterization-topology} is left to the interested
reader. To able to state it, we denote for an interior vertex $v\in \iV(T)$
and a child edge $e\in E(T)$ of $v$ the child of $v$ indicent with $e$ by
$v_e$.

\begin{lem}\label{lem:size-A(v)}
  Suppose $T$ is an $X$-tree and $\cL$ is a minimal topological lasso for
  $T$. Then, for all $v\in \iV(T)$, we have $|\cA(v)|={|ch(v)|\choose 2}$. In
  particular, for any two distinct child edges $e_1$ and $e_2$ of $v$ there
  exists precisely one pair $(a_1,a_2)\in L(v_{e_1})\times L(v_{e_2})$ such
  that $a_1a_2\in\cL$.
\end{lem}
%%%% the following proof is correct but commented out
%to shorten the paper 2013-04-29
%\begin{proof}
%Suppose for contradiction that there exist some $v\in \iV(T)$
%such that $|\cA(v)|> {deg^+(v)\choose 2}$. Then there exists
%an edge $e=\{e_1,e_2\}\in E_{T,v}$ and
%pairs $(a,b),(a',b')\in L(v_{e_1})\times L(v_{e_2})$
%such that  $ab,a'b'\in\cL$ and $|\{a_,b,a',b'\}|\geq 3$.
%Since $\cL$ is a minimal 
%topological lasso for $T$ and  so 
%$|\cL|=\sum_{w\in \iV(T)} {deg^+(v)\choose 2}$ is 
%implied by Theorem~\ref{theo:characterization-topology}, 
 %there must exist some vertex $w\in \iV(T)$ such that 
%$|\cA(w)|< {deg^+(v)\choose 2}$. But then $G(\cL,w)$ 
%is not a clique and so, in view of 
%Theorem~\ref{theo:characterization-topology}, 
% $\cL$ is not a topological lasso for $T$
%which is impossible.
%\epf
%\end{proof}

Note that Lemma~\ref{lem:size-A(v)} immediately implies that any two minimal
topological lassos for the same $X$-tree must be of equal size.

To be able to establish Proposition~\ref{prop:gamma-l-connected}, we require a
further definition.  Suppose $T$ is an $X$-tree and $\cL$ is a topological
lasso for $T$. Then for all $v\in V(T)$, we denote by $\Gamma_v(\cL)$ the
subgraph of $\Gamma(\cL)$ induced by $L(v)$. Note that in case $v$ is a leaf
of $T$ and thus an element in $X$ the only vertex in $\Gamma_v(\cL)$ is $v$
(and $E(\Gamma_v(\cL))=\emptyset$).


\begin{pro}~\label{prop:gamma-l-connected} Suppose $T$ is an $X$-tree and
  $\cL$ is a topological lasso for $T$.  Then, for all $v\in V(T)$, the graph
  $\Gamma_v(\cL)$ is connected.  In particular, $\Gamma(\cL)$ is connected.
\end{pro}
\begin{proof}
  Assume for contradiction that there exists some vertex $v\in V(T)$ such that
  $\Gamma_v(\cL)$ is not connected. Then $v$ cannot be a leaf of $T$ and so
  $v\in \iV(T)$ must hold. Without loss of generality we may assume that $v$
  is such that for all descendants $w\in V(T)$ of $v$ the induced graph
  $\Gamma_w(\cL)$ is connected. Since $\cL$ is a topological lasso for $T$ and
  so $G(\cL,v)$ is a clique, it follows for any two distinct children
  $v_1,v_2\in ch(v)$ that there exists a pair $(x_1,x_2)\in L(v_1)\times
  L(v_2)$ such that $x_1x_2\in \cL$.  Since the assumption on $v$ implies that
  the graphs $\Gamma_{w}(\cL)$ are connected for all children $w\in ch(v)$, it
  follows that $\Gamma_v(\cL)$ is connected which is impossible.  Thus,
  $\Gamma_v(\cL)$ is connected, for all $v\in V(T)$.  That $\Gamma(\cL)$ is
  connected is a trivial consequence.  \qquad
\end{proof}


\section{The case that $\Gamma(\cL)$ is a 
block graph}\label{sec:blockgraph}



To establish a further property of $\Gamma(\cL)$ which we will do in
Proposition~\ref{prop:x-i-unique}, we require some terminology related to
block graphs (see e.\,g.\,\cite{diestel}).  Suppose $G$ is a graph. Then a
vertex of $G$ is called a {\em cut vertex} if its deletion (plus its incident
edges) disconnects $G$. A graph is called a {\em block} if it has at least one
vertex, is connected, and does not contain a cut vertex. A {\em block of a
  graph $G$} is a maximal connected subgraph of $G$ that is a block and a
graph is called a {\em block graph} if all of its blocks are cliques. For
convenience, we refer to a block graph with vertex set $X$ as a {\em block
  graph on X}.

As the example of the two minimal topological lassos $
%\cL=
\{ab,cd,ef,ac,ce,ea\}$ and $
%\cL'=
\{ab, bc,$ $cd, de, ef, fa\}$ for the $\{a,\ldots,f\}$-tree depicted in
Fig.~\ref{fig:block-graph-motivation}(ii) indicates, the graph $\Gamma(\cL)$
associated to a minimal topological lasso $\cL$ may be but need not be a block
graph.  However if it is
%$\cL$ is a minimal topological lasso for some $X$-tree 
%$T$ such that $\Gamma(\cL)$ is a block graph 
then Lemma~\ref{lem:size-A(v)} can be strengthened to the following central
result where for all positive integers $n$ we put $\langle n\rangle
:=\{1,\ldots, n\}$ and set $\langle 0\rangle:=\emptyset$.

\begin{pro}\label{prop:x-i-unique}
  Suppose $T$ is an $X$-tree and $\cL$ is a minimal topological lasso for $T$
  such that $\Gamma(\cL)$ is a block graph.  Let $v\in \iV(T)$ and let
  $v_1,\ldots, v_l\in V(T)$ denote the children of $v$ where $l=|ch(v)|$.
  Then, for all $i\in \langle l\rangle$, there exists a unique leaf $x_i\in
  L(v_i)$ such that $x_sx_t\in \cL$ holds for all $s,t\in\langle l\rangle$
  distinct.
\end{pro}
\begin{proof}
  For all $v\in \iV(T)$ and all $w\in ch(v)$, put
$$
L^v_w:=\{x\in L(w): \mbox{ there exist } w'\in ch(v)-\{w\}
\mbox{ and } y\in L(w')
\mbox{ such that } xy\in \cL  \}.
$$
We need to show that $|L^v_w|=1$ holds for all $v\in \iV(T)$ and all $w\in
ch(v)$. To see this, note first that since $G(\cL,v)$ is a clique for all
$v\in \iV(T)$, we have, for all $w\in ch(v)$ with $v\in \iV(T)$, that
$L^v_w\not=\emptyset$.  Thus, $|L^v_w|\geq 1$ holds for all such $v$ and $w$.


To establish equality, suppose there exists some interior vertex $v\in \iV(T)$
and some child $v_1\in ch(v)$ such that $|L^v_{v_1}|\geq 2$.  Choose two
distinct leaves $x_1$ and $y_1$ of $T$ contained in $L^v_{v_1}$ and denote the
parent edge of $v_1$ by $e_1$. Note that $v_1=v_{e_1}$.  Since $y_1\in
L^v_{v_1}$, there exists a child edge $e_2$ of $v$ distinct from $e_1$ and
some $x_2\in L(v_{e_2})$ such that $y_1x_2\in\cL$. In view of $x_1\in
L^v_{v_1}$, we distinguish between the cases that (i) $x_1z\not\in\cL$ holds
for all $z\in L(v_{e_2})$ and (ii) there exists some $z\in L(v_{e_2})$ such
that $x_1z\in\cL$.

Assume first that Case~(i) holds.  Then since $x_1\in L^v_{v_1}$ there exists
a further child edge $e_3$ of $v$ and some $y_3\in L(v_{e_3})$ such that
$x_1y_3\in\cL$.  Since, by Theorem~\ref{theo:characterization-topology},
$G(\cL,v)$ is a clique and so $\{e_2,e_3\}$ is an edge in $G(\cL,v)$, there
must exist leaves $y_2\in L(v_{e_2})$ and $x_3\in L(v_{e_3})$ such that
$y_2x_3\in\cL$. By Proposition~\ref{prop:gamma-l-connected}, the graphs
$\Gamma_{v_{e_i}}(\cL)$, $i=2,3$, are connected and, by definition, clearly do
not share a vertex. Hence, there must exist a cycle in $\Gamma(\cL)$ whose
vertex set contains $\bigcup_{j\in\langle 3\rangle} \{x_j,y_j\}$. But then
$x_1x_2\in \cL$ must hold since $\Gamma(\cL)$ is a block graph and so every
block in $\Gamma(\cL)$ is a clique. By Lemma~\ref{lem:size-A(v)} applied to
$e_1$ and $e_2$, it follows that $x_1=y_1$ as $x_1,y_1\in L(v_1)$ and
$y_1x_2\in \cL$ which is impossible.

Now assume that Case~(ii) holds, that is, there exists some $z\in L(v_{e_2})$
such that $x_1z\in\cL$. Then Lemma~\ref{lem:size-A(v)} applied to $e_1$ and
$e_2$ implies $x_1=y_1$ as $y_1x_2\in\cL$ also holds which is impossible.
\qquad
\end{proof}

To illustrate Proposition~\ref{prop:x-i-unique}, let $T$ be the $X$-tree
depicted in Fig.~\ref{fig:block-graph-motivation}(ii) and let $\cL$ be the set
of cords of $X$ whose $\Gamma(\cL)$ graph is pictured in
Fig.~\ref{fig:block-graph-motivation}(i).  Using the notation from
Proposition~\ref{prop:x-i-unique} and labelling the children of the root of
$T$ from left to right by $v_1$, $v_2$ and $v_3$ it is easy to see that
Proposition~\ref{prop:x-i-unique} holds for $x_1=a$, $x_2=c$ and $x_3=e$.

The next result is the main result of this section and lies at the heart of
Corollary~\ref{cor:bijection} which provides for an $X$-tree $T$ and a minimal
topological lasso $\cL$ for $T$ such that $\Gamma(\cL)$ is a block graph a
close link between the blocks of $\Gamma(\cL)$, the interior vertices of $T$
and, for all $v\in \iV(T)$, the child-edge graphs $G(\cL,v)$. To establish it,
we denote for all $v\in V(T)-\{\rho_T\}$ the parent edge of $v$ by $e_v$ and
the set of blocks of a graph $G$ by $Block(G)$.


\begin{thm}\label{theo:unique-block}
  Suppose $T$ is an $X$-tree and $\cL$ is a minimal topological lasso for $T$
  such that $\Gamma(\cL)$ is a block graph. Then, for all $v\in \iV(T)$, there
  exists a unique block $B\in Block(\Gamma(\cL))$ such that $v=lca_T(V(B))$.
\end{thm}
\begin{proof}
  We first show existence. Suppose $v\in \iV(T)$. Let $v_1,\ldots,v_l\in V(T)$
  denote the children of $v$ where $l=|ch(v)|$.  By
  Proposition~\ref{prop:x-i-unique}, there exists, for all $i\in\langle
  l\rangle$, a unique leaf $x_i\in L(v_i)$ such that, for all $s,t\in \langle
  l\rangle$ distinct, we have $x_sx_t\in \cL$. Put $A=\{x_1,\ldots,x_l\}$.
  Clearly, $v=lca_T(A)$ and the graph $G(v)$ with vertex set $A$ and edge set
  $E=\{\{x,y\}\in {A\choose 2}: xy\in\cL\}$ is a clique.  Then since
  $\Gamma(\cL)$ is a block graph there must exist a block $B\in
  Block(\Gamma(\cL))$ that contains $G(v)$ as an induced subgraph.

  We claim that the graphs $G(v)$ and $B$ are equal.  In view of the facts
  that $A\subseteq V(B)$, the blocks in a bock graph are cliques, and $G(v)$
  is a clique it suffices to show that $V(B)\subseteq A$. Suppose for
  contradiction that there exists some $y\in V(B)-A$. Note first that $yx\in
  \cL$ must hold for all $x\in A$.  Next note that $y$ cannot be a descendant
  of $v$ since otherwise there would exist some $i\in\langle l\rangle$ such
  that $y\in L(v_i)$. Choose some $j\in \langle l\rangle-\{i\}$. Then
  Lemma~\ref{lem:size-A(v)} applied to $e_{v_i}$ and $e_{v_j}$
%the respective parent edges of $v_i$ and  $v_j$
  implies $x_i=y$ as $yx_j,x_ix_j\in\cL$ which is impossible.

  Choose some $x\in A$ and put $w=lca_T(x,y)$. Then $v$ is a descendant of $w$
  and $w=lca_T(x,y)$ holds for all $x\in A$. Let $w_1\in V(T)$ and $w_2\in
  \iV(T)$ denote two distinct children of $w$ such that $y\in L(w_1)$ and
  $x\in L(w_2)$. Then Lemma~\ref{lem:size-A(v)} applied to $e_{w_1}$ and
  $e_{w_2}$
%the respective parent edges of $w_1$ and $w_2$ 
  implies $x_i=x_j$ for all $i,j\in\langle l\rangle$ distinct since $yx\in\cL$
  holds for all $x\in A$ which is impossible.  Thus, $V(B)\subseteq A$, as
  required. This concludes the proof of the existence part of the theorem.

  We next show uniqueness.  Suppose for contradiction that there exists some
  $v\in \iV(T)$ and distinct blocks $B,B'\in Block(\Gamma(\cL))$ such that
  $lca_T(B)=v=lca_T(B')$.
%Without loss of generality
%we may assume that $v$ is minimal, that is, for all
%descendents $w\in \iV(T)$ of $v$ there exists a unique  block $B_w$
%in $\Gamma(\cL)$ such that $w=lca_T(B_w)$. 
  Since every block of $\Gamma(\cL)$ contains at least two vertices as
  $\Gamma(\cL)$ is connected and $|X|\geq 3$, we may choose distinct vertices
  $b_1,b_2\in V(B)$ and $b_1',b_2'\in V(B')$ such that
  $lca_T(b_1,b_2)=lca_T(B)=v=lca_T(B')=lca_T(b_1',b_2')$.  Note that $b_1b_2$
  and $b_1'b_2' $ must be cords in $\cL$ as $B$ and $B'$ are cliques of
  $\Gamma(\cL)$.  We distinguish between the cases that (i)
  $\{b_1,b_2\}\cap\{b_1',b_2'\}=\emptyset $ and (ii)
  $\{b_1,b_2\}\cap\{b_1',b_2'\}\not=\emptyset $.

  We first show that Case (i) cannot hold. Assume for contradiction that Case
  (i) holds, that is, $\{b_1,b_2\}\cap\{b_1',b_2'\}=\emptyset $.  We claim
  that $lca_T(b_1,b_1')=v$. Assume for contradiction that
  $w:=lca_T(b_1,b_1')\not=v$. Let $v_1\in ch(v)$ such that $v_1$ lies on the
  path from $v$ to $w$. If $v\not =lca_T(b_2,b_2')$ then there exists a
  descendant $w'\in V(T)$ of $v$ such that $lca_T(b_2,b_2')=w'$. Let $v_2\in
  ch(v)$ such that $v_2$ that lies on the path from $v$ to $w'$. Then
  Lemma~\ref{lem:size-A(v)} applied to $e_{v_1}$ and $e_{v_2}$
%the respective parent edges of $v_1$ and $v_2$ 
implies $b_1=b_1'$ and $b_2=b_2'$ as $b_1b_2,b_1'b_2'\in \cL$
which is impossible. Thus, $lca_T(b_2,b_2')=v$
must hold. Let $v_2,v_2'\in ch(v)$ such that $b_2\in L(v_2)$
and $b_2'\in L(v_2')$. 
%By Lemma~\ref{lem:size-A(v)} applied to 
%$e_{v_2}$ and $e_{v_2'}$
%the respective parent edges of $v_2$ and $v_2'$ 
%there must exists a pair $(c,c')\in L_T(v_2)\times L_T(v_2')$ 
%such that $cc'\in \cL$.
Then since $b_1,b_1'\in L(v_1)$ and $b_1b_2, b_1'b_2'\in \cL$,
Proposition~\ref{prop:x-i-unique} implies $b_1'=b_1$.
%, $b_2=c$, and that $c'=b_2'$. 
Consequently, $\{b_1,b_2\}\cap\{b_1',b_2'\}\not=\emptyset $ which is
impossible.
%
%$C$: $b_1,b_2=c,c'=b_2',b_1'=b_1$ is a cycle in $\Gamma(\cL)$
%and so there must exist a block $B$ of $\Gamma(\cL)$ that contains
%$C$. Since every block in a blockgraph is a clique it follows that
%$b_1b_2'\in\cL$. But $b_1'b_2'\in \cL$ holds too and so, 
%by Lemma~\ref{lem:size-A(v)}, we obtain that $b_1=b_1'$ which is impossible.
Thus, $lca_T(b_2,b_2')=v$ cannot hold and so
$$
lca_T(b_1,b_1')=v,
$$ 
as claimed.  Swapping the roles of $b_1,b_1'$ and $b_2,b_2'$ in the previous
claim implies that $v= lca_T(b_2,b_2')$ must hold, too.  For $i=1,2$ let
$v_i,v_i'\in ch(v)$ such that $b_i\in L(v_i)$ and $b_i'\in L(v_i')$.  Then, by
Lemma~\ref{lem:size-A(v)}, there exist pairs $(c,c')\in L(v_1)\times L(v_1')$
and $(d,d')\in L(v_2)\times L(v_2')$ such that $cc',dd'\in\cL$.  Since
$(b_1,b_2)\in L(v_1)\times L(v_2)$ and $(b_1',b_2')\in L(v_1')\times L(v_2')$
and $b_1b_2,b_1'b_2'\in\cL$, Proposition~\ref{prop:x-i-unique} implies that
$c=b_1$, $b_2=d$, $d'=b_2'$ and $c'=b_1'$. But then $C$: $c'=b_1',b_2'=d',
d=b_2, b_1=c,c'$ is a cycle in $\Gamma(\cL)$.  Since $\Gamma(\cL)$ is a block
graph it follows that there must exist a block $B^C$ in $\Gamma(\cL)$ that
contains $C$.  Since $\{b_1,b_2\}\subseteq V(B^C)\cap V(B)$ and two distinct
blocks of a block graph can share at most one vertex it follows that $B^C$ and
$B$ must coincide. Since $\{b_1',b_2'\}\subseteq V(B^C)\cap V(B')$ holds too,
similar arguments imply that $B^C$ must also coincide with $B'$.  Thus, $B$
and $B'$ must be equal which is impossible.  Hence Case~(i) cannot hold, as
required.

Thus, Case (ii) must hold, that is,
$\{b_1,b_2\}\cap\{b_1',b_2'\}\not=\emptyset $. Since any two distinct blocks
in a block graph can share at most one vertex it follows that
$|\{b_1,b_2\}\cap\{b_1',b_2'\}|=1$.  Without loss of generality we may assume
that $b_1=b_1'$.  We first claim that
$$
lca_T(b_2,b_2')=v.
$$
Assume to the contrary that $lca_T(b_2,b_2')\not=v$.  Then there exist
distinct children $v_1,v_2 \in ch(v)$ such that $b_1\in L(v_1)$ and
$b_2,b_2'\in L(v_2)$ hold.  Since both $b_1b_2$ and $b_1'b_2'=b_1b_2'$ are
cords in $\cL$, Lemma~\ref{lem:size-A(v)} applied to $e_{v_1}$ and $e_{v_2}$
%the respective parent edges of $v_1$ and $v_2$ 
implies $b_2'=b_2$. Hence, $|\{b_1,b_2\}\cap\{b_1',b_2'\}|=2$ which 
is impossible. Thus, $lca_T(b_2,b_2')=v$, as claimed.

Let $v_1,v_2,v_2'\in ch(v)$ such that $b_1\in L(v_1)$, $b_2\in L(v_2)$, and
$b_2'\in L(v_2')$.  By Lemma~\ref{lem:size-A(v)}, there exist some $(c,c')\in
L(v_2)\times L(v_2')$ such that $cc'\in \cL$.  Since we also have
$(b_1,b_2)\in L(v_1)\times L(v_2)$ with $b_1b_2\in \cL$ holding and
$(b_1,b_2')\in L(v_1)\times L(v_2')$ with $b_2'b_1=b_2'b_1'\in \cL$ holding,
Proposition~\ref{prop:x-i-unique} implies that $b_2=c$ and $b_2'=c'$. Hence,
$C$: $b_1=b_1',b_2'=c', c=b_2,b_1$ is a cylce in $\Gamma(\cL)$ and so similar
arguments as in the corresponding subcase for Case (i) imply that
%
%Hence, there must exist
%a block $B^*$ in $\Gamma(\cL)$ that contains $C$. 
%Since $\{b_1,b_2\}\subseteq V(B^*)\cap V(B)$ and two distinct block in
%a block graph can share at most one vertex it follows that $B^*$ and 
%$B$ must conincide. Since $\{b_1,b_2'\}\subseteq V(B^*)\cap V(B')$
%similar arguments imply that $B^*$ must also conincide with $B'$.
%Thus, 
%
$B$ and $B'$ must coincide which is impossible. Thus, $lca_T(b_2,b_2')=v$
cannot hold which concludes the discussion of Case (ii) and thus the proof of
the uniqueness part of the theorem.
\end{proof}

In view of Theorem~\ref{theo:unique-block}, we denote for $T$ an $X$-tree, a
minimal topological lasso $\cL$ for $T$ such that $\Gamma(\cL)$ is a block
graph, and a vertex $v\in \iV(T)$ the unique block $B$ in $\Gamma(\cL)$ for
which $v=lca_T(V(B))$ holds by $B_v^{\cL}$, or simply by $B_v$ if the set
$\cL$ of cords is clear from the context.  Moreover, we denote for all $x\in
L(v)$ the child of $v$ on the path from $v$ to $x$ by $v_x$.

\begin{cor}\label{cor:bijection}
  Suppose $T$ is an $X$-tree and $\cL$ is a minimal topological lasso for $T$
  such that $\Gamma(\cL)$ is a block graph. Then the map
$$
\psi:\iV(T)\to Block(\Gamma(\cL)) : v\mapsto B_v
$$
is a bijection with inverse map $\psi^{-1}:Block(\Gamma(\cL))\to \iV(T)$:
$B\mapsto lca_T(V(B))$. Moreover, the map
$$
\chi:Block(\Gamma(\cL)) \to \{G(\cL, v)\,:\, v\in \iV(T)\}
: B\mapsto G(\cL,\psi^{-1}(B))
$$
is bijective and,
 for all $B\in Block(\Gamma(\cL))$, the map
$$
\xi_B:V(B) \to V_{\psi^{-1}(B)} 
: x\mapsto e_{\psi^{-1}(B)_x}
$$ 
induces a graph isomorphism between $B$ and the child-edge graph $G(\cL,
\psi^{-1}(B))$.
\end{cor}
\begin{proof}
  In view of Theorem~\ref{theo:unique-block}, the map $\psi$ is clearly
  well-defined and injective. To see that $\psi$ is surjective let $B\in
  Block(\Gamma(\cL))$ and put $v_B=lca_T(V(B))$. Clearly, $v_B\in
  \iV(T)$. Since $B_{v_B}=\psi(v_B)$ is a block in $\Gamma(\cL)$ for which
  also $v_B=lca_T(V(B_{v_B}))$ holds, Theorem~\ref{theo:unique-block} implies
  that $\psi(v_B)$ and $B$ must coincide.  Consequently, $\psi$ must also be
  surjective and thus bijective. That the map $\psi^{-1}$ is as stated is
  trivial.  Combined with Theorem~\ref{theo:characterization-topology}, the
  bijectivity of the map $\psi$ implies in particular that, for all $B\in
  Block(\Gamma(\cL))$, the map $\xi_B: V(B) \to V_{\psi^{-1}(B)} $ from $V(B)$
  to the vertex set $V_{\psi^{-1}(B)} $ of the child-edge graph $G(\cL,
  \psi^{-1}(B))$ induces a graph isomorphism between $B$ and $G(\cL,
  \psi^{-1}(B))$.

  To see that the map $\chi$ is bijective note first that $\chi$ is
  well-defined since $\psi^{-1}(B)\in \iV(T)$ holds for all blocks $B\in
  Block(\Gamma(\cL))$. To see that $\chi$ is injective assume that there exist
  blocks $B_1,B_2\in Block(\Gamma(\cL))$ such that $\chi(B_1)=\chi(B_2)$ but
  $B_1$ and $B_2$ are distinct.  Then $\psi^{-1}(B_1)\not= \psi^{-1}(B_2)$ as
  $\psi$ is a bijection from $\iV(T)$ to $Block(\Gamma(\cL)) $.  Combined with
  the fact that, for all $B\in Block(\Gamma(\cL))$, the map $\xi_B$ induces a
  graph isomorphism between $B$ and $G(\cL, \psi^{-1}(B))$ it follows that
  $\chi(B_1)=G(\cL,\psi^{-1}(B_1))\not=G(\cL,\psi^{-1}(B_2))=\chi(B_2)$ which
  is impossible. Thus, $\chi$ must be injective. Combined with the fact that
  $|Blocks(\Gamma(\cL))|=|\iV(T)|=|\{G(\cL, v)\,:\, v\in \iV(T)\}|$ it follows
  that $\chi$ must also be surjective and thus bijective.
\end{proof}

%Note that Corollary~\ref{cor:bijection} immediatelly implies that
%the map 
%$\chi:Block(\Gamma(\cL)) \to\{G(\cL, v)\,:\, v\in \iV(T)\}$:
%$B\mapsto G(\cL, lca_T(V(B))$ is also bijective

\section{A special type of minimal topological lasso}
\label{sec:distinguished}

Returning to the example depicted in Fig.~\ref{fig:block-graph-motivation}, it
should be noted that, in addition to being a block graph, $\Gamma(\cL)$ enjoys
a very special property where $\cL$ is the minimal topological lasso
considered in that example. More precisely, every vertex of $\Gamma(\cL)$ is
contained in at most two blocks.  Put differently, $\Gamma(\cL)$ is a
claw-free graph. Motivated by this, we call a minimal topological lasso $\cL$
{\em distinguished} if $\Gamma(\cL)$ is a claw-free block graph.  Note that
such block graphs are precisely the {\em line graphs of (unrooted) trees}
where for any graph $G$ the associated line graph has vertex set $E(G)$ and
two vertices $a,b\in E(G)$ are joined by an edge if $a\cap b\not=\emptyset$
\cite{H72}.

In this section, we show in Theorem~\ref{theo:transform} that distinguished
minimal topological lassos are a very special type of lasso in that for every
$X$-tree $T$ any minimal topological lasso $\cL$ for $T$ can be transformed
into a distinguished minimal topological lasso $\cL^*$ for $T$ via a {\em
  repeated application} (i.\,e.\,$l\geq 0$ applications) of the rule:

\begin{enumerate}
\item[(R)] If $xy,yz\in \cL$ and $lca_T(y,z)$ is a descendant of $lca_T(x,y)$
  in $T$ then delete $xy$ from the edge set of $\Gamma(\cL)$ and add the edge
  $xz$ to it.
\end{enumerate}

Before we make this more precise which we will do next, we remark that since a
topological lasso for a star tree is in particular a distinguished minimal
topological lasso for it, we will for this and the next two sections restrict
our attention to {\em non-degenerate} $X$-trees, that is, $X$-trees that are
not star trees on $X$.

Suppose $T$ is a non-degenerate $X$-tree and $\cL$ is a set of cords of
$X$. Let $\iV(T)$ denote a set of colors and let
$$
\gamma_{(\cL,T)}:\cL\to \iV(T):\, ab\mapsto lca_T(a,b)
$$
denote an edge coloring of $\Gamma(\cL)$ in terms of the interior vertices of
$T$. Note that if $\cL$ is a topological lasso for $T$ then
Theorem~\ref{theo:characterization-topology} implies that $\gamma_{(\cL,T)}$
is surjective. Returning to Rule (R), note that a repeated application of that
rule to such a set $\cL$ of cords results in a set $\cL'$ of cords that is
also a topological lasso for $T$. Furthermore, note that if $\cL$ is a minimal
topological lasso for $T$ then $\cL'$ is necessarily also a minimal
topological lasso for $T$. Finally note for all $v\in \iV(T)$ that
$|\gamma_{(\cL,T)}^{-1}(v)|=1$ or $|\gamma_{(\cL,T)}^{-1}(v)|\geq 3$ must hold
in this case.


\begin{lem}
  \label{lem:3-edges-cycle}
  Suppose $T$ is a non-degenerate $X$-tree and $\cL$ is a minimal topological
  lasso for $T$.  Put $\gamma=\gamma_{(\cL,T)}$ and assume that $v\in \iV(T)$
  such that $|\gamma^{-1}(v)|\geq 3 $. Then for any three pairwise distinct
  cords $c_1,c_2,c_3\in \gamma^{-1}(v)$, there exists a
%(not necessarily induced)
  cycle $C_v$ in $\Gamma(\cL)$ such that $c_1,c_2,c_3\in E(C_v)$ and, for all
  $c\in E(C_v)$, $\gamma(c)$ either equals $v$ or is a descendant of $v$.
\end{lem}
\begin{proof}
  Let $v\in \iV(T)$ and let $c_1=x_1y_1$, $c_2=x_2y_2$ and $c_3=x_3y_3$ denote
  three pairwise distinct cords in $\gamma^{-1}(v)$.  For all $i\in\langle
  3\rangle$, let $v_i\in ch(v) $ such that $v_i$ lies on the path from $v$ to
  $x_i$ in $T$ and let $w_i\in ch(v)$ such that $w_i$ lies on the path from
  $v$ to $y_i$ in $T$.  Then, by Lemma~\ref{lem:size-A(v)}, there exists
  unique pairs $(s_1,t_1)\in L(v_1)\times L(v_2)$, $(s_2,t_2)\in L(w_2)\times
  L(w_3)$, and $(s_3,t_3)\in L(w_1)\times L(v_3)$ such that, for all
  $i\in\langle 3\rangle$, we have $s_it_i\in \cL$.  Since for all such $i$, we
  also have that $x_i\in L(v_i)$ and $y_i\in L(w_i)$ and, by
  Proposition~\ref{prop:gamma-l-connected}, the graphs $\Gamma_{v_i}(\cL)$ and
  $\Gamma_{w_i}(\cL)$ are connected, it follows that there exists a
%(not necessarily induced)
  cycle $C_v$ in $\Gamma(\cL)$ that contains, for all $i\in\langle 3\rangle$,
  the cords $c_i$ and $s_it_i$ in its edge set.

  It remains to show that for every edge $c\in E(C_v)$, we have that
  $\gamma(c)$ either equals $v$ or is a descendant of $v$.  Suppose $c\in
  E(C_v)$.  If there exists some $i\in\langle 3\rangle$ such that
  $c\in\{c_i,s_it_i\}$ then $\gamma (c)=v$ clearly holds. So assume that
% there exists no $i\in\langle 3\rangle$ such that 
%$e\in\{c_i,s_it_i\}$. Then 
  this is not the case. Without loss of generality, we may assume that $c$
  lies on the path $P$ from $x_1$ to $s_1$ in $C_v$ that does not cross
  $y_1$. Since $P$ is a subgraph of $\Gamma_{v_1}(\cL)$ and, implied by
  Proposition~\ref{prop:gamma-l-connected}, every edge in $\Gamma_{v_1}(\cL)$
  is colored via $\gamma$ with a descendant of $v_1$, it follows that
  $\gamma(c)$ is a descendant of $v$.
\end{proof}

To establish Theorem~\ref{theo:transform}, we require further terminology.
Suppose $T$ is a non-degenerate $X$-tree, $\cL$ is a minimal topological lasso
for $T$, and $v\in \iV(T)$.  Then we denote by $H_{\cL}(v)$ the induced
subgraph of $\Gamma(\cL)$ whose vertex set is the set of all $x\in X$ that are
incident with some cord $c\in \cL$ for which $\gamma_{(\cL,T)}(c)=v$ holds.
%and whose edge set is $\{c\in\cL\,:\, \gamma(c)=v\}$. 
Moreover, we denote the set of cut vertices of a connected block graph $G$ by
$Cut(G)$ and note that in every connected block graph $G$ there must exist a
vertex that is contained in at most one block of $G$. This last observation is
central to the proof of Theorem~\ref{theo:transform}(ii).

%continuing with the notation from Lemma~\ref{lem:3-edges-cycle}
%and putting $Z=\{v\in \iV(T)\, :\, |\gamma^{-1}(v)|\geq 3\}$,
%we denote by $H_{\cL}(v)$ the subgraph of $\Gamma(\cL)$
%whose vertex set is $\bigcup_{v\in Z}V(C_v)$ and whose
%edge set is $\bigcup_{v\in Z}E(C_v)$.

\begin{thm}
  \label{theo:transform}
  Suppose $T$ is a non-degenerate $X$-tree and $\cL$ is a minimal topological
  lasso for $T$. Then there exists an ordering $\sigma: v_0, v_1,\ldots,
  v_k=\rho_T$, $k=|\iV(T)|$, of $\iV(T)$ such that the following holds:
  \begin{enumerate}
  \item[(i)] There exists a sequence $ \cL_{v_0}=\cL,\cL_{v_1},\ldots,
    \cL^{\dagger}=\cL_{v_{k}}$ of minimal topological lassos $\cL_{v_i}$ for
    $T$, $i\in \langle k\rangle$, such that for all such $i$, we have:
    \begin{enumerate}
    \item[(L1)] $\cL_{v_i}$ is obtained from $\cL_{v_{i-1}}$ via a repeated
      application of Rule (R) and $H_{\cL_{v_i}}(v_i)$ is a maximal clique in
      $\Gamma(\cL_{v_i})$.

%%There exists a sequence $\cL_{v_{i-1}}=Z_{i-1}^0,
%%Z_{i-1}^1,\ldots, Z_{i-1}^r=\cL_{v_i}$ 
%%of pairwise distinct minimal topological lassos $Z_{i-1}^q$
%%for $T$, $0\leq q\leq r$, such that for all $q\in\langle r\rangle$ we have that
%%$Z_{i-1}^q$ is obtained from $Z_{i-1}^{q-1}$ via an application of Rule (R) and 
%%$H_{Z_{i-1}^r}(v_i)$ is a maximal
%%clique in $Z_{i-1}^r$.
    \item[(L2)] For all $j\in\langle i-1\rangle$, $H_{\cL_{v_{i}}}(v_j)$ is a
      maximal clique in $\Gamma(\cL_{v_{i}})$.
    \end{enumerate}
    In particular,
%$\cL^*$ is a minimal topological lasso for $T$ and
    $\Gamma(\cL^{\dagger})$ is a block graph.
  \item[(ii)] If $\Gamma(\cL)$ is a block graph then there exists a sequence
    $\cL_{v_0}=\cL,\cL_{v_1},\ldots, \cL^*=\cL_{v_{k}}$ of minimal topological
    lassos $\cL_{v_i}$ for $T$, $i\in\langle k\rangle$, such that for all such
    $i$, we have:
    \begin{enumerate}
    \item[(L1')] $\cL_{v_i}$ is obtained from $\cL_{v_{i-1}}$ via a repeated
      application of Rule (R) and $\Gamma(\cL_{v_i})$ is a block graph.
    \item[(L2')] $\Gamma_{v_i}(\cL_{v_i})$ is a claw-free block graph.
    \end{enumerate}
    In particular, $\cL^*$ is a distinguished minimal topological lasso for
    $T$.
  \end{enumerate}
\end{thm}
\begin{proof}
  For all $i\in \langle k\rangle$, put $\cL_i=\cL_{v_{i}}$ and
  $\gamma_i=\gamma_{(\cL_i,T)}$.  Clearly, if $\cL$ is distinguished then the
  sequences as described in (i) and (ii) exist. So assume that $\cL$ is not
  distinguished.  For all $v\in \iV(T)$, let $l(v)$ denote the length of the
  path from the root $\rho_T$ of $T$ to $v$ and put $h=\max_{v\in
    \iV(T)}\{l(v)\}$.  Note that $h\geq 1$ as $T$ is non-degenerate.  For all
  $i\in \langle h\rangle$, let $V(i)\subseteq \iV(T)$ denote the set of all
  interior vertices $v$ of $T$ such that $l(v)=i$.  Let $\sigma$ denote an
  ordering of the vertices in $\iV(T)$ such that the vertices in $V(h)$ come
  first (in any order), then (again in any order) the vertices in $V(h-1)$ and
  so on with the last vertex in that ordering being $\rho_T$.

  (i) Suppose $v\in \iV(T)$. If $v\in V(h)$ then
%, by the definition of $\sigma$, 
  we may assume without loss of generality that $v=v_1$. Then $v_1$ is the
  parent of a pseudo-cherry of $T$ and so
  Theorem~\ref{theo:characterization-topology} implies that $H_{\cL}(v_1)$ is
  a maximal clique in $\Gamma(\cL)$. Thus, $\cL_1:=\cL$ is a minimal
  topological lasso for $T$ that satisfies Properties~(L1) and (L2).

  So assume that $v\not\in V(h)$. Then there exists some $|V(h)|< i\leq k$
  such that $v=v_i$. Without loss of generality, we may assume that $v_i$ is
  such that, for all $j\in \langle i-1\rangle$, $\cL_j$ is a minimal
  topological lasso for $T$ that satisfies Properties~(L1) and (L2).  If $v_i$
  is the parent of a pseudo-cherry of $T$ then similar arguments as before
  imply that $\cL_i:=\cL_{i-1}$ is a minimal topological lasso for $T$ that
  satisfies Properties~(L1) and (L2). So assume that $v_i$ is not the parent
  of a pseudo-cherry of $T$.
% To show that $\cL_{i-1}$ can
%be transformed into a minimal topological lasso 
%$\cL_{i}$ for $T$ that satisfies Properties~(L1) and (L2)
%by repeatedly applying of Rule (R), 
  We distinguish between the cases that $H_{\cL_{i-1}}(v)$ is a maximal clique
  in $\cL_{i-1}$ and that it is not.

  Assume first that $H_{\cL_{i-1}}(v)$ is a maximal clique in
  $\cL_{i-1}$. Then since $\cL_{i-1}$ is a minimal topological lasso for $T$
  that satisfies Properties~(L1) and (L2), it is easy to see that
  $\cL_{i}:=\cL_{i-1}$ is also a minimal topological lasso for $T$ that
  satisfies Properties~(L1) and (L2).  So assume that $H_{\cL_{i-1}}(v)$ is
  not a maximal clique in $\cL_{i-1}$.  Then $H_{\cL_{i-1}}(v)$ must contain
  three pairwise distinct edges, $e_1=x_1y_1$, $e_2=x_2y_2$, and $e_3=x_3y_3$
  say, such that $\{e_1, e_2,e_3\}$ is not the edge set of a $3$-clique in
  $H_{\cL_{i-1}}(v)$.
% the subgraph of $H_{\cL_{i-1}}(v)$ whose edge set is $\{e_1,e_2,e_3\}$
% is not a 3-clique.
  For all $i\in\langle 3\rangle $, put $z_i=lca_T(x_i,y_i)$.  Then
  Lemma~\ref{lem:3-edges-cycle} combined with a repeated application of Rule
  (R) to $\cL_{i-1}$ implies that, for all $i\in\langle 3\rangle$, we can find
  elements $x_i'\in L(z_i)$ such that
$$
\cL_{i-1}'=\cL_{i-1}-\{x_1y_1,x_2y_2,x_3y_3\}\cup 
\{x_1'x_2', x_2'x_3',x_3'x_1'\}
$$
is a minimal topological lasso for $T$ and the cords $x_1'x_2'$, $x_2'x_3'$,
and $x_3'x_1'$ form a $3$-clique in $H_{\cL_{i-1}'}(v)$.  Transforming
$\cL_{i-1}'$ further by processing any three pairwise distinct edges in
$H_{\cL_{i-1}'}(v)$ that do not already form a $3$-clique in the same way and
so on eventually yields a minimal topological lasso $\cL_{i}$ for $T$ such
that any three pairwise distinct edges in $H_{\cL_{i}}(v)$ form a
$3$-clique. But this implies that $H_{\cL_{i}}(v)$ is a maximal clique in
$\Gamma(\cL_{i})$
% Denoting by $\cL_{i-1}=Z_{i-1}^0,Z_{i-1}^1,\ldots, Z_{i-1}^r=
% \cL_i$  the resulting sequence of minimal topological lassos for $T$
%it follows that 
and so Property~(L1) is satisfied by $\cL_i$. Since only edges $e$ of
$\Gamma(\cL_{i-1})$ have been modified by the above transformation for which
$\gamma_{i-1}(e)=v$ holds and, by assumption, $\cL_{i-1}$ satisfies
Property~(L2) it follows that $\cL_{i}$ also satisfies that property.

Processing the successor of $v_i$ in $\sigma$ in the same way and so on yields
a minimal topological lasso $\cL^{\dagger}$ for $T$ for which
$\Gamma(\cL^{\dagger})$ is a block graph. This completes the proof of (i).


(ii) For all $i\in \langle k\rangle$ and all vertices $w\in \iV(T)$ put
$B^i_w=B^{\cL_i}_w$.  Suppose that $v\in \iV(T)$. If $v\in V(h)$ then
%, by the definition of $\sigma$, 
we may assume without loss of generality that $v=v_1$. Then $v$ is the parent
of a pseudo-cherry of $T$ and so $\cL_1:=\cL$ clearly satisfies
Properties~(L1') and (L2').

So assume that $v\not\in V(h)$. Then there exists some $|V(h)|< i\leq k$ such
that $v=v_i$. Without loss of generality, we may assume that $v_i$ is minimal,
that is, for all $j\in\langle i-1\rangle$, we have that $\cL_j$ is a minimal
topological lasso for $T$ that satisfies Properties~(L1') and (L2'). If $v$ is
the parent of a pseudo-cherry of $T$ then similar arguments as before imply
that $\cL_i:=\cL_{i-1}$ satisfies Properties~(L1') and (L2'). So assume that
$v$ is not the parent of a pseudo-cherry of $T$.  If $\Gamma_v(\cL_{i-1})$ is
a claw-free block graph then setting $\cL_i:=\cL_{i-1}$ implies that $\cL_i$
satisfies Properties~(L1') and (L2').

So assume that this is not the case, that is, there exists a vertex $x\in
L(v)$ that, in addition to being a vertex in the block $B_v^{i-1}$ of
$\Gamma(\cL_{i-1})$ and thus of $\Gamma_v(\cL_{i-1})$, is also a vertex in
$l\geq 2$ further blocks $B_1,\ldots, B_l$ of $\Gamma_v(\cL_{i-1})$ which are
also blocks in $\Gamma(\cL)$. Then there exists a path $P$ from $v$ to $x$ in
$T$ that contains, for all $l\geq 2$, the vertices $\psi^{-1}(B_1),\ldots,
\psi^{-1}(B_l)$ in its vertex set where $\psi:\iV(T)\to \Block(\Gamma(\cL))$
is the map from Corollary~\ref{cor:bijection}. Let $w\in ch(v)$ denote the
child of $v$ that lies on $P$. Note that since $l\geq 2$, we have $w\in
\iV(T)$. Without loss of generality, we may assume that $w=v_{i-1}$.  The fact
that $\Gamma(\cL_{i-1})$ is a block graph and so $\Gamma_{v_{i-1}}(\cL_{i-1})$
is a block graph combined with the fact that $\Gamma_{v_{i-1}}(\cL_{i-1})$ is
connected implies, in view of the observation preceding
Theorem~\ref{theo:transform}, that we may choose some $y\in
L(v_{i-1})-Cut(\Gamma_{v_{i-1}}(\cL_{i-1}))$. Then $y$ is a vertex in
precisely one block of $\Gamma_{v_{i-1}}(\cL_{i-1})$ and thus can be a vertex
in at most two blocks of $\Gamma_v(\cL_{i-1})$.  Consequently, $y\not=x$.
Applying Rule (R) repeatedly to $\cL_{i-1}$, let $\cL_i$ denote the set of
cords obtained from $\cL_{i-1}$ by replacing, for all $i\leq l\leq k$, every
cord of $\cL_{i-1}$ of the form $xa$ with $a\in V(B_{v_l}^{i-1})$ by the cord
$ya$. Then, by construction, $\cL_i$ is a minimal topological lasso for $T$
and $\Gamma(\cL_i)$ is a block graph. Hence, $\cL_i$ satisfies
Property~(L1'). Moreover, since $\Gamma_{v_{i-1}}(\cL_{i-1})$ is claw-free it
follows that $\Gamma_{v_i}(\cL_i)$ is claw-free and so $\cL_i$ satisfies
Property~(L2'), too.

Applying the above arguments to the successor of $v_i$ in $\sigma$ and so on
eventually yields a minimal topological lasso $\cL_k$ for $T$ that satisfies
Properties~(L1') and (L2'). Thus, $\Gamma_{v_k}(\cL_k) $ is a claw-free block
graph and, so, $\cL^*$ is a distinguished minimal topological lasso for $T$.
\end{proof}

To illustrate Theorem~\ref{theo:transform}, let $X=\{a,\ldots, f\}$ and
consider the $X$-tree $T'$ depicted in
Fig.~\ref{fig:block-graph-motivation}(iii) along with the set
$\cL=\{ad,ec,fa,ef,cd,bd\}$ of cords of $X$ which we depict in
Fig.~\ref{fig:transformation}(i) in the form of $\Gamma(\cL)$.
%
\begin{figure}[h]
  \begin{center}
    \input{figures/dist-min-lass/transformation.pdft}
  \end{center}
  \caption{ For $X=\{a,\ldots, f\}$ and the $X$-tree $T'$ pictured in
    Fig.~\ref{fig:block-graph-motivation}(iii), we depict in (i) the minimal
    topological lasso $\cL=\{ad,ec,fa,fe,cd,bd\}$ for $T'$ in the form of
    $\Gamma(\cL)$.  In the same way as in (i), we depict in (ii) the
    transformed minimal topological lasso $\cL^{\dagger}$ for $T'$ such that
    $\Gamma(\cL^{\dagger})$ is a block graph and in (iii) the distinguished
    minimal topological lasso $\cL^*$ for $T'$ obtained from $\cL^{\dagger}$
    -- see text for details.}
  \label{fig:transformation}
\end{figure}
%
Using for example Theorem~\ref{theo:characterization-topology}, it is
straight-forward to check that $\cL$ is a minimal topological lasso for $T'$
but $\Gamma(\cL)$ is clearly not a block graph and so $\cL$ is also not
distinguished. To transform $\cL$ into a distinguished minimal topological
lasso $\cL^*$ for $T'$ as described in Theorem~\ref{theo:transform}, consider
the ordering $v_1=lca_{T'}(e,f)$, $v_2=lca_{T'}(c,d)$, $v_3=lca_{T'}(a,d)$,
$v_4=\rho_{T'}$ of the interior vertices of $T'$. For all $i\in\langle
4\rangle$, put $\cL_i=\cL_{v_i}$. Then we first transform $\cL$ into a minimal
topological lasso $\cL^{\dagger}$ for $T'$ as described in
Theorem~\ref{theo:transform}(i). For this we have $\cL=\cL_0=\cL_1=\cL_2$ and
$\cL_3$ is obtained from $\cL_2$ by first applying Rule (R) to the cords $ec,
cd \in \cL_2$ resulting in the deletion of the cord $ce$ from $\cL_2 $ and the
addition of the cord $ed$ to $\cL_2$ and then to the cords $fe,ed\in\cL_2$
resulting in the deletion of the cord $ed$ from $\cL_2$ and the addition of
the cord $fd$ to it. The graph $\Gamma(\cL_3)$ is depicted in
Fig.~\ref{fig:transformation}(ii).  Note that $\cL_3=\cL^{\dagger}$ and that
although $\Gamma(\cL^{\dagger})$ is clearly a block graph $\cL^{\dagger}$ is
not distinguished.

To transform $\cL^{\dagger}$ into a distinguished minimal topological lasso
$\cL^*$ for $T'$, we next apply Theorem~\ref{theo:transform}(ii). For this, we
need only consider the vertex $d$ of $\Gamma(\cL^{\dagger})$ that is, we have
$\cL^{\dagger}=\cL_0=\cL_1=\cL_2=\cL_3$.  Since the child of $v_4$ on the path
from $v_4$ to $d$ is $v_3$, we may choose $a$ as the element $y$ in
$L(v_3)-Cut(\Gamma_{v_3}(\cL_3))$. Then applying Rule (R) to the cords
$bd,da\in \cL_3$ implies the deletion of $bd$ from $\cL_3$ and the addition of
the cord $ab$ to it. The resulting minimal topological lasso for $T'$ is
$\cL^*$ which we depict in Fig.~\ref{fig:transformation}(iii) in the form of
$\Gamma(\cL^*)$.
 
We conclude this section by remarking in passing that combined with
Theorem~\ref{theo:characterization-topology} which implies that any minimum
sized topological lasso for an $X$-tree $T$ must have $\sum_{v\in
  \iV(T)}{|ch(v)|\choose 2}$ cords, Theorem~\ref{theo:transform} and
Corollary~\ref{cor:bijection} imply that the minimum sized topological lassos
of an $X$-tree $T$ are precisely the minimal topological lassos of $T$.
 
\section{A sufficient condition for a minimal topological lasso to be
  distinguished}
\label{sec:sufficient}
In this section, we turn our attention towards presenting a sufficient
condition for a minimal topological lasso for some $X$-tree $T$ to be a
distinguished minimal topological lasso for $T$.  In the next section, we will
show that this condition is also sufficient.
%As in the previous section, we restrict our 
%attention to non-degenerate
%$X$-trees.

We start our discussion with introducing some more terminology.  Suppose $T$
is a non-degenerate $X$-tree. Put $cl(T)=\{L(v): v\in \iV(T)-\{\rho_T\}\}$ and
note that $cl(T)\not=\emptyset$. For all $A\in cl(T)$, put $cl_A(T):=\{B\in
cl(T): B\subsetneq A\}$ and note that a vertex $v\in \iV(T)-\{\rho_T\}$ is the
parent of a pseudo-cherry of $T$ if and only if $cl_{L(v)}(T)=\emptyset$.  For
$\sigma$ a total ordering of $X$ and $\min_{\sigma}(C)$ denoting the minimal
element of a non-empty subset $C$ of $X$, we call a map of the form
$$
f:cl(T)\to X:
A\mapsto \left\{\begin{array}{cc}
\min_{\sigma}(A-\{f(B): B\in cl_A(T)\})
 & \mbox{ if }cl_A(T)\not=\emptyset,\\
\min_{\sigma}(A)  & \mbox{ else. }
\end{array}
\right.
$$ 
a {\em cluster marker map (for $T$ and $\sigma$)}.  Note that since
$|\iV(T')|\leq |X|-1$ holds for all $X$-trees $T'$ and so $A-\{f(B): B\in
cl_A(T)\}\not=\emptyset$ must hold for all $A\in cl(T)$ with
$cl_A(T)\not=\emptyset $, it follows that $f$ is well-defined.
%Hence, $f_{\sigma}$ is well-defined.
%if it  satisfies
%the following two condition for all $A\in cl(T)$:  
%\begin{enumerate}
%\item[(C1)] $f_{\sigma}(A)\in A$, 
%\item[(C2)] $f_{\sigma}(A)$ is the minimal element in
%$A-\{f(B): B\in cl_A(T)\}$ with respect to $\sigma$.
%\end{enumerate} 
Also note that if $v\in \iV(T)$ is the parent of a pseudo-cherry $C$ of $T$
then $f(L(v))=f(C)= \min_{\sigma}(C)$ as $cl_C(T)=\emptyset$ in this
case. Finally, note that it is easy to see that a cluster marker map must be
injective but need not be surjective.


We are now ready to present a construction of a distinguished minimal
topological lasso which underpins the aforementioned sufficient condition that
a minimal topological lasso must satisfy to be distinguished.  Suppose that
$T$ is a non-degenerate $X$-tree, that $\sigma$ is a total ordering of $X$,
and that $f:cl(T)\to X$ is a cluster marker map for $T$ and $\sigma$. We first
associate to every interior vertex $v\in \iV(T)$ a set $\cL_{(T,f)} (v)$
defined as follows. Let $l_1,\ldots, l_{k_v}$ denote the children of $v$ that
are leaves of $T$ and let $v_1,\ldots v_{p_v}$ denote the children of $v$ that
are also interior vertices of $T$. Note that $k_v=0$ or $p_v=0$ might hold but
not both. Put ${\emptyset \choose 2}={\langle 1\rangle \choose
  2}=\emptyset$. Then we set
$$
\cL_{(T,f)}(v):=\bigcup_{\{i,j\}\in {\langle k_v\rangle\choose 2}}\{l_il_j\}
\cup
\bigcup_{\{i,j\}\in {\langle p_v\rangle\choose 2}}\{f(L(v_i))f(L(v_j))\}
\cup
\bigcup_{i\in \langle k_v\rangle,\,\,j\in \langle p_v\rangle} 
\{l_if(L(v_j))\}.
$$
Note that 
%since $T$ does not have interior vertices with just
%one child we have, for all $v\in \iV(T)$, that 
$|\cL_{(T,f)}(v)|\geq 1$
must hold for all $v\in \iV(T)$. Finally, we set
$$
\cL_{(T,f)}:=\bigcup_{v\in \iV(T)} \cL_{(T,f)}(v).
$$
%In case there is no danger of ambiguity as to which $X$-tree $T$,
%total ordering $\sigma$ of $X$ and cluster marker map
%$f_{\sigma}$ for $T$ and $\sigma$ we are referring to,
%we will write, for all $v\in \iV(T)$, $\cL(v)$ rather than $\cL_{(T,f)}(v)$,
%and $\cL$ rather than $\cL_{(T,f)}$.

To illustrate these definitions, consider the $X=\{a,\ldots, f\}$-tree $T'$
depicted in Fig.~\ref{fig:block-graph-motivation}(iii).  Let $\sigma$ denote
the lexicographic ordering of the elements in $X$. Then the map $f:cl(T')\to
X$ defined by setting
$$
f(\{c,d\})=c, \,\,\,
f(\{e,f\})=e,\,\,\mbox{ and } f(X-\{b\})=a
$$
is a cluster marker map for $T'$ and $\sigma$ and 
$\cL_{(T,f)}$
(or more precisely the graph $\Gamma(\cL_{(T',f)})$) is depicted in
Fig.~\ref{fig:block-graph-motivation}(i).

To help establish Theorem~\ref{theo: distinguished-lasso-verification}, we
require some intermediate results which are of interest in their own right and
which we present next. To this end, we denote for a vertex $v\in
\iV(T)-\{\rho_T\}$ by $T(v)$ the $L(v)$-tree with root $v$ obtained from $T$
by deleting the parent edge of $v$.


\begin{lem}
  \label{lem:insights}
  Suppose $T$ is a non-degenerate $X$-tree, $\sigma$ is a total ordering of
  $X$, and $f:cl(T)\to X$ is a cluster marker map for $T$ and $\sigma$. Then
  the following hold
  \begin{enumerate}
  \item[(i)] $\cL_{(T,f)}$ is a minimal topological lasso for $T$.
  \item[(ii)] $\Gamma(\cL_{(T,f)})$ is connected.
  \item[(iii)] If $v$ and $w$ are distinct interior vertices of $T$ then
    $|\bigcup \cL_{(T,f)}(v)\cap \bigcup \cL_{(T,f)}(w)|\leq 1$.
%such that $\bigcup \cL_{(T,f)}(v)\cap \bigcup \cL_{(T,f)}(w)\not=\emptyset$
%then $|\bigcup \cL_{(T,f)}(v)\cap \bigcup \cL_{(T,f)}(w)|= 1$.
  \item[(iv)] Suppose $x\in X$. Then there exist distinct vertices $v,w\in
    \iV(T)$ such that $x\in \bigcup \cL_{(T,f)}(v)\cap \bigcup \cL_{(T,f)}(w)$
    if and only if there exists some $u\in \iV(T)-\{\rho_T\}$ such that
    $x=f(L(u))$.
  \end{enumerate}
\end{lem}
\begin{proof}
  For all $v\in \iV(T)$, set $\cL(v)=\cL_{(T,f)}(v)$.

  (i) This is an immediate consequence of
  Theorem~\ref{theo:characterization-topology} and the respective definitions
  of the set $\cL(v)$ where $v\in \iV(T)$ and the graph $G(\cL',v)$ where
  $\cL'$ is a set of cords of $X$ and $v$ is again an interior vertex of $T$.
 

  (ii) This is an immediate consequence of
  Proposition~\ref{prop:gamma-l-connected} combined with
  Lemma~\ref{lem:insights}(i).


  (iii) This is an immediate consequence of the fact that, for all vertices
  $u\in \iV(T)$ and all $x,y\in \bigcup\cL(u)$ distinct, we have
  $u=lca_T(x,y)$.

  (iv) Let $x\in X$ and assume first that there exist distinct vertices
  $v,w\in \iV(T)$ such that $x\in \bigcup \cL(v)\cap \bigcup \cL(w)$ but
  $x\not =f(L_T(u))$, for all $u\in \iV(T)-\{\rho_T\}$. Then $x$ must be a
  leaf of $T$ that is simultaneously adjacent with $v$ and $w$ which is
  impossible. Thus, there must exist some $u\in \iV(T)$ such that $x=f(L(u))$.

  Conversely, assume that $x=f(L(u))$ for some $u\in \iV(T)-\{\rho_T\}$. Then
  $x\in L(u)$ and so there must exist an interior vertex $w$ of $T(u)$ that is
  adjacent with $x$.  Hence, $x\in\bigcup \cL(w)$.  Let $v$ denote the parent
  of $u$ in $T$ which exists since $u\not=\rho_T$.  Then $x=f(L(u))\in \bigcup
  \cL(v)$ and so $x\in \bigcup \cL(v)\cap \bigcup \cL(w)$, as required.
\end{proof}

Note that $u\in \{v,w\}$ need not hold for $u$, $v$ and $w$ as in the
statement of Lemma~\ref{lem:insights}(iv).  Indeed, suppose $T$ is the
$X=\{a,b,c,d\}$-tree with unique cherry $\{a,b\}$ and $d$ adjacent with the
root $\rho_T$ of $T$. Let $\sigma$ denote the lexicographic ordering of $X$
and let $f:cl(T)\to X$ be (the unique) cluster marker map for $T$ and
$\sigma$.  Set $x=b$, $v=lca_T(a,b)$, $w=\rho_T$.  Then $x=f(L(u))$ where
$u=lca_T(a,c)$ and $x\in \bigcup\cL(v)\cap \bigcup\cL(w)$ but $u\not\in
\{v,w\}$.



%\begin{lem}\label{lem:connected}
%Suppose $T$ is a non-degenerate  $X$-tree 
%and $f_{\sigma}:cl(T)\to X$ is a cluster
%marker map for $T$ with respect to some total ordering $\sigma$ 
%of $X$. Then $\Gamma(\cL_{(T,f_{\sigma})})$ is connected.
%\end{lem}
%\begin{proof}
%Set $f=f_{\sigma}$.
%%For all $v\in \iV(T)$, set $\cL(v)=\cL_{(T,f_{\sigma})}(v)$,
%%and set $\cL=\cL_{(T,f_{\sigma})}$.
%Note first that the claim holds clearly in case $T$ is the
%star tree since in that case $\Gamma(\cL)=\Gamma(\cL(\rho_T))$ and
%$\Gamma(\cL(\rho_T))$ is a clique. So assume from now on that $T$ is not
%a star tree. It suffices to show that for all $v\in \iV(T)$ the graph
%$\Gamma(\cL^v)$ is connected where $\cL^v:=\bigcup_{w\in \iV(T(v))}\cL(w)$.
%Assume for contradiction that there exists some $v\in \iV(T)$ such that
%$\Gamma(\cL^v)$ is not connected. Then $v$ cannot be the parent of a
%pseudo-cherry of $T$ since $\Gamma(\cL^v)=\Gamma(\cL(v))$ holds 
%in that case which implies that $\Gamma(\cL^v)$ 
%is connected as $\Gamma(\cL(v))$ is a clique. 
%Hence, at least one child of $v$ must be an interior vertex of $T$. 
%Without loss of generality we may assume that $v$ is minimal in the sense
%that for all vertices $w\in \iV(T(v))-\{v\}$  
%we have that $\Gamma(\cL^w)$ is connected. 
%
%Since $\Gamma(\cL(v))$ is a clique, to obtain the required contradiction
%it now suffices to show that for every child $w\in V(T)$ of $v$ that is not
%a leaf of $T$ there exist some $w'\in \iV(T)-\{v\}$ 
%such that $f_{\sigma}(L_T(w))\in \bigcup \cL(w')$. 
%Let $w\in V(T)$ denote such a child of $v$.
%Then $f(L_T(w))\in \bigcup \cL(v)\subseteq L_T(v)$. Consequently, there must
%exist a vertex $v'\in \iV(T(v))$ such that  $f(L_T(w))$ is adjacent with
%$v'$. But then $f(L_T(w))\in \bigcup \cL(v')$, as required. 
%\qquad \end{proof}

\begin{pro}
  \label{prop:block}
  Suppose $T$ is a non-degenerate $X$-tree, $\sigma$ is a total ordering of
  $X$, and $f:cl(T)\to X$ is a cluster marker map for $T$ and $\sigma$. Then
  $\Gamma(\cL_{(T,f)})$ is a connected block graph and every block of
  $\Gamma(\cL_{(T,f)})$ is of the form $\Gamma(\cL_{(T,f)}(v))$, for some
  $v\in \iV(T)$.
\end{pro}
\begin{proof}
  For all $v\in \iV(T)$, set $\cL(v)=\cL_{(T,f)}(v)$ and put
  $\cL=\cL_{(T,f)}$.  We claim that if $C$ is a cycle in $\Gamma(\cL)$ of
  length at least three then there must exist some $v\in \iV(T)$ such that $C$
  is contained in $\Gamma(\cL(v))$. Assume to the contrary that this is not
  the case, that is, there exists some cycle $C:u_1,u_2,\ldots,
  u_l,u_{l+1}=u_1$, $l\geq 3$, in $\Gamma(\cL)$ such that, for all $v\in
  \iV(T)$, we have that $C$ is not a cycle in $\Gamma(\cL(v))$. Without loss
  of generality, we may assume that $C$ is of minimal length. For all
  $i\in\langle l\rangle$, put $v_i=lca_T(u_i,u_{i+1})$. Then, by the
  construction of $\Gamma(\cL)$, we have for all such $i$ that $u_iu_{i+1}$ is
  an edge in $\Gamma(\cL(v_i))$ and, by the minimality of $C$, that
  $v_i\not=v_j$ for all $i,j\in\langle l\rangle$ distinct. Put $Y=V(C)$ and
  let $T'=T|_Y$ denote the $Y$-tree obtained by restricting $T$ to $Y$. Note
  that $lca_T(u_i,u_{i+1})=lca_{T'}(u_i,u_{i+1})$ holds for all $i\in\langle
  l\rangle$.  Thus, the map $\phi:E(C)\to \iV(T')$ defined by putting
  $u_iu_{i+1}\mapsto lca_{T}(u_i,u_{i+1})$, $i\in\langle l \rangle$, is
  well-defined.  Since $|E(C)|=l$ and for any finite set $Z$ with three or
  more elements a $Z$-tree has at most $|Z|-1$ interior vertices, it follows
  that there exist $i,j\in \langle l \rangle$ distinct such that
  $\phi(u_i,u_{i+1})=\phi(u_j,u_{j+1})$. Consequently,
  $v_i=lca_T(u_i,u_{i+1})=lca_T(u_j,u_{j+1})=v_j$ which is impossible and thus
  proves the claim.  Combined with Lemma~\ref{lem:insights}(ii) and (iii), it
  follows that $\Gamma(\cL)$ is a connected block graph. That the blocks of
  $\Gamma(\cL)$ are of the required form is an immediate consequence of the
  construction of $\Gamma(\cL)$.
\end{proof}


To be able to establish that $\cL_{(T,f)}(v)$ is indeed a distinguished
minimal topological lasso for $T$ and $f$ as above, we require a further
concept. Suppose $A, B\subseteq X$ are two distinct non-empty subsets of
$X$. Then $A$ and $B$ are said to be {\em compatible} if $A\cap
B\in\{\emptyset, A,B\}$. As is well-known (see
e.\,g.\,\cite{DHKMS11,semple2003phylogenetics}), for any $X$-tree $T'$ and any
two vertices $v,w\in V(T')$ the subsets $L(v)$ and $L(w)$ of $X$ are
compatible.

\begin{thm}
  \label{theo: distinguished-lasso-verification}
  Suppose $T$ is a non-degenerate $X$-tree, $\sigma$ is a total ordering of
  $X$ and $f:cl(T)\to X$ is a cluster marker map for $T$ and $\sigma$. Then
  $\cL_{(T,f)}$ is a distinguished minimal topological lasso for $T$.
\end{thm}
\begin{proof}
  For all $v\in \iV(T)$ put $\cL(v) = \cL_{(T,f)}(v)$ and put
  $\cL=\cL_{(T,f)}$.  In view of Proposition~\ref{prop:block} and
  Lemma~\ref{lem:insights}(i), it suffices to show that $\Gamma(\cL)$ is
  claw-free.  Assume to the contrary that this is not the case and that there
  exists some $x\in X$ that is contained in the vertex set of $m\geq 3$ blocks
  $A_1,\ldots,A_m$ of $\Gamma(\cL)$. Then, by Proposition~\ref{prop:block},
  there exist distinct interior vertices $v_1, \ldots, v_m$ of $T$ such that,
  for all $i\in\langle m\rangle$, we have $V(A_i)=\bigcup\cL(v_i)\subseteq
  L(v_i)$.  Since for all $v,w\in V(T)$ distinct, the sets $L(v)$ and $L(w)$
  are compatible, it follows that there exists a path $P$ from $\rho_T$ to $x$
  that contains the vertices $v_1,\ldots, v_m$ in its vertex set. Without loss
  of generality we may assume that $m=3$ and that, starting at $\rho_T$ and
  moving along $P$ the vertex $v_1$ is encountered first then $v_2$ and then
  $v_3$. Note that $cl_{L(v_i)}(T)\not=\emptyset$, for $i=1,2$.  Since $T$ is
  a tree and so $x$ can neither be adjacent with $v_1$ nor with $v_2$ it
  follows that there must exist for $i=1,2$ some $B_i\in cl_{L(v_i)}(T)$ such
  that $x=f(B_i)$. But this is impossible as $B_2\in cl_{L(v_1)}(T)$ and so
  $f(B_1)\not=f(B_2)$ as $f$ is a cluster marker map for $T$ and $\sigma$.
\end{proof}
 

\section{Characterising distinguished minimal topological lassos}
\label{sec:characterization-distinguished}

In this section, we establish the converse of Theorem~\ref{theo:
  distinguished-lasso-verification} which allows us to characterise
distinguished minimal topological lasso of non-degenerate $X$-trees.  We start
with a well-known construction for associating an unrooted tree to a connected
block graph (see e.\,g.\,\cite{diestel}).  Suppose that $G$ is a connected
block graph. Then we denote by $T_G$ the (unrooted) tree associated to $G$
with vertex set $Cut(G)\cup Block(G)$ and whose edges are of the from
$\{a,B\}$ where $a\in Cut(G)$, $B\in Block(G)$ and $a\in B$. Note that if a
vertex $v\in V(T_G)$ is a leaf of $T_G$ then $v\in Block(G)$.

Suppose $T$ is a non-degenerate $X$-tree and $\cL$ is a distinguished minimal
topological lasso for $T$.  Let $v$ denote an interior vertex of $T$ whose
children are $v_1\ldots,v_l$ where $l=|ch(v)|$. Then
Corollary~\ref{cor:bijection} combined with Proposition~\ref{prop:x-i-unique}
implies that for all $i\in\langle l\rangle$ there exists a unique leaf $x_i\in
L(v_i)$ of $T$ such that, for all $i,j\in\langle l\rangle$ distinct,
$x_ix_j\in \cL$ and $\{x_1,\ldots, x_l\}=V(B_v)$. Since $\Gamma(\cL)$ is
claw-free, every vertex of $B_v$ is contained in at most one further block of
$\Gamma(\cL)$. Thus, if $w\in V(B_v)$ and $w\in V(B) $ holds too for some
block $B\in Block(\Gamma(\cL))$ distinct from $B_v$ then $w$ must be a cut
vertex of $\Gamma(\cL)$. For every vertex $v'\in \iV(T)$ that is the child of
some vertex $v\in \iV(T)$, we denote the unique element $x\in L(v')$ contained
in $V(B_v)$ by $c_{B_{v'}}$ in case $x\in Cut(\Gamma(\cL))$.  Note that it is
not difficult to observe that, in the tree $T_{\Gamma(\cL)}$, the vertex
$c_{B_{v'}}$ is the vertex adjacent with $B_v$ that lies on the path from
$B_v$ to $B_{v'}$.

The following result lies at the heart of Theorem~\ref{theo:characterization}
and establishes a crucial relationship between the non-root interior vertices
of $T$ and the cut vertices of $\Gamma(\cL)$.

\begin{lem}
  \label{lem:bijection-theta}
  Suppose $T$ is an $X$-tree and $\cL$ is a distinguished minimal topological
  lasso for $T$. Then the map
$$
\theta :\iV(T)-\{\rho_T\} \to Cut(\Gamma(\cL)):\,\,\,v\mapsto c_{B_{v}}
$$ 
is bijective.
\end{lem}
\begin{proof}
  Clearly, $\theta$ is well-defined and injective. To see that $\theta$ is
  bijective let $ T_{\Gamma(\cL)}^-$ denote the tree obtained from $
  T_{\Gamma(\cL)}$ by suppressing all degree two vertices. Then
  $Block(\Gamma(\cL))=V(T_{\Gamma(\cL)}^-)$ and Corollary~\ref{cor:bijection}
  implies that $|Block(\Gamma(\cL))|=|\iV(T)|$ as $\Gamma(\cL)$ is a block
  graph. Since $\Gamma(\cL)$ is claw-free, we clearly also have
  $|Cut(\Gamma(\cL))|=|E(T_{\Gamma(\cL)}^-)|$. Combined with the fact that f
  $|V(T')|= |E(T')|+1$ holds for every tree $T'$, it follows that
  $|Cut(\Gamma(\cL))|=|Block(\Gamma(\cL))|-1=|\iV(T)|-1=
  |\iV(T)-\{\rho_T\}|$. Thus, $\theta$ is bijective.  \qquad
\end{proof}

%Since $\Gamma(\cL)$ is claw-free and so
%$ |\iV(T)-\{\rho_T\}|=|Cut(\Gamma(\cL))|$ holds as every vertex
%of $T'$ in $Cut(\Gamma(\cL))$ has degree two it follows that
%$\theta$ is in fact bijective. 



%Let $c\in Cut(\Gamma(\cL))$ be a cutvertex of $\Gamma(\cL)$ that is adjacent
%with $B_{\rho_T}$ and let $v_c\in \iV(T)$ denote a 
%child of $\rho_T$ in $T$ such that
%$c$ lies on the directed path from $B_{\rho_T}$ to $\psi(v_c)$ in $T^*$.
% 
%Continuing with the notation introduced above, we have the following
%result which, starting with the cutvertices in $Cut(\Gamma(\cL))$
%adjacent to $B_{\rho_T}$ in $T^*$,  
%forms the core of a top down approach for associating to every 
%vertex $v\in \iV(T)$ a unique element $c_v\in Cut(\Gamma(\cL))$
%such that $c_v\in L_T(v)$ (see below). 
%
%\begin{lem}\label{lem:construction}
%Suppose $T$ is an $X$-tree but not the star tree and let 
%$\cL$ be a distinguished minimal topological lasso for $T$. Let 
%$c\in Cut(\Gamma(\cL))$ be a cutvertex of $\Gamma(\cL)$ that is adjacent
%with $B_{\rho_T}$. Then $c \in L_T(v_c)$.
%\end{lem} 
%\begin{proof}
%Suppose for contradiction that $c \not\in L_T(v_c)$. Then there must exist
%a child $w\in \iV(T)-\{v_c\}$ of $\rho_T$ such that $c\in L_T(w)$. 
%Since $\Gamma(\cL)$ is claw-free, there exists a 
%unique vertex in $v_1 \in \iV(T)$ of $T$ such that $B_{\rho_T},c,B_{v_1}$
%is a path of length two in $T^*$.
%Combined with the fact that 
%$c\in V(B_{v_1})\cap V(B_{\rho_T})$, it follows that $B_{v_1}$
%is a vertex on the path $P$ from $B_w$ to $c$ in $T^*$. Note that $P$
%cannot contain $B_{\rho}$ as a vertex as otherwise $\Gamma_w(\cL)$ would 
%equal $\Gamma(\cL)$ and so $\Gamma_{v_c}(\cL)$ would be a
%subgraph of $\Gamma_w(\cL)$ which is impossible as $v_c$ and $w$
%are distinct children of $\rho_T$ and so  $\Gamma_w(\cL)$ and 
%$\Gamma_{v_c}(\cL)$ do not share a vertex. But then
%there exsists a cylce $C$ in $T'$ that contains
% $B_{\rho_T},B_{v_1}, B_w$ in its vertex set which is impossible as $T'$ is
%a tree. Hence,  $c \in L_T(v_c)$, as required. 
%\qquad \end{proof}
%
%Lemma~\ref{lem:construction} combined with Corollary~\ref{cor:bijection}
%and Proposition~\ref{prop:x-i-unique}
%implies that for every child $v\in \iV(T)$ of $\rho_T$ in $T$ there
%exists a unique cut vertex $c_v\in Cut(\Gamma(\cL)$ incident with
%$B_{\rho_T}$ such that $c_v\in L_T(v)$.
%
%
%
%Continuing with the assumptions and notations introduced above, we 
%next state the aforementioned top-down approach. For all ver
%
%
%
%Continuing with the presentation of the aforementioned top-down approach,
%we next delete $c$ and the edges $\{B_{\rho_T},c\}$ and $\{c,v_1\}$ from $T'$
%which results in two (unrooted and undirected)
%trees $T_1$ and $T_2$ one of which contains
%$B_{\rho_T}$. Without loss of generality we may assume that $B_{\rho_T}$ 
%is not a vertex of $T_1$. Then we associate a rooted directed tree 
%$T_1^{v_c}$ to $T_1$ by rooting it a $v_c$ and directing all edges away 
%from $v_c$.  We then process  $T_1^{v_c}$ as the tree $T^*$ before.
%
%If $B^*$ is an isolated vertex then we also remove $B^*$. 
%If  $B^*$ is not an isolated vertex then we associate to $T_2$ the
%rooted directed tree $T_2^{\rho_T}$ by rooting it at $\rho_T$ and directing
%all edges away from $\rho_T$. We then process  $T_2^{\rho_T}$ 
%as the tree $T^*$ before.
%
%
%
Armed with this result, we are now ready to establish the converse of
Theorem~\ref{theo: distinguished-lasso-verification} which yields the
aforementioned characterisation of distinguished minimal topological lassos of
non-degenerate $X$-trees.

\begin{thm}
  \label{theo:characterization}
  Suppose $T$ is a non-degenerate $X$-tree and $\cL$ is a set of cords of
  $X$. Then $\cL$ is a distinguished minimal topological lasso for $T$ if and
  only if there exists a total ordering $\sigma$ of $X$ and a cluster marker
  map $f$ for $T$ and $\sigma$ such that $\cL_{(T,f)}=\cL$.
\end{thm}
\begin{proof}
  Assume first that $\sigma$ is some total ordering of $X$ and that
  $f:cl(T)\to X$ is a cluster marker map for $T$ and $\sigma$. Then, by
  Theorem~\ref{theo: distinguished-lasso-verification}, $\cL_{(T,f)}$ is a
  distinguished minimal topological lasso for $T$.

  Conversely assume that $\cL$ is a distinguished minimal topological lasso
  for $T$ and consider an embedding of $T$ into the plane.  By abuse of
  terminology, we will refer to this embedding of $T$ also as $T$.  We start
  with defining a total ordering $\sigma$ of $X$.  To this end, we first
  define a map $t:\iV(T)-\{\rho_T\}\to \mathbb N$ by setting, for all $v\in
  \iV(T)-\{\rho_T\}$, $t(v)$ to be the length of the path from $\rho_T$ and
  $v$. Put $h=\max\{t(v)\,:\, v\in \iV(T)-\{\rho_T\}\}$ and note that $h\geq
  1$ as $T$ is non-degenerate.
%Without loss of generality, we may assume that $T$
%is embedded into the plane so that every interior vertex
%so that all edges of $T$ have unit weight and
%all interior vertices $v$ and $w$ of $T$
%for which $t(v)=t(w)$ holds are drawn equally far away from $\rho_T$. 
  Starting at the left most interior vertex $v$ of $T$ for which $t(v)=h$
  holds and moving, for all $l \in \langle h\rangle$, from left to right, we
  enumerate all interior vertices of $T$ but the root. We next put $n=|X|$ and
  $X=\langle n\rangle$ and relabel the elements in $X$ such that when
  traversing the circular ordering induced by $T$ on $X\cup\{\rho_T\}$ in a
  counter-clockwise fashion we have $\rho_T,1,2,3,\ldots, n,\rho_T$. To
  reflect this with regards to $\cL$, we relabel the elements of the cords in
  $\cL$ accordingly and denote the resulting distinguished minimal topological
  lasso for $T$ also by $\cL$.
%Let $\tau$ denote the usual ordering
%of the natural numbers. 

  By Lemma~\ref{lem:bijection-theta}, the map $\theta :\iV(T)-\{\rho_T\} \to
  Cut(\Gamma(\cL))$ defined in that lemma is bijective. Put
  $m=|Cut(\Gamma(\cL))|$ and let $v_1,v_2,\ldots, v_m$ denote the enumeration
  of the vertices in $\iV(T)-\{\rho_T\}$ obtained above. Also, set
  $Y=X-\{\theta(v_i): i\in \langle m\rangle\}$. Let $y_1,y_2,\ldots, y_l$
  denote an arbitrary but fixed total ordering of the elements of $Y$ where
  $l=|Y|$. Then we define $\sigma$ to be the total ordering of $X$ given by
$$
\sigma:\,\, \theta(v_1),\theta(v_2),\ldots, \theta(v_{i-1}),
\theta(v_i),\theta(v_{i+1}),
,\ldots, \theta(v_m),y_1,y_2,\ldots, y_l
$$ 
where $\theta(v_1)$ is the minimal element and $y_l$ is the maximal element.
Note that if $v\in \iV(T)$ is the parent of a pseudo-cherry $C$ of $T$ then
$\theta(v)=\min_{\sigma}C$.


We briefly interrupt the proof of the theorem to illustrate these definitions
by means of an example. Put $X=\langle 13\rangle$ and consider the $X$-tree
$T$ depicted in Fig.~\ref{fig:illustration-main-theorem}(i) (ignoring the
labelling of the interior vertices for the moment) and the distinguished
minimal topological lasso $\cL$ for $T$ pictured in the form of $\Gamma(\cL)$
in Fig.~\ref{fig:illustration-main-theorem}(ii). Then the labelling of the
interior vertices of $T$ gives the enumeration of those vertices considered in
the proof of Theorem~\ref{theo:characterization}. The total ordering $\sigma$
of $X$ restricted to the elements in $\{\theta(v_1),\ldots, \theta(v_6)\}$ is
$3,5,12,1,10,7$.
%
\begin{figure}[h]
  \begin{center}
    \input{figures/dist-min-lass/main-theorem.pdft}
  \end{center}
  \caption{ For $X=\langle 13\rangle$ and the depicted $X$-tree $T$, the
    enumeration of the interior vertices of $T$ considered in the proof of
    Theorem~\ref{theo:characterization} is indicated in (i). With regards to
    this enumeration and the distinguished minimal topological lasso $\cL$ for
    $T$ pictured in the form of $\Gamma(\cL)$ in (ii), the total ordering
    $\sigma$ of $X$ considered in that proof restricted to the elements in
    $\{\theta(v_1),\ldots, \theta(v_6)\}$ is $3,5,12,1,10,7$.}
  \label{fig:illustration-main-theorem}
\end{figure}

Returning to the proof of the theorem, we claim that the map $f:cl(T)\to X$
given, for all $A\in cl(T)$, by setting $f(A)=\theta(lca(A))$ is a cluster
marker map for $T$ and $\sigma$ where for all such $A$ we put
$lca(A)=lca_T(A)$. Indeed, suppose $A\in cl(T)$.  Then
$\theta(lca(A))=c_{B_{lca(A)}}\in L(lca(A))$ holds by construction.  We
distinguish between the cases that $cl_A(T)\not =\emptyset$ and that
$cl_A(T)=\emptyset$. If $cl_A(T)\not =\emptyset$ then since $\theta$ is
bijective it follows that $\theta(lca(A))\not=\theta(v)$ holds for all
descendants $v\in \iV(T)$ of $lca(A)$.  Combined with the definition of
$\sigma$, we obtain $f(A)=\theta(lca(A))=\min_{\sigma}(A-\{\theta(lca(D))\,:\,
D\in cl_A(T)\})= \min_{\sigma}(A-\{f(D)\,:\, D\in cl_A(T)\})$, as required.
If $cl_A(T)=\emptyset$ then, as was observed above,
$f(A)=\theta(lca(A))=\min_{\sigma}A$. Thus, $f$ is a cluster marker map for
$T$ and $\sigma$, as claimed.

It remains to show that $\cL_{(T,f)}=\cL$. To see this note first that, by
Theorem~\ref{theo: distinguished-lasso-verification}, $\cL_{(T,f)}$ is a
distinguished minimal topological lasso for $T$. Since
Lemma~\ref{lem:size-A(v)} implies that any two minimal topological lasso for
$T$ must be of the same size and thus $|\cL_{(T,f)}|=|\cL|$ holds, it
therefore suffices to show that $\cL\subseteq \cL_{(T,f)}$. Suppose $a,b\in X$
distinct such that $ab\in \cL$.  Then there exists some interior vertex $v\in
\iV(T)$ such that $v=lca_T(a,b)$.  Hence, $a,b\in V(B_v)$. We claim that
$ab\in \cL_{(T,f)}(v)$.  To establish this claim, we distinguish between the
cases that (i) $a\in ch(v)$ and (ii) that $a\not\in ch(v)$.

Assume first that Case (i) holds, that is, $a$ is a child of $v$.  If $b\in
ch(v)$ then the claim is an immediate consequence of the definition of
$\cL_{(T,f)}(v)$. So assume that $b\not\in ch(v)$. Let $v'\in \iV(T)$ denote
the child of $v$ for which $b\in L(v)$ holds. Then
$b=c_{B_{v'}}=\theta(v')=f(L(v'))$ follows by the observation preceding
Lemma~\ref{lem:bijection-theta} combined with the fact that $b\in
V(B_v)$. Hence, $ab=af(L(v'))\in \cL_{(T,f)}(v)$, as claimed.

Assume next that Case (ii) holds, that is, $a$ is not a child of $v$.  In view
of the previous subcase it suffices to consider the case that $b\not\in
ch(v)$. Let $v',v''\in \iV(T)$ denote the children of $v$ such that $a\in
L(v')$ and $b\in L(v'')$.  Then, again by the observation preceding
Lemma~\ref{lem:bijection-theta} combined with the fact that $a,b\in V(B_v)$,
we have $a=c_{B_{v'}}=\theta(v')=f(L(v'))$ and
$b=c_{B_{v''}}=\theta(v'')=f(L(v''))$ and so $ab=f(L(v'))f(L(v''))\in
\cL_{(T,f)}(v)$ follows, as claimed.  This concludes the proof of the claim
and thus the proof of the theorem.
\end{proof}

We now take a brief break from our study of distinguished minimal topological
lassos to point out a sufficient condition for a set of cords to be a strong
lasso for some $X$-tree which is implied by
Theorem~\ref{theo:characterization}. To make this more precise, we need to
introduce some more terminology from \cite{HP13}. Suppose $T$ is an $X$-tree
and $\cL$ is a set of cords of $X$.  Then $\cL$ is called an {\em equidistant
  lasso} for $T$ if, for all equidistant, proper edge-weightings $\omega$ and
$\omega'$ of $T$, we have that $\omega=\omega'$ holds whenever $(T,\omega)$
and $(T,\omega')$ are $\cL$-isometric. Moreover, $\cL$ is called a {\em strong
  lasso} for $T$ if $\cL$ is simultaneously an equidistant and a topological
lasso for $T$ (see \cite{DHS11} for more on such lassos in the unrooted case).

Like a topological lasso for a $X$-tree $T$, an equidistant lasso $\cL$ for
$T$ can also be characterised in terms of a property of the child-edge graph
$G(\cL,v)$ associated to $T$ and $\cL$ where $v\in \iV(T)$. Namely, a set
$\cL$ of cords of $X$ is an equidistant lasso for an $X$-tree $T$ if and only
if, for every vertex $v\in \iV(T)$, the graph $G(\cL,v)$ has at least one edge
(see \cite[Theorem 6.1]{HP13}).  Since for $\sigma$ some total ordering of $X$
and $f:\iV(T)-\{\rho_T\}\to X$ a cluster marker map for $T$ and $\sigma$ the
graphs $G(\cL_{(T,f)},v)$ clearly satisfy this property for all $v\in \iV(T)$,
it follows that $\cL_{(T,f)}$ is also an equidistant lasso for $T$ and thus a
strong lasso for $T$. Defining a strong lasso $\cL$ of an $X$-tree to be {\em
  minimal} in analogy to when a topological lasso is minimal,
Theorem~\ref{theo:characterization} implies



\begin{cor}
  \label{corollary:strong-lasso-characterization}
  Suppose $T$ is a non-degenerate $X$-tree, $\cL$ is a set of cords of $X$,
  $\sigma$ is a total ordering of $X$, and $f:cl(T)\to X$ is a cluster marker
  map for $T$ and $\sigma$.  Then $\cL_{(T,f)}$ is a minimal strong lasso for
  $T$.
\end{cor}


\section{Heredity of distinguished minimal topological lassos}
\label{sec:subtree}

In this section, we turn our attention to the problems of characterising when
a distinguished minimal topological lasso of an $X$-tree $T$ induces a
distinguished minimal topological lasso for a subtree of $T$ and, conversely
when distinguished minimal topological lassos of $X$-trees can be combined to
form a distinguished minimal topological lasso of a supertree for those trees
(see e.\,g.\,\cite{BE00} for more on such trees). This will also allow us to
partially answer the rooted analogon of a question raised in \cite{DHS11} for
supertrees within the unrooted framework.  To make this more precise, we
require further terminology.  Suppose $\cL$ a set of cords of $X$ and
$Y\subseteq X$ is a non-empty subset. Then we set
$$
\cL|_Y=\{ab\in\cL\,:\, a,b\in Y\}.
$$
Clearly, $\Gamma(\cL|_Y)$ is the subgraph of $\Gamma(\cL)$ induced by $Y$ but
$Y=\bigcup \cL|_Y$ need not hold. Moreover, if $\cL$ is a minimal topological
lasso for an $X$-tree $T$ and $|Y|\geq 3$ such that every interior vertex of
$T$ is also an interior vertex of $T|_Y$ then
Theorem~\ref{theo:characterization-topology} implies that $\cL|_Y$ is a
minimal topological lasso for $T|_Y$. In particular, $\Gamma(\cL|_Y)$ must be
connected in this case.  The next result is a strengthening of this
observation.

\begin{thm}
  \label{theo:subtree}
  Suppose $T$ is an $X$-tree, $\cL$ is a distinguished minimal topological
  lasso for $T$, and $Y\subseteq X$ is a subset of size at least three. Then
  $\cL|_Y$ is a distinguished minimal topological lasso for $T|_Y$ if and only
  if $\Gamma(\cL|_Y)$ is connected.
\end{thm}
\begin{proof}
  Assume first that $\cL|_Y$ is a distinguished minimal topological lasso for
  $T|_Y$. Then, by Proposition~\ref{prop:gamma-l-connected}, $\Gamma(\cL|_Y)$
  is connected.

  Conversely, assume that $\Gamma(\cL|_Y)$ is connected. Then the statement
  clearly holds if $T$ is the star tree on $X$. So assume that $T$ is
  non-degenerate. Let $Y\subseteq X$ be of size at least three and assume
  first that $T|_Y$ is the star tree on $Y$. We claim that $\Gamma(\cL|_Y)$ is
  a clique. Assume to the contrary that this is not the case, that is, there
  exist elements $y,y'\in Y$ distinct such that $yy'\not\in \cL$. Since
  $\Gamma(\cL|_Y)$ is connected, there must exist a path
  $P:x_1=y,x_2,\ldots,x_l=y'$, $l\geq 2$, in $\Gamma(\cL|_Y)$ from $y$ to
  $y'$. Since the vertex set of $\Gamma(\cL|_Y)$ is $Y$, it follows that
  $X'=\{x_1,x_2,\ldots,x_l\}\subseteq Y$. Combined with the fact that
  $lca_T(x,x')=lca_T(Y)$ holds for all $x,x'\in X'$ distinct as $T|_Y$ is a
  star tree on $Y$, we obtain $X'\subseteq V(B_{lca_T(Y)})$.  Thus, $yy'\in
  \cL$ which is impossible and thus proves the claim.  That $\cL|_Y$ is a
  distinguished minimal topological lasso for $T|_Y$ is a trivial consequence.

  So assume that $T|_Y$ is non-degenerate. Since $\cL$ is a distinguished
  minimal topological lasso for $T$, Theorem~\ref{theo:characterization}
  implies that there exists a total ordering $\omega$ of $X$ and a cluster
  marker map $f_{\omega}: cl(T)\to X$ for $T$ and $\omega$ such that
  $\cL=\cL_{(T,f_{\omega})}$. Moreover, Lemma~\ref{lem:insights}(iv) implies
  that the cut-vertices of $\Gamma(\cL)$ are of the form $f_{\omega}(L_T(v))$
  where $v\in \iV(T)$.

  To see that $\cL|_Y$ is a distinguished minimal topological lasso for $T|_Y$
  and some total ordering of $Y$ note first that the restriction $\sigma$ of
  $\omega$ to $Y$ induces a total ordering of $Y$.  Furthermore, the
  aforementioned form of the cut-vertices of $\Gamma(\cL)$ combined with the
  assumption that $\Gamma(\cL|_Y)$ is connected implies that, for all $A\in
  cl(T)$ with $A\cap Y\not=\emptyset$, we must have $f_{\omega}(A)\in Y$. For
  all $A\in cl(T|_Y)$ denote by $A^T$ the set-inclusion minimal superset of
  $A$ contained in $cl(T)$.  Then since $f_{\omega}$ is a cluster marker map
  for $T$ and $\omega$ it follows that the map
$$
f_{\sigma}:cl(T|_Y)\to Y\,:\, A\mapsto f_{\omega}(A^T)
$$
is a cluster marker map for $T|_Y$ and $\sigma$.  By
Theorem~\ref{theo:characterization} it now suffices to establish that
$\cL|_Y=\cL_{(T|_Y,f_{\sigma})}$. Since both $\cL|_Y$ and
$\cL_{(T|_Y,f_{\sigma})}$ are minimal topological lassos for $T|_Y$ and so
$|\cL|_Y|=|\cL_{(T|_Y,f_{\sigma})}|$ is implied by Lemma~\ref{lem:size-A(v)}
it suffices to show that $\cL|_Y\subseteq \cL_{(T|_Y,f_{\sigma})}$.

Suppose $ab\in \cL|_Y$, that is, $ab\in\cL$ and $a,b\in Y$. Since $Y$ is the
leaf set of $T|_Y$, there must exist a vertex $v\in \iV(T|_Y)$ such that
$v=lca_{T|_Y}(a,b)$. Clearly, $v\in \iV(T)$. If $a$ and $b$ are both adjacent
with $v$ in $T$ then $a$ and $b$ are also adjacent with $v$ in $T|_Y$. Thus
$ab\in\cL_{(T|_Y,f_{\sigma})}(v)$ in this case.  So assume that at least one
of $a$ and $b$ is not adjacent with $v$ in $T$.  Without loss of generality
let $a$ denote that vertex.  Then since $ab\in \cL= \cL_{(T,f_{\omega})}$, it
follows that there must exist a unique child $v'\in \iV(T)$ of $v$ such that
$a\in L_T(v')$ and $a=f_{\omega}(L_T(v'))$. Hence, $a\in V(B_v)$ and a
cut-vertex of $\Gamma(\cL)$.

We claim that $v'\in \iV(T|_Y)$. Assume for contradiction that $v'\not\in
\iV(T|_Y)$. Then since $f_{\omega}$ is a cluster marker map for $T$ and and
$\omega$, it follows that $a$ cannot be a cut vertex in $\Gamma(\cL|_Y)$.
Since $\Gamma(\cL)$ is a claw-free block graph, no edge in the unique block
$B'\in Block(\Gamma(\cL))-\{B_v\}$ that also contains $a$ in its vertex set
can therefore be incident with $a$ in $\Gamma(\cL|_Y)$. Since $\Gamma(\cL|_Y)$
is assumed to be connected, to obtain the required contradiction it now
suffices to show that there exists some $c\in Y\cap L_T(v')$ distinct from $a$
such that every path from $c$ to $b$ in $\Gamma(\cL)$ crosses $a$.  But this
is a consequence of the facts that $v$ is not the parent of $a$ in $T|_Y$ and,
implied by Proposition~\ref{prop:gamma-l-connected}, that the subgraph
$\Gamma_{v'}(\cL)$ of $\Gamma(\cL)$ induced by $L_T(v')$ is the connected
component of $\Gamma(\cL)$ containing $a$ obtained from $\Gamma(\cL)$ by
deleting all edges in $B_v$ that are incident with $a$. This concludes the
proof of the claim

To conclude the proof of the theorem, note that if $b$ is adjacent with $v$ in
$T|_Y$ then $ab= f_{\omega}(L_T(v'))b=f_{\omega}((L_{T|_Y}(v'))^T)b=
f_{\sigma}(L_{T|_Y}(v'))b\in\cL_{(T|_Y,f_{\sigma})}(v) \subseteq
\cL_{(T|_Y,f_{\sigma})}$.  If $b$ is not adjacent with $v$ in $T|_Y$ then
there exists a child $v''\in \iV(T)$ of $v$ such that
$b=f_{\omega}(L_T(v''))$.  In view of the previous claim, we have $v''\in
\iV(T|_Y)$.  But now arguments similar to the ones used before imply that
%and so $b=f_{\omega}(L_T(v''))=f_{\omega}((L_{T|_Y}(v''))^T)
%=f_{\sigma}(L_{T|_Y}(v''))$. Thus,
%$ab=f_{\sigma}(L_{T|_Y}(v'))f_{\sigma}(L_{T|_Y}(v''))\in 
%\cL_{(T|_Y,f_{\sigma})}(v)\subseteq \cL_{(T|_Y,f_{\sigma})}$.
$ab\in \cL_{(T|_Y,f_{\sigma})}(v)\subseteq \cL_{(T|_Y,f_{\sigma})}$.
\end{proof}

We now turn our attention to supertrees which are formally defined as
follows. Suppose $\mathcal T=\{T_1,\ldots, T_l\}$, $l\geq 1$, is a set of
$Y_i$-trees $T_i$ with $Y_i\subseteq X$ and $|Y_i|\geq 3$, $i\in\langle
l\rangle$, and $T$ is an $X$-tree. Then $T$ is a called a {\em supertree} of
$\mathcal T $ if $T$ displays every tree in $\mathcal T$ where we say that
some $X$-tree $T$ {\em displays} some $Y$-tree $T'$ for $Y\subseteq X$ with
$|Y|\geq 3$ if $T|_Y$ and $T'$ are equivalent. More precisely, we have the
following result which relies on the fact that in case $\cL$ is a
distinguished minimal topological lasso for a {\em binary} $X$-tree $T$, that
is, every vertex of $T$ but the leaves has two children, $\Gamma(\cL)$ must be
a path. In particular, $\cL$ induces a total ordering of the elements in $X$
in this case.  For $Y\subseteq X$ a non-empty subset of $X$, we denote the
maximal and minimal element in $Y$ with regards to that ordering by
$\min_{\cL}(Y)$ and $\max_{\cL}(Y)$, respectively.

\begin{cor}
  \label{cor:supertree}
  Suppose $X'$ and $X''$ are two non-empty subsets of $X$ such that $X=X'\cup
  X''$ and $X'\cap X''\not=\emptyset$ and $T'$ and $T''$ are $X'$-trees and
  $X''$-tree, respectively. Suppose also that $\cL'$ and $\cL''$ are
  distinguished minimal topological lassos for $T'$ and $T''$, respectively,
  such that $\cL'|_{X'\cap X''} =\cL''|_{X'\cap X''}$ and
  $\Gamma(\cL''|_{X'\cap X''})$ is connected.
%and $\omega(\cL')$ and $\omega(\cL'')$ coincide on $X'\cap X''$.
  If $T$ is a binary $X$-tree that displays both $T'$ and $T''$ then
  $\cL=\cL'\cup\cL''$ is a distinguished minimal topological lasso for $T$ if
  and only if $\min_{\cL'}(X'\cap X'')\in \{\min_{\cL'}(X'),
  \min_{\cL''}(X'')\}$ and $\max_{\cL'}(X'\cap X'')\in \{\max_{\cL'}(X'),
  \max_{\cL''}(X'')\}$.
\end{cor}

Continuing with the assumptions of Corollary~\ref{cor:supertree}, we also have
that if $\min_{\cL'}(X'\cap X'')\in \{\min_{\cL'}(X'), \min_{\cL''}(X'')\}$
and $\max_{\cL'}(X'\cap X'')\in \{\max_{\cL'}(X'), \max_{\cL''}(X'')\}$ holds
then $\cL'\cup \cL''$ is a (minimal) strong lasso for $T$ as every minimal
topological lasso for an $X$-tree is also an equidistant lasso for that
tree. However, not all strong lassos for $T$ are of this form. An example for
this is furnished for $X'=\{a,c,d\}$ and $X''=\{a,b,c\}$ by the $X'$-tree
$T'$, the $X''$-tree $T''$ and the $X'\cup X''$-tree $T$ depicted in
Fig.~\ref{fig:supertree} along with the set $\cL'=\{cd\}$ and
$\cL''=\{ab,bc\}$ of cords of $X'$ and $X''$, respectively. Clearly, $T$ is a
supertree of $\{T',T''\}$ and $\cL=\cL'\cup\cL''$ is a strong lasso for $T$
but $\cL'$ is not even an equidistant lasso for $T'$. Investigating further
the interplay between minimal topological lassos for $X$-trees and minimal
topological lassos for supertrees that display them might therefore be of
interest.

\begin{figure}[h]
  \begin{center}
    \input{figures/dist-min-lass/supertree.pdft}
  \end{center}
  \caption{ For $X'=\{a,c,d\}$ and $X''=\{a,b,c\}$ the $X'\cup X''$-tree $T$
    is a supertree for the depicted $X'$ and $X''$ trees $T'$ and $T''$,
    respectively. Clearly, $\cL'=\{cd\}$ and $\cL''=\{ab,bc\}$ are sets of
    cords of $X'$ and $X''$, respectively, and $\cL=\cL'\cup\cL''$ is a strong
    lasso for $T$ but $\cL'$ is not even an equidistant lasso for $T'$.  }
  \label{fig:supertree}
\end{figure}

We conclude with returning to Fig.~\ref{fig:block-graph-motivation} which
depicts two non-equivalent $X$-trees that are topologically lassoed by the
same set $\cL$ of cords of $X$. In fact, $\cL$ is even a minimal topological
lasso for both of them.  Understanding better the relationship between
$X$-trees that are topologically lassoed by the same set of cords of $X$ might
also be of interest to study further.

\section{Constructing a tree from a distinguished lasso}
\label{sec:constr-tree-from}

Let $(T,\omega)$ be an equidistant $X$-tree and let $\cL$ be a distinguished
minimal topological lasso for $T$.  To the graph $\Gamma(\cL)$ we can assign
an edge-weight function $\alpha_{\cL} \colon E(\Gamma(\cL)) \to \mathbb{R}^+$
by letting $\alpha_{\cL}(xy) = D_{(T,\omega)}(x,y)$ for all edges $xy \in
E(\Gamma(\cL))$.

We next associate an edge-weighted tree $(\AcL,\BecL)$ to $\Gamma(\cL)$ by
essentially replacing every block of $\Gamma(\cL)$ by an edge-weighted star
tree.  More precisely, let $B$ be a block in $\Gamma(\cL)$.  We first choose
distinct vertices $a,b \in V(B)$ and put $m_B := \alpha_{\cL}(\{a,b\})$.  Note
that since $\omega$ is a equidistant proper edge-weighting for $T$ this
definition of $m_B$ is independent of the choice of $a$ and $b$.  Next we
delete all edges from $B$, add a vertex $s_B$ to $V(B)$ and new edges
$\{a,s_B\}$ for all $a \in V(B) - \{s_B\}$ to obtain a star tree $S_B$ with
leaf set $V(B)$.  Finally, we put $\beta_B \colon E(S_B) \to \mathbb{R}^+,
\beta_B(e) = m_B/2$ for all $e \in E(S_B)$.  Once this has been done for all
blocks in $\Gamma(\cL)$ we obtain the tree $\AcL$.  We define an
edge-weighting $\beta_{\cL} \colon E(\AcL) \to \mathbb{R}^+$ by putting
$\beta_{\cL}(e) = \beta_B(e)$ for the unique block $B \in \Block(\Gamma(\cL))$
with $e$ an edge in $B$.

Now we present an algorithm for constructing an edge-weighted $X$-tree
$(T,\omega)$ from $\AcL$ and $\BecL$ which we call \textsc{TreeConstruct}.
The algorithm builds a tree $T$ using the graph $\AcL$ and the function
$\BecL$ by traversing $\AcL$ and building the tree by adding internal vertices
and their children.  The pseudocode for the algorithm is shown in
Figure~\ref{algorithm:treeconstruct2}.

Note that we make use of the distance function $D_{(T,\omega)} \colon V
\times V \to \mathbb{R}^+$ that is induced by the edge-weight function
$\omega$ by letting, for all pairs $x,y \in V$, $D_{(T,\omega)}(x,y)
= \sum_{e \in P(x,y)} \omega(e)$ where $P(x,y)$ is the set of edges on the
path from $x$ to $y$.

\begin{algorithm}
  \caption{\textsc{Order}($\AcL,u,v,o$)}
  \label{algorithm:order}
  
  \begin{algorithmic}
    \Require Graph $\AcL$ derived from a distinguished minimal topological
    lasso $\cL$, vertices $u,v \in V(\AcL)$ and an $n$-tuple $o$.
    \Ensure An $(n+1)$-tuple $o^*$.

    \State Let $C_u := \{c \in V(\AcL) \colon cu \in \cL, c \neq v\}$.
    \State Put $o^* := (o,u)$.
    \ForAll{$v \in C_u$}
    \If{$\deg_{\AcL}(v) > 1$}
    \State Let $u^* \in v(\AcL)$ be the vertex such that $u^*v \in \cL$,
    \State Put $o^* := \text{\textsc{Order}}(\AcL, u^*, v, o^*)$,
    \State Put $o^* := (o^*, u)$.
    \EndIf
    \EndFor

    \State \Return $o^*$.
  \end{algorithmic}
\end{algorithm}

\begin{figure}
\centering
\parbox{0cm}{\begin{tabbing}
XX\= XX\= XX\=  XX\= XX\= XX\= XX\=  XXXXX\=  XXXXXXXXXXX\= \kill \\
{\textsc{TreeConstruct}($\AcL,\BecL$)} \\
\rule{\columnwidth}{0.5pt}\\
\textbf{Input:} \> \> \> Graph $\AcL$ derived from a distinguished minimal topological lasso $\cL$\\
                \> \> \> and an edge-weight function $\BecL \colon \cL \to \mathbb{R}_{>0}$.\\
\textbf{Output:} \> \> \> An $X$-tree $T = (V,E)$ with root $\rho$ and an
                          equidistant proper edge-\\
                 \> \> \> weighting $\omega \colon E \to
                          \mathbb{R}_{\geq 0}$.\\\\

\lnum{\phantom{0}1} \> Label all vertices in $\AcL$ as unprocessed.\\

\lnum{\phantom{0}2} \> Let $v$ be some degree 1 vertex in $\AcL$ and $u$ be the vertex
           adjacent to $v$.\\

\lnum{\phantom{0}3} \> Let $(u_1,u_2,\dotsc,u_n) := \text{\textsc{Order}}(\AcL, u, v, \emptyset)$.\\

\lnum{\phantom{0}4} \> Let $C_{u_1} := \{c \in V(\AcL) \colon \{c,u_1\} \in E(\AcL)\}$.\\
\lnum{\phantom{0}5} \> Put $T:=(V,E)$ with $V := \{u_1\} \cup C_{u_1}$, $E := \{\{u_1,c\} \colon c \in C_{u_1}\}$
              and $ \rho := u_1$.\\
\lnum{\phantom{0}6} \> Define a function $\omega \colon E \to \rr_{\geq 0}$ and let
              $\omega(\{u_1,c\}) := \BecL(\{u_1,v\})$ for all $c \in C_{u_1}$.\\
\lnum{\phantom{0}7} \> Label all $c \in C_{u_1}$ in $\AcL$ as processed.\\

\lnum{\phantom{0}8} \> \textbf{Foreach} $i \in (2,\dotsc,n)$ \textbf{do}:\\
\lnum{\phantom{0}9} \> \> \textbf{If} $u_i$ is already processed, \textbf{continue}
                 with $i := i+1$.  \textbf{EndIf.}\\
\lnum{10} \> \> Let $C_{u_i} := \{c \in V(\AcL) \colon \{c,u_i\} \in E(\AcL)$ and $c$
                 is unprocessed.$\}$\\
\lnum{11} \> \> Let $v \in V(\AcL)$ be the unique vertex such that both $\{u_i,v\}
                 \in E(\AcL)$\\
            \> \> \> and $\{v,u_{i-1}\} \in E(\AcL)$.\\
\lnum{12} \> \> \textbf{If} $\DTw(\rho,v) < \BecL(\{v,u_i\})$
                 \textbf{then}:\\
\lnum{13} \> \> \> Put $V^* := V \cup \{u_i\} \cup C_{u_i}$,\\
           \> \> \> \> $E^* := E \cup \{\{u_i,c\} \colon c \in C_{u_i}\} \cup
                       \{\{\rho,u_i\}\}$, and\\
           \> \> \> \> $\rho^* := u_i$.\\
\lnum{14} \> \> \> Define a function $\omega^* \colon E^* \to
                    \mathbb{R}_{\geq 0}$ and let $\omega^*(e) := \omega(e)$
                    for all $e \in E$,\\
           \> \> \> \> $\omega^*(\{u_i,c\}) := \BecL(\{u_i,v\})$ for all $c
                       \in C_{u_i}$, and\\
           \> \> \> \> $\omega^*(\{\rho,u_i\}) := \BecL(\{v,u_i\}) -
                       \DTw(v,\rho)$.\\

\lnum{15} \> \> \> Label all $\{u_i\} \cup C_{u_i}$ in $\AcL$ as processed and put
                    $\rho := \rho^*$.\\

\lnum{16} \> \> \textbf{Else}:\\
\lnum{17} \> \> \> Let $p,q \in V$ be the unique vertices such that $\{p,q\}
                    \in E$ and\\
           \> \> \> \> $\DTw(v,p) > \BecL(\{v,u_i\}) > \DTw(v,q)$.\\
           \> \> \> Put $V^*:= V \cup \{u_i\} \cup C_{u_i}$, and\\
           \> \> \> \>  $E^*:= E \cup \{\{u_i,c\} \colon c \in C_{u_i}\} \cup
                        \{\{p,u_i\},\{u_i,q\}\} - \{\{p,q\}\}$.\\
\lnum{18} \> \> \> Define a function $\omega^* \colon E^* \to
                    \mathbb{R}_{\geq 0}$ and let\\
           \> \> \> \> $\omega^*(e) := \omega(e)$ for all $e \in E - \{\{p,q\}\}$,\\
           \> \> \> \> $\omega^*(\{u_i,c\}) := \BecL(\{u_i,v\})$ for all $c \in
                       C_{u_i}$,\\
           \> \> \> \> $\omega^*(\{p,u_i\}):= \DTw(v,p) - \BecL(\{u_i,v\})$, and\\
           \> \> \> \> $\omega^*(\{u_i,q\}):= \BecL(\{u_i,v\}) - \DTw(v,q)$.\\

\lnum{19} \> \> \> Label all $\{u_i\} \cup C_{u_i}$ in $\AcL$ as processed.\\

\lnum{20} \> \> \textbf{EndIf.}\\

\lnum{21} \> \> Put $V:=V^*, E:=E^*$ and $\omega:=\omega^*$.\\

\lnum{22} \> \textbf{EndFor.}

         \end{tabbing}}
\caption{Pseudocode for \textsc{TreeConstruct}.}
\label{algorithm:treeconstruct2}
\end{figure}

We next show that the construction ConstructTree is correct.  So assume for
the remainder of this section that $T_O$ is an $X$-tree, that $\cL$ is a
distinguished minimal topological lasso for $T_O$ and that the distance $D_O(x,y)$
is known for all $xy \in \cL$.  Let $(T,\omega)$ be the $X$-tree $T$ with
equidistant proper edge-weighting $\omega \colon E(T) \to
\mathcal{R}_{\geq 0}$ returned by \textsc{ConstructTree}.

We begin with a lemma which shows that the two conditional cases in step 4 are
valid.  Specifically, we cannot have a case where
$D_{(T,\omega^*)}(v,\rho) = \beta(\{v,u^*\})$ so the algorithm continues to
run until all vertices in $\AcL$ have been exhausted.

\begin{lem}
  \label{lem:nonequal-adjacent}
  Let $B,B' \in Block(\Gamma(\cL))$ denote two distinct blocks of
  $\Gamma(\cL)$ such that $V(B) \cap V(B') \neq \emptyset$.  Then for all $e
  \in E(B)$ and all $e' \in E(B')$ we have $\BecL(e) \neq \BecL(e')$.
\end{lem}

\begin{proof}
  Assume for the purpose of contradiction that $\BecL(e) = \BecL(e')$.  Let
  $\alpha = \BecL(e)$, $\alpha' = \BecL(e')$ and $y \in V(B) \cap V(B')$.  Note
  that $y \in \Cut(\Gamma(\cL))$.  Choose some $a \in V(B) - \{y\}$ and $a'
  \in V(B') - \{y\}$ which must exist since the size of a block in
  $\Gamma(\cL)$ is at least two.  Then, on the one hand, our assumptions on
  $e$ and $e'$ combined with the definition of the edge-weighting $\omega$
  implies that the paths $P$ and $P'$ from $v = \lca_{T}(y,a)$ to $a$ and $v'
  = \lca_{T}(y,a')$ to $a'$, respectively, have the same length.  But on the
  other hand the choice of $y$ implies that either $v'$ is a descendant of $v$
  in $T$ or $v$ is a descendant of $v'$ in $T$ and so $P$ and $P'$ must have
  different lengths: a contradiction.
\end{proof}

\begin{lem}
  \label{lem:t-is-x-tree}
  $T$ is an $X$-tree and $\omega$ is a proper equidistant edge-weighting for
  $T$.
\end{lem}

\begin{proof}
  We proceed by induction on the number of times steps 2 and 3 are executed
  or, equivalently, the number of stars in $\AcL$ processed.

  The case is that $T$ is a $Y$-tree, where $Y \subseteq X$, following the
  first, and only, execution of step 2.  Since step 2 is simply setting $T$
  equal to a star from $\AcL$ it is clear that $T$ will be a $Y$-tree.
  Further, since the edge-weights are transferred directly from $\AcL$ and
  these are positive and equal for all edges within a star by definition,
  $\omega$ will be an equidistant proper edge-weighting.

  For the inductive step we assume that $T = (V,E)$ is a $Y$-tree and that
  $\omega$ is an equidistant proper edge-weighting prior to step 3 and we show
  that the tree $T^* = (V^*,E^*)$ obtained following step 3 is a $Y^*$-tree,
  where $Y^* \supset Y$ and that the edge-weighting $\omega^* \colon E(T^*)
  \to \rr_{>0}$ obtained is an equidistant proper edge-weighting.

  First we show that $T^*$ is a $Y^*$-tree.  Since $T$ is a tree it is
  connected and $|V| = |E| + 1$.  In step 3 there are two cases: either we add
  to $T$ one internal vertex $u$, a set of leaves $C_u$ and $|C_u|+1$ edges;
  or we replace an edge $\{u,v\}$ with a path $u,w,v$, where $w$ is a new
  vertex not present in $T$, add a set of leaves $C_w$ and $|C_w|$ edges.  In
  either case it is clear that $T^*$ is connected and it remains that
  $|V^*| = |E^*| + 1$ so $T^*$ is a $Y^*$-tree.

  Now we show that $\omega^*$ is an equidistant proper edge-weighting.  Again
  there are two cases in step 3 which we deal with separately.  In the first
  case we have that $D_{(T,\omega)}(v,\rho) < \BecL(\{v,u^*\})$ so
  $\BecL(\{v,u^*\}) - D_{(T,\omega)}(v,\rho)$ is positive.  All other values
  for $\omega^*$ come from either $\omega$ or $\BecL$ which are positive
  functions, so $\omega^*$ is a proper edge-weighting.  To see that it is
  equidistant note that both $T$ and all $c \in C_{u^*}$ become children of
  $u^*$.  Since $T$ is equidistant, $D_{(T,\omega)}(v,\rho)$ is the distance
  between all leaves of $T$ and $\rho$, therefore the distance between all
  leaves of $T^*$ and $\rho^*$ becomes $\BecL(\{v,u^*\})$, so $\omega^*$ is
  equidistant.

  In the second case we have that $D_{(T,\omega)}(v,\rho) > \BecL(\{v,u^*\})$
  and also the inequality $D_{(T,\omega)}(v,p) > \BecL(\{v,u^*\}) >
  D_{(T,\omega)}(v,q)$.  It is clear then that all calculated values for
  $\omega^*$ will be positive and, as before, the remainder come from either
  $\omega$ or $\BecL$, therefore $\omega^*$ is proper.  Similar to the first
  case, the children of $u^*$ will be the leaves $C_{u^*}$ and a tree with $q$
  as its root.  This tree will be ultrametric since it is a subtree of $T$ so
  the distance between $q$ and all of its leaf descendants will be
  $D_{(T,\omega)}(v,q)$.  It follows that the distance between $u^*$ and these
  leaves will be $\omega^*(\{u^*,q\}) + D_{(T,\omega)}(v,q) = \BecL(\{u^*,v\})
  - D_{(T,\omega)}(v,q) + D_{(T,\omega)}(v,q) = \BecL(\{u^*,v\})$ and this is
  equal to the edge-weight set for $\omega^*(\{u^*,c\})$ for all $c \in
  C_{u^*}$.  Now since the distance between $p$ and all of its leaf
  descendants in $T$ is $D_{(T,\omega)}(v,p)$ and we are setting
  $\omega^*(\{p,u^*\}) := D_{(T,\omega)}(v,p) - \BecL(\{u^*,v\})$, the
  distance between $p$ and its leaf descendants in $T^*$ remains the same,
  therefore $\omega^*$ is equidistant.

  Finally, it is clear that the algorithm will terminate after all stars in
  $\AcL$ have been exhausted and since each unprocessed leaf in $\AcL$ is
  processed for every star we will have that $T^*$ is a $Y^*$-tree with $Y^* =
  X$ following the final execution of either step 2 or 3.
\end{proof}

\begin{lem}
  \label{lem:t-t-star-equal-dists}
  For each cord $xy \in \cL$ we have that $D_{(T,\omega)}(x,y) = D_O(x,y)$.
\end{lem}

\begin{proof}
  By definition, a star with centre $u$ and leaves $v_1,\dotsc,v_k$ is present
  in $\AcL$ if and only if the cords $\{v_iv_j \colon 1 \leq i < j \leq k\}$
  are present in $\cL$.  Also by definition $\beta(\{u_i,v_i\}) =
  D_{(T,\omega)}(v_i,v_j)/2$ for all $1 \leq i,j \leq k$ and $i \neq j$.
  Whenever an internal vertex $u^*$ is added in either steps 2 or 3 we also
  attach the set $C_{u^*}$ of leaves to it.  It is ensured initially that
  $\omega^*(\{u^*,v\}) = \beta(\{u^*,v\})$ for all $v \in C_{u^*}$.  Any of
  these edges $\{u,v\}$ may later be removed and replaced by a path $u,w,v$ by
  subsequent executions of step 3, but the length of the path $u,w,v$ is kept
  equal to the original length of $\{u,v\}$ each time.  Therefore
  $D_{(T,\omega)}(x,y) = D_O(x,y)$ for all $x,y \in \cL$.
\end{proof}

As a result of Lemmata~\ref{lem:t-is-x-tree} and
\ref{lem:t-t-star-equal-dists} and the definitions of a topological lasso we
obtain the following theorem:

\begin{thm}
  \label{thm:t-t-star-l-isometric}
  $(T,\omega)$ and $(T^*,\omega^*)$ are $\cL$-isometric.  In particular, $T$
  and $T^*$ are equivalent.
\end{thm}

\section{Conclusion}
\label{sec:conclusion-dist}



%%% Local Variables:
%%% TeX-master: "thesis"
%%% End:
