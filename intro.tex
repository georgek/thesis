\chapter{Introduction}
\label{cha:introduction}

Metric spaces are a generalisation of the world in which we live where
physical objects appear to exist in 3-dimensional space and we have a notion
of the distance between pairs of objects.  The distance that many people think
of is the Euclidean distance, the length of the straight line between the
objects that one would measure with a ruler.  Others may be more familiar with
the Manhattan distance, the distance one must travel between locations by
traversing the grid-like streets of a city.  These distances share a few key
properties; these are the properties of a metric.  We therefore live in a
metric space, or at least a close approximation of one.

Increasingly, our world is becoming filled with another, more abstract type of
object that ``lives'' outside of the physical space: data.  Regardless of the
form that data takes, be it numbers, text or pictures, the presence of data
always comes with the need for a method of comparison.  Since we are most at
ease with thinking about our own world, it seems natural that we should
imagine data points as ``living'' in a space with a distance defined which
holds the same key properties as the distances we use every day.  Thus, the
data points become members of their own metric space.

There are a number of problems that present themselves when we have a metric
space.  The problem of identifying groups of similar objects based on their
relative proximity is called clustering.  The problem is very old, ubiquitous
and has a rich and diverse set of applications.  Despite the fact that
intelligent beings seem to be naturally adept at the task, it is a well-known
hard problem for a computer, and in more than one sense.  The first difficulty
is in defining precisely what one actually expects in terms of a metric,
rather than the vague objective of ``groups of similar objects''.  The second
difficulty is in the sense of computational complexity.  In the first part of
this thesis we focus mainly on the first difficulty in the context of
partitional clustering, which seeks to partition a set into a given number of
subsets.  Many objective criteria have been developed to measure the
usefulness of a partition as a solution to a given clustering problem, but we
show that even seemingly similar optimisation criteria can produce vastly
different results (see Section~\ref{sec:worst-case-perf}).

In order to show that criteria can produce vastly different partitions it is
necessary to be able to compare partitions.  Many metrics have been devised
for that purpose already.  We find ourselves with several layers of metric
spaces and we introduce the concept of ``lifting'' a metric space to the space
of its power set.  In this way we introduce a new metric for comparing
partitions that can take into account the fact that the data lives in a metric
space (see Section~\ref{sec:assignment-metric}).

Metrics present themselves in the context of that most well-known and useful
data structure: the tree.

In a rooted edge-weighted tree with leaf set $X$, the graph-theoretic distance
between the leaves is a metric.  It is well-known that such a tree can be
reconstructed uniquely from a distance on the elements of $X$ that is a
stronger version of a metric: an ultrametric.  But in practice, we may not
always have access to the complete matrix of distances.  The problem arises of
constructing trees from only an incomplete matrix of distances.  In the second
part of this thesis we investigate the circumstances under which such an
incomplete matrix uniquely determines a rooted edge-weighted tree.  We call
this ``lassoing'' the tree.  We then look at the circumstances under which a
supertree can be constructed from two or more lassoed subtrees.

We next present an outline of this thesis. In Chapter~\ref{cha:background} we
introduce the concept of metric spaces and review various metrics that have
been used over the years.  We then introduce partitions and review metrics for
comparing partitions.  This leads to a review of partitional clustering, the
problem of finding partitions of a dataset.  Finally, we introduce trees and
their applications.  In Chapter~\ref{cha:sum-squar-clust} we focus on
partitional clustering and its difficulties by comparing and contrasting two
closely related partitional clustering criteria, called sum-of-squares
criteria.  We show that the two criteria can produce completely different
results.  Chapter~\ref{cha:background2} concerns the problem of
reconstructing rooted edge-weighted trees from subtrees for which only partial
distance information is available.


%%% Local Variables:
%%% TeX-master: "thesis"
%%% End:
