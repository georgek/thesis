\chapter{Introduction}
\label{cha:introduction}

\pagenumbering{arabic}

Metric spaces arise in a wide variety of contexts.  Whenever we are presented
with information, one of the first things we would like to do is make
comparisons.  A metric supplies us with a distance between objects that fits
our intuition about distances.  This is probably because we live in a metric
space---or at least a good approximation of one!

There are a number of problems that present themselves when we have a metric
space.  The problem of identifying groups of similar objects based on their
relative proximity is called cluster analysis.  The problem is very old,
ubiquitous and has a rich and diverse set of applications.  Despite the fact
that intelligent beings seem to be naturally adept at the task, it is a
well-known hard problem for a computer, and in more than one sense.  The first
difficulty is in defining precisely what one actually expects in terms of a
metric, rather than the vague objective of ``groups of similar objects''.  The
second difficulty is in the sense of computational complexity.  Indeed, as we
will see, efficient generation of exact solutions to the problem is unlikely
to be possible in general.

Metrics present themselves in the context of that most well-known and useful
data structure: the tree.  It is well-established that an edge-weighted tree
is essentially defined by the distances between its leaf nodes.  But in
practice, we may not always have access to the complete matrix of distances.
An interesting problem arises in constructing trees from only a sparse
distance matrix.  We investigate the circumstances under which a sparse matrix
uniquely defines an edge-weighted tree.  We say that such a matrix ``lassos''
the tree.  We focus on a special type of commonly occurring tree that has
distances between leaves satisfying a stronger version of a metric: an
ultrametric.  A further problem is then the construction of ultrametric
supertrees from two or more lassoed subtrees.

In Chapter~\ref{cha:background} we introduce the concept of metric spaces and
review various metrics that have been used over the years.  We then introduce
partitions and review metrics for comparing partitions.  This leads to a
review of partitional clustering, the problem of finding partitions of a
dataset.  Finally, we introduce trees and their applications.  In
Chapter~\ref{cha:sum-squar-clust} we compare and contrast two closely related
partitional clustering criteria, called sum-of-squares criteria, and in the
process introduce a new metric for comparing partitions.
Chapter~\ref{cha:constr-ultr-supertr} looks at metrics from the perspective of
trees and considers the problem of constructing supertrees from partial
distance matrices.



%%% Local Variables:
%%% TeX-master: "thesis"
%%% End:
