\chapter{Introduction}
\label{cha:introduction}

Metric spaces are a generalisation of the world in which we live where
physical objects appear to exist in 3-dimensional space and we have a notion
of the distance between pairs of objects.  The distance that many people think
of is the Euclidean distance, the length of the straight line between the
objects that one would measure with a ruler.  Others may be more familiar with
the Manhattan distance, the distance one must travel between locations by
traversing the grid-like streets of a city.  These distances share a few key
properties; these are the properties of a metric.  We therefore live in a
metric space, or at least a close approximation of one.

Increasingly, our world is becoming filled with another, more abstract type of
object that ``lives'' outside of the physical space: data.  Regardless of the
form that data takes, be it numbers, text or pictures, the presence of data
always comes with the need for a method of comparison.  Since we are most at
ease with thinking about our own world, it seems natural that we should
imagine data points as ``living'' in a space with a distance defined which
holds the same key properties as the distances we use every day.  Thus, the
data points become members of their own metric space.

There are a number of problems that present themselves when we have a metric
space.  The problem of identifying groups of similar objects based on their
relative proximity is called clustering.  The problem is very old, ubiquitous
and has a rich and diverse set of applications.  Despite the fact that
intelligent beings seem to be naturally adept at the task, it is a well-known
hard problem for a computer, and in more than one sense.  The first difficulty
is in defining precisely what one actually expects in terms of a metric,
rather than the vague objective of ``groups of similar objects''.  The second
difficulty is in the sense of computational complexity.  In the first part of
this thesis we focus mainly on the first difficulty in the context of
partitional clustering, which seeks to partition a set into a given number of
subsets.  Many objective criteria have been developed to measure the
usefulness of a partition as a solution to a given clustering problem, but we
show that even seemingly similar optimisation criteria can produce vastly
different results (see Section~\ref{sec:worst-case-perf}).

In order to show that criteria can produce vastly different partitions it is
necessary to be able to compare partitions.  Many metrics have been devised
for that purpose already.  We find ourselves with several layers of metric
spaces and we introduce the concept of ``lifting'' a metric space to the space
of its power set.  In this way we introduce a new metric for comparing
partitions that can take into account the fact that the data lives in a metric
space (see Section~\ref{sec:lifting-metric-space}).

Metrics present themselves in the context of that most well-known and useful
data structure: the tree.  In a rooted edge-weighted tree with leaf set $X$,
the graph-theoretic distance between the leaves is a metric called a tree
metric.  It is well known that a tree is uniquely defined by and can be
reconstructed from its tree metric.  Tree construction is appropriate and
desirable in many areas such as in evolutionary biology.  For example, if we
had a dataset corresponding to varieties of some organism, we could find a
partitional clustering in which each cluster corresponds to a continent,
reflecting the expected result that organisms on the same continent are more
closely related, and those on different continents are more distantly related.
However, a tree can show us not only this information but the relationships
and lineages of each variety in our set.  In this way, trees can be considered
hierarchical clusterings and a step up from partitional clusterings.

In practice, though, we may not always have access to the full tree metric.
Rather we might be presented with a partial distance between pairs of leaves,
that is where the distance between some pairs elements is unknown.  Studying
the properties of partial distances and their ability to uniquely identify a
tree has given rise to the theory of ``lassoing'' a tree.  It turns out that
certain subsets of all pairwise distances do indeed contain enough information
to uniquely identify either the topology of a tree, its edge weights, or
both.  This theory is reviewed in Section~\ref{sec:lassoing-corralling}.

We therefore investigate the problem of constructing an edge-weighted tree
from a given partial distance.  We begin by investigating the properties of a
special type of lasso and then we present an algorithm for constructing the
unique tree which corresponds to such a lasso.  We then turn to the more
general problem and present an algorithm which constructs a tree from a
partial distance and returns the set of given distances which uniquely
determine the constructed tree.

This thesis is organised as follows: in Chapter~\ref{cha:background} we
introduce the concept of metric spaces and review various metrics that have
been used over the years.  We then introduce partitions and review metrics for
comparing partitions.  This leads to a review of partitional clustering, the
problem of finding partitions of a dataset.  Finally, we introduce trees and
their applications.  In Chapter~\ref{cha:sum-squar-clust} we focus on
partitional clustering and its difficulties by comparing and contrasting two
closely related partitional clustering criteria, called sum-of-squares
criteria.  We show that the two criteria can produce completely different
results.  In Chapter~\ref{cha:background2} we introduce trees, tree metrics,
tree construction and lassos.  In Chapter~\ref{cha:dist-minim-topl} we
investigate the properties of a special kind of lasso which we call a
distinguished minimal topological lasso and present an algorithm for tree
construction from such a lasso.  In Chapter~\ref{cha:lasso-construction} we
present a general algorithm for tree reconstruction from partial distances
which is consistent according to the theory of lassos.  Finally we conclude
the thesis and suggest further work in Chapter~\ref{cha:conclusion}.

%%% Local Variables:
%%% TeX-master: "thesis"
%%% End:
